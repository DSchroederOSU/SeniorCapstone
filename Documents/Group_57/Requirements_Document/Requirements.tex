\documentclass[onecolumn, draftclsnofoot,10pt, compsoc]{IEEEtran}
    \usepackage{graphicx}
    \usepackage{url}
    \usepackage{setspace}
    
    \usepackage[margin=0.75in]{geometry}
    \geometry{textheight=9.5in, textwidth=7in}
    
    % 1. Fill in these details
    \def \CapstoneTeamName{		The Dream Team}
    \def \CapstoneTeamNumber{		57}
    \def \GroupMemberOne{			Daniel Schroeder}
    \def \GroupMemberTwo{			Aubrey Thenell}
    \def \GroupMemberThree{			Parker Bruni}
    \def \CapstoneProjectName{		A Scalable Web Application Framework for Monitoring Energy Usage on Campus  }
    \def \CapstoneSponsorCompany{	Oregon State Office of Sustainability}
    \def \CapstoneSponsorPerson{		Jack Woods}
    
    % 2. Uncomment the appropriate line below so that the document type works
    \def \DocType{		%Problem Statement
            Requirements Document
            %Technology Review
            %Design Document
            %Progress Report
            }
          
    \newcommand{\NameSigPair}[1]{\par
    \makebox[2.75in][r]{#1} \hfil 	\makebox[3.25in]{\makebox[2.25in]{\hrulefill} \hfill		\makebox[.75in]{\hrulefill}}
    \par\vspace{-12pt} \textit{\tiny\noindent
    \makebox[2.75in]{} \hfil		\makebox[3.25in]{\makebox[2.25in][r]{Signature} \hfill	\makebox[.75in][r]{Date}}}}
    % 3. If the document is not to be signed, uncomment the RENEWcommand below
    %\renewcommand{\NameSigPair}[1]{#1}
    
    %%%%%%%%%%%%%%%%%%%%%%%%%%%%%%%%%%%%%%%
    \begin{document}
    \begin{titlepage}
        \pagenumbering{gobble}
        \begin{singlespace}
          \includegraphics[height=4cm]{coe_v_spot1.eps}
            \hfill 
            % 4. If you have a logo, use this includegraphics command to put it on the coversheet.
            %\includegraphics[height=4cm]{CompanyLogo}   
            \par\vspace{.2in}
            \centering
            \scshape{
                \huge CS Capstone \DocType \par
                {\large\today}\par
                \vspace{.5in}
                \textbf{\Huge\CapstoneProjectName}\par
                \vfill
                {\large Prepared for}\par
                \Huge \CapstoneSponsorCompany\par
                \vspace{5pt}
                {\Large\NameSigPair{\CapstoneSponsorPerson}\par}
                {\large Prepared by }\par
                Group\CapstoneTeamNumber\par
                % 5. comment out the line below this one if you do not wish to name your team
                \CapstoneTeamName\par 
                \vspace{5pt}
                {\Large
                    \NameSigPair{\GroupMemberOne}\par
                    \NameSigPair{\GroupMemberTwo}\par
                    \NameSigPair{\GroupMemberThree}\par
                }
                \vspace{20pt}
            }
            \begin{abstract}
            % 6. Fill in your abstract  
            This document outlines the requirements set forth by The Office of Sustainability as to what should be included in the final product of our project.
            
            \end{abstract}     
        \end{singlespace}
    \end{titlepage}
    \newpage
    \pagenumbering{arabic}
    \tableofcontents
    % 7. uncomment this (if applicable). Consider adding a page break.
    %\listoffigures
    %\listoftables
    \clearpage
    
    % 8. now you write!
    \section{Introduction}
    \subsection{Purpose}
    \subsection{Scope}
    \subsection{Definitions, acronyms, and abbreviations}
    \subsection{References}
    \subsection{Overview}
    \section{Overall description}
    \subsection{Product perspective}
    \subsection{Product functions}
    \subsection{User characteristics}
    \subsection{Constraints}
    \subsection{Assumptions and dependencies}
    \section{Specific requirements}
    (See 5.3.1 through 5.3.8 for explanations of possible
    specific requirements. See also Annex A for several different ways of organizing
    this section of the SRS.)
    \section{Appendixes}
    \section{Index}
    
    \end{document}