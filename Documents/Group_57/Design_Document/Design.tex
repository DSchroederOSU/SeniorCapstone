\documentclass[journal,10pt,onecolumn,compsoc]{IEEEtran} 
    \usepackage[margin=0.75in]{geometry} 
    \usepackage{pdfpages} 
    \usepackage{graphicx} 
    \setlength{\parskip}{\baselineskip} \setlength\parindent{24pt} 
    
    \usepackage[english]{babel}
    \usepackage{graphicx}
    \usepackage{url}
    \usepackage{setspace}
    \usepackage{geometry}
    \usepackage{hyperref}
    \usepackage{float}
    \usepackage{pgfgantt}
    \usepackage{rotating} 
    \usepackage{ragged2e} % provides \RaggedLeft
    \hypersetup{
        colorlinks,
        citecolor=black,
        filecolor=black,
        linkcolor=black,
        urlcolor=black
    }
    \geometry{textheight=9.5in, textwidth=7in}
    
    % 1. Fill in these details
    \def \CapstoneTeamName{		The Dream Team}
    \def \CapstoneTeamNumber{		57}
    \def \GroupMemberOne{			Daniel Schroeder}
    \def \GroupMemberTwo{			Aubrey Thenell}
    \def \GroupMemberThree{			Parker Bruni}
    \def \CapstoneProjectName{		A Scalable Web Application Framework for Monitoring Energy Usage on Campus  }
    \def \CapstoneSponsorCompany{	Oregon State Office of Sustainability}
    \def \CapstoneSponsorPerson{		Jack Woods}
    
    % 2. Uncomment the appropriate line below so that the document type works
    \def \DocType{		%Problem Statement
            %Requirements Document
            %Technology Review
            Design Document
            %Progress Report
            }
          
    \newcommand{\NameSigPair}[1]{\par
    \makebox[2.75in][r]{#1} \hfil 	\makebox[3.25in]{\makebox[2.25in]{\hrulefill} \hfill		\makebox[.75in]{\hrulefill}}
    \par\vspace{-12pt} \textit{\tiny\noindent
    \makebox[2.75in]{} \hfil		\makebox[3.25in]{\makebox[2.25in][r]{Signature} \hfill	\makebox[.75in][r]{Date}}}}
    % 3. If the document is not to be signed, uncomment the RENEWcommand below
    %\renewcommand{\NameSigPair}[1]{#1}
    
    %%%%%%%%%%%%%%%%%%%%%%%%%%%%%%%%%%%%%%%
    \title{Design Document for: \linebreak Scalable Web Application Framework for Monitoring Energy Usage on Campus}
    \author{Daniel Schroeder, Aubrey Thenell, Parker Bruni}
    \date{\today}
    
    \begin{document}
    \maketitle
    \vspace{2cm}
    \noindent \textbf{Abstract} \\
                \indent 
                Data visualization is an important technique for analyzing data trends and identifying problems that may not have been apparent. This document outlines our plan to create a web application which will gather energy data from data aquisition servers and create information management dashboards. Our application will allow users to generate charts and graphs to see energy use over time, compare energy usage across multiple buildings, and create collections of different visualizations.
                
                
    
    \newpage
    \pagenumbering{arabic}
    \tableofcontents
    % 7. uncomment this (if applicable). Consider adding a page break.
    %\listoffigures
    %\listoftables
    \clearpage
    
    % 8. now you write!
    \section{Introduction}
    \subsection{Purpose}
    Identify the purpose of this SDD and its intended audience. (e.g. “This software
    design document describes the architecture and system design of XX. ….”).
    \subsection{Scope}
    Provide a description and scope of the software and explain the goals, objectives and
    benefits of your project. This will provide the basis for the brief description of your
    product.
    \subsection{Overview}
    Provide an overview of this document and its organization.
    \subsection{Reference Material}
    List any documents, if any, which were used as sources of information for the test plan.
    \subsection{Definitions and Acronyms}
    Provide definitions of all terms, acronyms, and abbreviations that might exist to
    properly interpret the SDD. These definitions should be items used in the SDD that
    are most likely not known to the audience. 
    \section{System Overview}
    Give a general description of the functionality, context and design of your project.
    Provide any background information if necessary. 
    \section{System Architecture}
    \subsection{Architectural Design}
    Develop a modular program structure and explain the relationships between the
    modules to achieve the complete functionality of the system. This is a high level
    overview of how responsibilities of the system were partitioned and then assigned to
    subsystems. Identify each high level subsystem and the roles or responsibilities
    assigned to it. Describe how these subsystems collaborate with each other in order to
    achieve the desired functionality. Don’t go into too much detail about the individual
    subsystems. The main purpose is to gain a general understanding of how and why
    the system was decomposed, and how the individual parts work together. Provide a
    diagram showing the major subsystems and data repositories and their
    interconnections. Describe the diagram if required. 
    \subsection{Decomposition Description}
    Provide a decomposition of the subsystems in the architectural design. Supplement
    with text as needed. You may choose to give a functional description or an object
    oriented description.
    For a functional description, put top level data flow diagram (DFD) and structural
    decomposition diagrams.
    For an OO description, put subsystem model, object diagrams, generalization
    hierarchy diagram(s) (if any), aggregation hierarchy diagram(s) (if any), interface
    specifications, and sequence diagrams here.
    \subsection{Design Rationale}
    Discuss the rationale for selecting the architecture described in 3.1 including critical
    issues and trade/offs that was considered. You may discuss other architectures that
    were considered, provided that you explain why you didn’t choose them. 
    \section{Data Design}
    \subsection{Data Description}
    Explain how the information domain of your system is transformed into data
    structures. Describe how the major data or system entities are stored, processed and
    organized. List any databases or data storage items.
    \subsection{Data Dictionary}
    Alphabetically list the system entities or major data along with their types and
    descriptions. If you provided a functional description in Section 3.2, list all the
    functions and function parameters. If you provided an OO description, list the objects
    and its attributes, methods and method parameters.
    \section{Component Design}
    In this section, we take a closer look at what each component does in a more
    systematic way. If you gave a functional description in section 3.2, provide a
    summary of your algorithm for each function listed in 3.2 in procedural description
    language (PDL) or pseudo code. If you gave an OO description, summarize each
    object member function for all the objects listed in 3.2 in PDL or pseudocode.
    Describe any local data when necessary. 
    \section{Human Interface Design}
    \subsection{Overview of User Interface}
    Describe the functionality of the system from the user’s perspective. Explain how the
    user will be able to use your system to complete all the expected features and the
    feedback information that will be displayed for the user.
    \subsection{Screen Images}
    Display screenshots showing the interface from the user’s perspective. These can be
    hand drawn or you can use an automated drawing tool. Just make them as accurate
    as possible. (Graph paper works well.) 
    \section{Requirements Matrix}
    Provide a cross--reference that traces components and data structures to the
    requirements in your SRS document.
    Use a tabular format to show which system components satisfy each of the functional
    requirements from the SRS. Refer to the functional requirements by the
    numbers/codes that you gave them in the SRS. 
    \section{Appendixes}
    Appendices may be included, either directly or by reference, to provide supporting details
    that could aid in the understanding of the Software Design Document. 
    %\section{Index}
    
    \end{document}