\documentclass[journal,10pt,onecolumn,compsoc]{IEEEtran} 
    \usepackage[margin=0.75in]{geometry} 
    \usepackage{pdfpages} 
    \usepackage{graphicx}  
    \graphicspath{/images}

    \usepackage{tikz}
    \usetikzlibrary{shapes,arrows}
    \tikzstyle{block} = [rectangle, draw, fill=blue!20, 
    text width=5.5em, text centered, rounded corners, minimum height=4em]
    \tikzstyle{subblock} = [rectangle, draw, fill=blue!20, 
    text width=7em, text centered, rounded corners, minimum height=2em]
    \tikzstyle{datablock} = [rectangle, draw, fill=red!20, 
    text width=7em, text centered, rounded corners, minimum height=2em]
    \tikzstyle{line} = [draw, -latex']  
   
    \newenvironment{paddedtikzpicture}{\vspace*{0.5cm}
    \begin{center}\begin{tikzpicture}}
    {\
    \end{tikzpicture}\end{center}
    } 
    \usepackage[english]{babel} 
    \setlength{\parskip}{\baselineskip} \setlength\parindent{24pt}
    \usepackage{url}
    \usepackage{setspace}
    \usepackage{geometry}
    \usepackage{hyperref}
    \usepackage{caption} 
    \usepackage{float}
    \usepackage{pgfgantt}
    \usepackage{rotating} 
    \usepackage{ragged2e} % provides \RaggedLeft
    \hypersetup{
        colorlinks,
        citecolor=black,
        filecolor=black,
        linkcolor=black,
        urlcolor=black
    }
    \geometry{textheight=9.5in, textwidth=7in}
    
    % 1. Fill in these details
    \def \CapstoneTeamName{		The Dream Team}
    \def \CapstoneTeamNumber{		57}
    \def \GroupMemberOne{			Daniel Schroeder}
    \def \GroupMemberTwo{			Aubrey Thenell}
    \def \GroupMemberThree{			Parker Bruni}
    \def \CapstoneProjectName{		A Scalable Web Application Framework for Monitoring Energy Usage on Campus  }
    \def \CapstoneSponsorCompany{	Oregon State Office of Sustainability}
    \def \CapstoneSponsorPerson{		Jack Woods}
    
    % 2. Uncomment the appropriate line below so that the document type works
    \def \DocType{		%Problem Statement
            %Requirements Document
            %Technology Review
            Design Document
            %Progress Report
            }
          
    \newcommand{\NameSigPair}[1]{\par
    \makebox[2.75in][r]{#1} \hfil 	\makebox[3.25in]{\makebox[2.25in]{\hrulefill} \hfill		\makebox[.75in]{\hrulefill}}
    \par\vspace{-12pt} \textit{\tiny\noindent
    \makebox[2.75in]{} \hfil		\makebox[3.25in]{\makebox[2.25in][r]{Signature} \hfill	\makebox[.75in][r]{Date}}}}
    % 3. If the document is not to be signed, uncomment the RENEWcommand below
    %\renewcommand{\NameSigPair}[1]{#1}
    
    %%%%%%%%%%%%%%%%%%%%%%%%%%%%%%%%%%%%%%%
    \title{Design Document for: \linebreak Scalable Web Application Framework for Monitoring Energy Usage on Campus}
    \author{Daniel Schroeder, Aubrey Thenell, Parker Bruni}
    \date{\today}
    
    \begin{document}
    \maketitle
    \vspace{2cm}
    \begin{center}
    \noindent \textbf{Abstract} \\
                \indent 
                
                
    \end{center}         
    
    \newpage
    \pagenumbering{arabic}
    \tableofcontents
    % 7. uncomment this (if applicable). Consider adding a page break.
    %\listoffigures
    %\listoftables
    \clearpage
    
    % 8. now you write!
    \section{Introduction}
    \subsection{Purpose}
	
    The purpose of this Software Design Document (SDD) is to provide details about the architecture of the web application as well
	as details about each component. It will describe the underlying design of each component their purpose within the context
	of the application that will allow Oregon State affiliates to effectively monitor energy usage on campus.
	
    \subsection{Scope}
    
	The software outlined in this document will act as an interface by which Oregon State students, faculty, and affiliates
	may monitor energy usage of various buildings located on the Oregon State Campus. It will act as a tool for users to 
	make informed infrastructure decisions and adjustments or act as a display piece to be presented within buildings on campus.
	It will serve the OSU office of sustainability as a replacement to an outdated and costly implementation. 
	
    \subsection{Overview}
    Oregon State University is constantly making strides to reduce its carbon footprint and reduce its energy consumption. There is a carbon neutrality goal for 2025 where the university is trying to generate as much energy as it uses and have a net carbon footprint of 0. Our web application aims to monitor the energy use of buildings on campus in order to create a visual representation of each building's consumption and incentivize people to monitor their consumption habits and reduce the campus's overall consumption.

    \noindent Our web application will resemble an administrative dashboard that has charts and graphs of energy use over time for individual buildings and subgroups of buildings on Oregon State's campus. 
    Our application will have a series of public facing pages that show general data use for all the buildings on campus that have AcquiSuites as well as user generated stories of personalized dashboards. Each page will be a grid-based dashboard with personalized blocks for displaying data through time-series charts or graphs. A user will be able to add different blocks to their pages to create a unique dashboard for their own unique interests and subsets of buildings. An example of this would be a page designated to only residence halls where each block shows a usage over time graph for energy consumption of each residence hall. Users will be able to apply date filters to their blocks to generate different data sets and more explicit visualizations.

    \noindent Our web application will be constructed using a MEAN Stack framework hosted on AWS. Our application will have a MongoDB database server and a Node.js application server hosted on a single virtual EC2 instance to make deployment easy and reliable.
   
    \subsection{Reference Material}
    \subsection{Definitions and Acronyms}
    This section is aimed towards defining terms and acronyms that are specific to the application that may otherwise be misunderstood or poorly interpreted.
    \begin{itemize}
        \item \textbf{Dashboard:}
            A grid-based information management tool for visually tracking and displaying metrics and data through graphs and charts.
        \item \textbf{Block:} 
            An individual graph or chart depicting time-series data.
        \item \textbf{Page:} 
            A dashboard of different blocks. 
        \item \textbf{Story:} 
            A user generated collection of pages. 
        \item \textbf{MEAN Stack:} 
            An acronym used to define a full stack application engineered from the MongoDB, Express.js, Angular.js, and Node.js frameworks. 
        \item \textbf{Bootstrap:} 
            Bootstrap CSS is a front-end framework that uses component templates to easily generate different HTML elements like buttons, navigation, or forms. 
        \item \textbf{OAuth 2.0:} 
            OAuth 2.0 is an authorization protocol that grants authentication through tokens rather than credentials. 
        \item \textbf{Filter:} 
            Filters are essentially parameters for fetching data from that constrain the subset of data being received to the specifications of the filter parameters (i.e. a date range). 
        \item \textbf{D3:} 
            D3.js is a visualization framework that appends charts and graphs to DOM elements on a webpage. 
        \item \textbf{AcquiSuite:} 
            AcquiSuites are data acquisition servers made by a company called Obvius that post building meter data to a designated IP address.
        \item \textbf{AWS:} 
            AWS is an acronym for Amazon Web Services which offers reliable cloud computing services for building and hosting web applications.
    \end{itemize}
    \section{System Overview} 
    Oregon State University is constantly making strides to reduce its carbon footprint and reduce its energy consumption. There is a carbon neutrality goal for 2025 where the university is trying to generate as much energy as it uses and have a net carbon footprint of 0. Our web application aims to monitor the energy use of buildings on campus in order to create a visual representation of each building's consumption and incentivize people to monitor their consumption habits and reduce the campus's overall consumption.

    \noindent Our web application will resemble an administrative dashboard that has charts and graphs of energy use over time for individual buildings and subgroups of buildings on Oregon State's campus.The application will have a series of public facing pages that show general data use for all the buildings on campus that have AcquiSuites as well as user generated stories of personalized dashboards. Each page will be a grid-based dashboard with personalized blocks for displaying data through time-series charts or graphs. A user will be able to add different blocks to their pages to create a unique dashboard for their own unique interests and subsets of buildings. An example of this would be a page designated to only residence halls where each block shows a usage over time graph for energy consumption of each residence hall. Users will be able to apply date filters to their blocks to generate different data sets and more explicit visualizations. 
    \noindent Our web application will be constructed using a MEAN Stack framework hosted on AWS. Our application will have a MongoDB database server and a Node.js application server hosted on a single virtual EC2 instance to make deployment easy and reliable.
    \section{System Architecture}
    \subsection{Architectural Design}
    \iffalse
    Develop a modular program structure and explain the relationships between the
    modules to achieve the complete functionality of the system. This is a high level
    overview of how responsibilities of the system were partitioned and then assigned to
    subsystems. Identify each high level subsystem and the roles or responsibilities
    assigned to it. Describe how these subsystems collaborate with each other in order to
    achieve the desired functionality. Don’t go into too much detail about the individual
    subsystems. The main purpose is to gain a general understanding of how and why
    the system was decomposed, and how the individual parts work together. Provide a
    diagram showing the major subsystems and data repositories and their
    interconnections. Describe the diagram if required. 
    \fi
    \begin{figure}[H] 
    \begin{paddedtikzpicture}[node distance = 2cm, auto]
        % Place nodes
        \node (ui)[block] {User Interface}; 
        \path (ui.east)+(2,4) node (nav) [block] {Navigation}; 
        \path [line] (ui)+(-0.3,0.7) -- (nav);
        \path (ui.east)+(2,2) node (block) [block] {Create Block}; 
        \path [line] (ui) -- (block);
        \path (ui.east)+(2,0) node (story) [block] {Create Story}; 
        \path [line] (ui) -- (story);
        \path (ui.east)+(2.3,-2) node (add) [block,  text width=7.5em] {Add Building/Meter};
        \path [line] (ui) -- (add);
        \path (ui.east)+(2,-4) node (log) [block] {Log In}; 
        \path [line] (ui)+(-0.3,-0.7) -- (log);
        
        %Create Block
        \path (block.east)+(2,0.5) node (ctype) [subblock] {Get Chart Type};
        \path [line] (block) -- (ctype);
        \path (block.east)+(2,-0.5) node (bdata)[subblock] {Get Building Data};
        \path [line] (block) -- (bdata); 
        \path (block.east)+(5,0) node (template)[subblock] {Retrieve Graph Template}; 
        \path [line] (bdata) -- (template); 
        \path [line] (ctype) -- (template); 
        \path (template.east)+(2,0) node (d3)[subblock] {Build D3}; 
        \path [line] (template) -- (d3); 
        
        %Create Story
        \path (story.east)+(2,0) node (createblock)[subblock] {Create Blocks};
        \path [line] (story) -- (createblock); 
        
        %Add Building
        \path (add.east)+(2,0) node (input)[subblock] {Get Input Data};
        \path [line] (add) -- (input); 
        \path (input.east)+(2,0) node (storeinput)[subblock] {Store Input Data};
        \path [line] (input) -- (storeinput); 

        %Log In
        \path (log.east)+(2,0) node (auth) [subblock]  {Google oAuth 2.0};
        \path [line] (log) -- (auth);
        \path (auth.east)+(2,0.5) node (store) [subblock] {Store User};
        \path [line] (auth) -- (store);
        \path (auth.east)+(2,-0.5) node (retrieve)[subblock] {Retrieve User}; 
        \path [line] (auth) -- (retrieve);
    \end{paddedtikzpicture}
    \caption{An overview of the application systems.} \label{fig}
    \end{figure}
    \subsection{Decomposition Description}
    \iffalse
    Provide a decomposition of the subsystems in the architectural design. Supplement
    with text as needed. You may choose to give a functional description or an object
    oriented description.
    For a functional description, put top level data flow diagram (DFD) and structural
    decomposition diagrams.
    For an OO description, put subsystem model, object diagrams, generalization
    hierarchy diagram(s) (if any), aggregation hierarchy diagram(s) (if any), interface
    specifications, and sequence diagrams here.
    \fi
    All database operations (retrieving data and storing data) are denoted with red fill and will be carried out by an AngularJS ``Service'' within the component's controller. 
    \subsubsection{Log In}
    \begin{figure}[H] 
    \begin{paddedtikzpicture}[node distance = 2cm, auto]
        % Place nodes
        \node (login)[block] {Log In}; 
        \path (login.east)+(2,0) node (googleauth) [block] {Contact Google oAuth API}; 
        \path [line] (login) -- (googleauth);
        \path (googleauth.east)+(2,0.5) node (store) [datablock] {Store User in Database}; 
        \path [line] (googleauth) -- (store);
        \path (googleauth.east)+(2,-0.5) node (retrieve) [datablock] {Retrieve User From Database}; 
        \path [line] (googleauth) -- (retrieve);
    \end{paddedtikzpicture} 
    \caption{A functional representation of the log in subsystem.} \label{fig}
    \end{figure}
    \subsubsection{Navigation} 
    \begin{figure}[H] 
    \begin{paddedtikzpicture}[node distance = 2cm, auto]
        % Place nodes
        \node (nav)[block] {Navigation Bar}; 
        \path (nav.east)+(2,0) node (route) [block] {Get Route from Nav Item}; 
        \path [line] (nav) -- (route);
        \path (route.east)+(2,0) node (view) [subblock] {Change View Container Content}; 
        \path [line] (route) -- (view);
        \path (view.east)+(2,0.7) node (render) [subblock, text width=8em] {Render New View HTML}; 
        \path [line] (view) -- (render);
        \path (view.east)+(2,-0.7) node (angular) [subblock, text width=8em] {Adjust Angular Modules for New View}; 
        \path [line] (view) -- (angular);
    \end{paddedtikzpicture} 
    \caption{A functional representation of the navigation subsystem.} \label{fig}
    \end{figure}
    \subsubsection{Create Block} 
    \begin{figure}[H] 
    \begin{paddedtikzpicture}[node distance = 2cm]
        % Place nodes
        \node (create)[block] {Create Block}; 
        \path (create.east)+(2,0) node (view) [block] {Serve Web Form In View};
        \path [line] (create) -- (view);
        \path (view.east)+(2,0.5) node (charts) [datablock] {Pull Charts from Database}; 
        \path [line] (view) -- (charts);
        \path (view.east)+(2,-0.5) node (buildings) [datablock] {Pull Buildings from Database}; 
        \path [line] (view) -- (buildings);
        \path (view.east)+(5.5,0) node (submit) [subblock, text width=8em] {Get User Form Data}; 
        \path [line] (charts) -- (submit);
        \path [line] (buildings) -- (submit);
        \path (submit.east)+(2,0) node (store) [datablock, text width=8em] {Store Block in User Object}; 
        \path [line] (submit) -- (store);
    \end{paddedtikzpicture} 
    \caption{A functional representation of the block creation subsystem.} \label{fig}
    \end{figure}
    \subsubsection{View Blocks}
    \begin{figure}[H] 
    \begin{paddedtikzpicture}[node distance = 2cm]
        % Place nodes
        \node (view)[block] {View Blocks}; 
        \path (view.east)+(2,0) node (retrieve) [datablock] {Retrieve Blocks From User Object};
        \path [line] (view) -- (retrieve);
        \path (retrieve.east)+(2,0) node (render) [subblock] {Render Blocks to Content View}; 
        \path [line] (retrieve) -- (render);
        \path (render.east)+(2,0.5) node (edit) [datablock, text width=8em] {Edit Block Fields}; 
        \path (render.east)+(2,-0.5) node (delete) [datablock, text width=8em] {Delete Block};
        \path [line] (render) -- (edit);
        \path [line] (render) -- (delete); 
    \end{paddedtikzpicture}  
    \caption{A functional representation of the block view subsystem.} \label{fig}
    \end{figure}
    Blocks are stored in the User object who created them in an array ``Blocks.'' This way a user can keep track of the specific blocks they create and reuse them to create stories.
    \subsubsection{Create Story}
    \begin{figure}[H] 
    \begin{paddedtikzpicture}[node distance = 2cm]
        % Place nodes
        \node (story)[block] {Create Story}; 
        \path (story.east)+(2,0) node (retrieve) [datablock] {Retrieve Blocks From User Object};
        \path [line] (story) -- (retrieve);
        \path (retrieve.east)+(2,0) node (add) [subblock, text width=8em] {Append Selected Block to Content View};
        \path [line] (retrieve) -- (add);
        \path (add.east)+(2,0) node (store) [datablock, text width=8em] {Store Block in Story Object};
        \path [line] (add) -- (store); 
    \end{paddedtikzpicture}   
    \caption{A functional representation of the story creation subsystem.} \label{fig}
    \end{figure}
    Stories are stored in the User object who created them in an array ``Stories.'' This allows the application to render stories created by the session user easily and will eliminate the use of SQL style querying to retrieve stories based on a user key.
    \subsubsection{Add Building/Meter}
    \begin{figure}[H] 
        \begin{paddedtikzpicture}[node distance = 2cm]
            % Place nodes
            \node (add)[block, text width=8em] {Add Building/Meter}; 
            \path (add.east)+(2,0) node (render) [subblock] {Render Input Form};
            \path [line] (add) -- (render);
            \path (render.east)+(2,0) node (store) [datablock, text width=8em] {Store Form Values As New Object};
            \path [line] (render) -- (store);
        \end{paddedtikzpicture}   
        \caption{A functional representation for storing new building and meter objects to the database.} \label{fig}
    \end{figure}
    \subsubsection{Get AcquiSuite Data}
    \begin{figure}[H] 
        \begin{paddedtikzpicture}[node distance = 2cm]
            % Place nodes
            \node (data)[block] {Receive AcquiSuite Data}; 
            \path (data.east)+(2,0) node (parse) [subblock] {Parse Data to JSON Object};
            \path [line] (data) -- (parse);
            \path (render.east)+(2,0) node (store) [datablock, text width=8em] {Store Data in Building Object};
            \path [line] (parse) -- (store);
        \end{paddedtikzpicture}   
        \caption{A functional representation for storing AcquiSuite data to a building object.} \label{fig}
    \end{figure}
    Timestamped data entries from Acquisuite Servers will live in the building objects the data came from. Similar to the rational behind storing blocks in the user object that created them, each building will have an array of meter data sorted by timestamp. This will allow our application to use a service to retrieve the data for any graphs or charts directly from the buildings objects in the relative block. A graph service can then fetch the building data, apply filters (like date ranges), and pass it to a D3 graph to render. This minimizes the amount of total data to be parsed and collected by narrowing the data to the relative block.
    \subsubsection{Create Graph}
    \begin{figure}[H] 
        \begin{paddedtikzpicture}[node distance = 2cm]
            % Place nodes
            \node (graph)[block] {Create Graph}; 
            \path (graph.east)+(2,0) node (get) [datablock] {Get Data From Parent Block};
            \path [line] (graph) -- (get);
            \path (get.east)+(2,0) node (fetch) [datablock, text width=8em] {Fetch Graph Template};
            \path [line] (get) -- (fetch);
            \path (fetch.east)+(2,0) node (render) [subblock, text width=8em] {Serve Graph to HTML};
            \path [line] (fetch) -- (render);
        \end{paddedtikzpicture}   
        \caption{A functional representation for building a graph in a block.} \label{fig}
    \end{figure}
    To render a chart or graph to the page, our application will gather all the parameters from the block component that contains the graph (the building(s), the chart type, and the date range). Then the graph component will fetch a D3 template for its specific chart type, input the data from the block, and serve it to the page. 
    \subsection{Design Rationale}
    The system's architecture revolves around a heavily modular system where individual components have their own functionality while also able to work cohesively with other components. The modular architecture also breaks up the back-end components into smaller services which simplifies functionality across the entire application.
    \section{Component Design}
    This section is designated to describing individual components in greater detail, including the type of element, designated functionality, and the reasoning behind their design. Overall, this section aims to provide an in-depth view of the application components and describe their individual roles in the application as a whole.
    \subsection{Block}
    A block is a component in our application that acts as a container for a graph, filter options, and  information about the data in the graph.
    \subsubsection{Element} 
    A block is an HTML element that uses the Bootstrap 4 ``card'' component for its design. It will have a title bar with the block's name, then a content container split into a grid. On the left, the block will display the building(s) being graphed, then contain a column of filter buttons, then the D3 graph in the right column. 
    \subsubsection{Functionality}
    The block component neatly organizes data for the user by encapsulating the data, filters, and graph components into one container. A block needs to pull data from the ``building'' objects that it contains, gather data from any filters, and pass all this data into the correct graph template to generate a D3 graph. There will be a service within the block's AngularJS controller that will gather this data and pass the desired output to the page.
    \subsubsection{Rationale} 
    The block component is an efficient way to create a reusable HTML template that can be reproduced throughout the application. This makes it simple to provide a consistent process for displaying graphs and data. Additionally, the block component yields a compact solution for organizing data from multiple sources. 
    
	\subsection{Login}
	The Login is an HTML button that will allow users to login with an account, which will have different permissions and preferences for each user.
	\subsubsection{Element}
	The login function will be a button in the corner of the screen or somewhere in the navbar. The button will not be present if user is already logged in.
	\subsubsection{Functionality}
    Clicking the login component will redirect the user to our authentication service. The application will use ``PassportJS'' as an authentication middleware which authenticates the user with Google oAuth 2.0 tokens and sets the req.user upon redirecting to the specified route. Once a user is logged in, they will have access to additional application components and navigation items based on role.
	\subsubsection{Rationale}
	From a developer standpoint, requiring users to login for certain features will make separation of privileges much easier. With authentication, users will also be able to have the convenience of Google SSO.
	
	\subsection{Navigation Bar}
	The Navigation bar is a commonly used UI feature that provides users the ability to navigate through different pages of the application.
    \subsubsection{Element}
    The navigation bar will be an HTML element with differentiable ``nav-items'' that render different views to the content view space. There will be a fixed-top navigation bar for generic item and a side bar navigation for more explicit functionalities like admin controls. 
    \subsubsection{Functionality}
    Clicking on a nav-item has one of two functionalities: 
    \begin{itemize}
        \item Redirects the user to a new page by having the route's get function perform a redirect.
        \item Renders a new view to the page by having the route's get function render a new HTML view.
    \end{itemize}
	The navigation bar shows which item is currently being served by changing the style of the active nav-item to be different than the rest.
	\subsubsection{Rationale}
    From a user perspective, having a navigation bar to navigate around the main features is an extremely useful UI design decision. From a design perspective, having a dedicated area to insert useful links is much more streamline than listing them. The navigation bar is a very useful tool for serving different content to the view space statically.
    \subsection{Graph Component}
    Our application will contain a graph component that accepts a graph type and data as input and constructs a graph using the D3 visualization framework.
    \subsubsection{Element}
    A graph component will be an SVG element, created with D3, that is served to the page inside of a block. 
    \subsubsection{Functionality}
    The graph component's functionality involves gathering different input parameters from the parent block component, fetching a template based on the graph type, and creating a unique graph. The graph component should have its own service to handle all the data organization and create a list of uniform parameters than can be passed to a D3 graph template.
    \subsubsection{Rationale}
    Having a structure like this simplifies the act of generating graphs by having a series of templates that can be fed data in a specific format. The graph service can take all the data and different filter parameters from the block object and format it into the right structure. All the computationally heavy functionality is black-boxed in the graph service and keeps the functionality handled by the user and front-end at a minimal. Another benefit to the graph component is that it can change the graph dynamically. Our application may include filters in the block component like a date range or a filter to change the graphed variable, which can trigger the graph service to reconstruct the data and create a new graph to serve based on the applied filters.
	\section{Human Interface Design}
    \subsection{Overview of User Interface}
    \iffalse
    Describe the functionality of the system from the user’s perspective. Explain how the
    user will be able to use your system to complete all the expected features and the
    feedback information that will be displayed for the user.
    \fi
    \subsection{Screen Images}
    \iffalse
    Display screenshots showing the interface from the user’s perspective. These can be
    hand drawn or you can use an automated drawing tool. Just make them as accurate
    as possible. (Graph paper works well.) 
    \fi 
	
	\subsubsection{Home Page (Public Access)}
    \begin{figure}[H]
        \centering
        \includegraphics[width=14cm]{images/Home_NOT_Logged_In.eps}
        \caption{A mock-up of a home page from a general public user access perspective (not logged-in to an account).}
    \end{figure}
	If a general public user does not have an account but wishes to access energy monitoring data, they will
	have a unique experience with the site that is different than those who are logged-in to authorized accounts. 
	For instance, they will not have the ability to create personal dashboards or stories, but will be allowed to view
	public stories, public dashboards, and general building dashboards. They will not have access to the same navigation 
	bar on the left side of the screen as a logged-in user. 
	
	\subsubsection{Home Page (Logged-In Access)}
    \begin{figure}[H]
        \centering
        \includegraphics[width=14cm]{images/Home_Logged_In.eps}
        \caption{A mock-up of a home page from an authorized user access perspective (logged-in to an account).}
    \end{figure}
	If a authorized user wishes to access energy monitoring data, they will be able to have special privileges 
	For instance, they will have the ability to create personal dashboards or stories which they may specify as private or public
	They will have access to a navigation bar on the left side of the screen which allows for unique data viewing and privileges. 
	
	
    \subsubsection{Buildings Page} 
    \begin{figure}[H]
        \centering
        \includegraphics[width=14cm]{images/buildings.eps}
        \caption{A mock-up of the page that shows all buildings.}
    \end{figure}
	Users will may select the ``Buildings'' tab in the nav bar to bring up an array of building cards to choose from.
	Every building on campus that is gathering metering data associated with this application will have a card in this array.
	A user may select a building card to bring up a generalized dashboard of information for each specific building. 
	The user will be able to search for buildings by name in a search bar. Administrators will be able to add buildings
	as the metering system expands to new buildings.
	
    \subsubsection{Selected Building Page}
    \begin{figure}[H]
        \centering
        \includegraphics[width=14cm]{images/Building_Select.eps}
        \caption{A mock-up of a selected building page.}
    \end{figure}
	Once a user has selected a building card, a generalized dashboard will be displayed to them.
	All buildings associated with this application will contain the same generalized dashboard to be viewed.
	These dashboards will display various cards of data detailing the most general information about their respective buildings.
	A user may full screen the application to remove menu clutter and allow for a more presentable display of the data.
	
	\subsubsection{Dashboards Page}
    \begin{figure}[H]
        \centering
        \includegraphics[width=14cm]{images/Dashboards.eps}
        \caption{A mock-up page of a list of user generated dashboards.}
    \end{figure}
	Users will may select the ``Dashboards'' tab in the nav bar to bring up a list of user-generated dashboards.
	The dashboards will contain user specified arrangements of data cards buildings or groups of buildings. 
	The list of dashboards will detail the name of each dashboard, the user that created each dashboard, and the date and time of when each was created.
	The user will be able to search for dashboards by name in a search bar. 
	A user may select a dashboard from the list to view or use the options menu for more specific tasks.
	
    \subsubsection{Selected Dashboard Page}
    \begin{figure}[H]
        \centering
        \includegraphics[width=14cm]{images/Dashboard_Select.eps}
        \caption{A mock-up of a selected dashboard page.}
    \end{figure}
	Once a user has selected a dashboard, a user-generated dashboard will be displayed to them.
	The dashboard will display various cards of data presenting information about various buildings energy usage.
	A user may full screen the application to remove menu clutter and allow for a more presentable display of the data.
	
    \subsubsection{Story Page}
    \begin{figure}[H]
        \centering
        \includegraphics[width=14cm]{images/Stories.eps}
        \caption{A mock-up of a list of user generated story pages.}
    \end{figure}
	Users will may select the ``My Stories'' tab in the nav bar to bring up a drop down list of their personal stories in the left side navigation bar.
	Each story will contain a list of dashboards that was generated by the current user to their preference. 
	The list of stories will include both their private and public stories.
	A user may select a story from the list to view or use the options menu for more specific tasks.
	
	
    \section{Appendixes}
    Appendices may be included, either directly or by reference, to provide supporting details
    that could aid in the understanding of the Software Design Document. 
    %\section{Index}
    
    \end{document}