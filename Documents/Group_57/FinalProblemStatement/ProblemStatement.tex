\documentclass[onecolumn, draftclsnofoot,10pt, compsoc]{IEEEtran}
    \usepackage{graphicx}
    \usepackage{url}
    \usepackage{setspace}
    
    \usepackage[margin=0.75in]{geometry}
    \geometry{textheight=9.5in, textwidth=7in}
    
    % 1. Fill in these details
    \def \CapstoneTeamName{		The Dream Team}
    \def \CapstoneTeamNumber{		57}
    \def \GroupMemberOne{			Daniel Schroeder}
    \def \GroupMemberTwo{			Aubrey Thenell}
    \def \GroupMemberThree{			Parker Bruni}
    \def \CapstoneProjectName{		A Scalable Web Application Framework for Monitoring Energy Usage on Campus  }
    \def \CapstoneSponsorCompany{	Oregon State Office of Sustainability}
    \def \CapstoneSponsorPerson{		Jack Woods}
    
    % 2. Uncomment the appropriate line below so that the document type works
    \def \DocType{		Problem Statement
            %Requirements Document
            %Technology Review
            %Design Document
            %Progress Report
            }
          
    \newcommand{\NameSigPair}[1]{\par
    \makebox[2.75in][r]{#1} \hfil 	\makebox[3.25in]{\makebox[2.25in]{\hrulefill} \hfill		\makebox[.75in]{\hrulefill}}
    \par\vspace{-12pt} \textit{\tiny\noindent
    \makebox[2.75in]{} \hfil		\makebox[3.25in]{\makebox[2.25in][r]{Signature} \hfill	\makebox[.75in][r]{Date}}}}
    % 3. If the document is not to be signed, uncomment the RENEWcommand below
    %\renewcommand{\NameSigPair}[1]{#1}
    
    %%%%%%%%%%%%%%%%%%%%%%%%%%%%%%%%%%%%%%%
    \begin{document}
    \begin{titlepage}
        \pagenumbering{gobble}
        \begin{singlespace}
          \includegraphics[height=4cm]{coe_v_spot1.eps}
            \hfill 
            % 4. If you have a logo, use this includegraphics command to put it on the coversheet.
            %\includegraphics[height=4cm]{CompanyLogo}   
            \par\vspace{.2in}
            \centering
            \scshape{
                \huge CS Capstone \DocType \par
                {\large\today}\par
                \vspace{.5in}
                \textbf{\Huge\CapstoneProjectName}\par
                \vfill
                {\large Prepared for}\par
                \Huge \CapstoneSponsorCompany\par
                \vspace{5pt}
                {\Large\NameSigPair{\CapstoneSponsorPerson}\par}
                {\large Prepared by }\par
                Group\CapstoneTeamNumber\par
                % 5. comment out the line below this one if you do not wish to name your team
                \CapstoneTeamName\par 
                \vspace{5pt}
                {\Large
                    \NameSigPair{\GroupMemberOne}\par
                    \NameSigPair{\GroupMemberTwo}\par
                    \NameSigPair{\GroupMemberThree}\par
                }
                \vspace{20pt}
            }
            \begin{abstract}
            % 6. Fill in your abstract  
            This document outlines the problem presented to us by Oregon State University’s Office of Sustainability, for whom we will be creating a web application that visualizes energy use on campus. This is a generic overview of the problem at hand, and the scheduled course of action.  
            \end{abstract}     
        \end{singlespace}
    \end{titlepage}
    \newpage
    \pagenumbering{arabic}
    \tableofcontents
    % 7. uncomment this (if applicable). Consider adding a page break.
    %\listoffigures
    %\listoftables
    \clearpage
    
    % 8. now you write!
    \section{Problem Definition}
  
    \section{Proposed Solution}
    \indent This web application will follow strict MVC guidelines to ensure that the Universities data and security is help to the utmost importance. The Express framework will act as our controller that will link the database with the web application's view and maintain permissions for security and data integrity.
    \section{Performance Metrics}
    When this web application is complete, it should have three areas of focus and deliverability: functionality, scalability, and the user interface. This web application will be able to gather data from Acquisuite sensors all over campus and graphically display that information in different subsets based on user preference and filters. The data will be displayed on an accepted graphic interface and have a general user platform, as well as an administrative platform.
    This web application will require a framework that can provide the following features:
    \begin{itemize}
        \item An interface where users and administrators can create custom dashboards from a series of pre-established graphs and charts.
        \item An interface where users can create new “blocks” of content (i.e. graphs and charts).
        \item An interface where users can customize the layout of their dashboard and change the positions of the different display blocks.
        \item The ability for blocks to receive data and update in real time.
        \item An interface where dashboards can display interactive building maps where users can view data associated to specific buildings in real time.
        \item A page with data for each building on campus.
        \item An interface for scalability that will allow administrators to add new buildings, meters, and subspaces.
    \end{itemize}
    
    
    \end{document}