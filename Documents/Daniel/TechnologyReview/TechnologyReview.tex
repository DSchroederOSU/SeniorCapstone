\documentclass[onecolumn, draftclsnofoot,10pt, compsoc]{IEEEtran}
\usepackage{graphicx}
\usepackage{url}
\usepackage{setspace}
\usepackage{hyperref}
\usepackage{changepage}% http://ctan.org/pkg/changepage

\def\name{Daniel Schroeder}

%pull in the necessary preamble matter for pygments output
\usepackage{fancyvrb}
\usepackage{color}
\usepackage[latin1]{inputenc}


\makeatletter
\def\PY@reset{\let\PY@it=\relax \let\PY@bf=\relax%
    \let\PY@ul=\relax \let\PY@tc=\relax%
    \let\PY@bc=\relax \let\PY@ff=\relax}
\def\PY@tok#1{\csname PY@tok@#1\endcsname}
\def\PY@toks#1+{\ifx\relax#1\empty\else%
    \PY@tok{#1}\expandafter\PY@toks\fi}
\def\PY@do#1{\PY@bc{\PY@tc{\PY@ul{%
    \PY@it{\PY@bf{\PY@ff{#1}}}}}}}
\def\PY#1#2{\PY@reset\PY@toks#1+\relax+\PY@do{#2}}

\expandafter\def\csname PY@tok@gd\endcsname{\def\PY@tc##1{\textcolor[rgb]{0.63,0.00,0.00}{##1}}}
\expandafter\def\csname PY@tok@gu\endcsname{\let\PY@bf=\textbf\def\PY@tc##1{\textcolor[rgb]{0.50,0.00,0.50}{##1}}}
\expandafter\def\csname PY@tok@gt\endcsname{\def\PY@tc##1{\textcolor[rgb]{0.00,0.25,0.82}{##1}}}
\expandafter\def\csname PY@tok@gs\endcsname{\let\PY@bf=\textbf}
\expandafter\def\csname PY@tok@gr\endcsname{\def\PY@tc##1{\textcolor[rgb]{1.00,0.00,0.00}{##1}}}
\expandafter\def\csname PY@tok@cm\endcsname{\let\PY@it=\textit\def\PY@tc##1{\textcolor[rgb]{0.25,0.50,0.50}{##1}}}
\expandafter\def\csname PY@tok@vg\endcsname{\def\PY@tc##1{\textcolor[rgb]{0.10,0.09,0.49}{##1}}}
\expandafter\def\csname PY@tok@m\endcsname{\def\PY@tc##1{\textcolor[rgb]{0.40,0.40,0.40}{##1}}}
\expandafter\def\csname PY@tok@mh\endcsname{\def\PY@tc##1{\textcolor[rgb]{0.40,0.40,0.40}{##1}}}
\expandafter\def\csname PY@tok@go\endcsname{\def\PY@tc##1{\textcolor[rgb]{0.50,0.50,0.50}{##1}}}
\expandafter\def\csname PY@tok@ge\endcsname{\let\PY@it=\textit}
\expandafter\def\csname PY@tok@vc\endcsname{\def\PY@tc##1{\textcolor[rgb]{0.10,0.09,0.49}{##1}}}
\expandafter\def\csname PY@tok@il\endcsname{\def\PY@tc##1{\textcolor[rgb]{0.40,0.40,0.40}{##1}}}
\expandafter\def\csname PY@tok@cs\endcsname{\let\PY@it=\textit\def\PY@tc##1{\textcolor[rgb]{0.25,0.50,0.50}{##1}}}
\expandafter\def\csname PY@tok@cp\endcsname{\def\PY@tc##1{\textcolor[rgb]{0.74,0.48,0.00}{##1}}}
\expandafter\def\csname PY@tok@gi\endcsname{\def\PY@tc##1{\textcolor[rgb]{0.00,0.63,0.00}{##1}}}
\expandafter\def\csname PY@tok@gh\endcsname{\let\PY@bf=\textbf\def\PY@tc##1{\textcolor[rgb]{0.00,0.00,0.50}{##1}}}
\expandafter\def\csname PY@tok@ni\endcsname{\let\PY@bf=\textbf\def\PY@tc##1{\textcolor[rgb]{0.60,0.60,0.60}{##1}}}
\expandafter\def\csname PY@tok@nl\endcsname{\def\PY@tc##1{\textcolor[rgb]{0.63,0.63,0.00}{##1}}}
\expandafter\def\csname PY@tok@nn\endcsname{\let\PY@bf=\textbf\def\PY@tc##1{\textcolor[rgb]{0.00,0.00,1.00}{##1}}}
\expandafter\def\csname PY@tok@no\endcsname{\def\PY@tc##1{\textcolor[rgb]{0.53,0.00,0.00}{##1}}}
\expandafter\def\csname PY@tok@na\endcsname{\def\PY@tc##1{\textcolor[rgb]{0.49,0.56,0.16}{##1}}}
\expandafter\def\csname PY@tok@nb\endcsname{\def\PY@tc##1{\textcolor[rgb]{0.00,0.50,0.00}{##1}}}
\expandafter\def\csname PY@tok@nc\endcsname{\let\PY@bf=\textbf\def\PY@tc##1{\textcolor[rgb]{0.00,0.00,1.00}{##1}}}
\expandafter\def\csname PY@tok@nd\endcsname{\def\PY@tc##1{\textcolor[rgb]{0.67,0.13,1.00}{##1}}}
\expandafter\def\csname PY@tok@ne\endcsname{\let\PY@bf=\textbf\def\PY@tc##1{\textcolor[rgb]{0.82,0.25,0.23}{##1}}}
\expandafter\def\csname PY@tok@nf\endcsname{\def\PY@tc##1{\textcolor[rgb]{0.00,0.00,1.00}{##1}}}
\expandafter\def\csname PY@tok@si\endcsname{\let\PY@bf=\textbf\def\PY@tc##1{\textcolor[rgb]{0.73,0.40,0.53}{##1}}}
\expandafter\def\csname PY@tok@s2\endcsname{\def\PY@tc##1{\textcolor[rgb]{0.73,0.13,0.13}{##1}}}
\expandafter\def\csname PY@tok@vi\endcsname{\def\PY@tc##1{\textcolor[rgb]{0.10,0.09,0.49}{##1}}}
\expandafter\def\csname PY@tok@nt\endcsname{\let\PY@bf=\textbf\def\PY@tc##1{\textcolor[rgb]{0.00,0.50,0.00}{##1}}}
\expandafter\def\csname PY@tok@nv\endcsname{\def\PY@tc##1{\textcolor[rgb]{0.10,0.09,0.49}{##1}}}
\expandafter\def\csname PY@tok@s1\endcsname{\def\PY@tc##1{\textcolor[rgb]{0.73,0.13,0.13}{##1}}}
\expandafter\def\csname PY@tok@sh\endcsname{\def\PY@tc##1{\textcolor[rgb]{0.73,0.13,0.13}{##1}}}
\expandafter\def\csname PY@tok@sc\endcsname{\def\PY@tc##1{\textcolor[rgb]{0.73,0.13,0.13}{##1}}}
\expandafter\def\csname PY@tok@sx\endcsname{\def\PY@tc##1{\textcolor[rgb]{0.00,0.50,0.00}{##1}}}
\expandafter\def\csname PY@tok@bp\endcsname{\def\PY@tc##1{\textcolor[rgb]{0.00,0.50,0.00}{##1}}}
\expandafter\def\csname PY@tok@c1\endcsname{\let\PY@it=\textit\def\PY@tc##1{\textcolor[rgb]{0.25,0.50,0.50}{##1}}}
\expandafter\def\csname PY@tok@kc\endcsname{\let\PY@bf=\textbf\def\PY@tc##1{\textcolor[rgb]{0.00,0.50,0.00}{##1}}}
\expandafter\def\csname PY@tok@c\endcsname{\let\PY@it=\textit\def\PY@tc##1{\textcolor[rgb]{0.25,0.50,0.50}{##1}}}
\expandafter\def\csname PY@tok@mf\endcsname{\def\PY@tc##1{\textcolor[rgb]{0.40,0.40,0.40}{##1}}}
\expandafter\def\csname PY@tok@err\endcsname{\def\PY@bc##1{\setlength{\fboxsep}{0pt}\fcolorbox[rgb]{1.00,0.00,0.00}{1,1,1}{\strut ##1}}}
\expandafter\def\csname PY@tok@kd\endcsname{\let\PY@bf=\textbf\def\PY@tc##1{\textcolor[rgb]{0.00,0.50,0.00}{##1}}}
\expandafter\def\csname PY@tok@ss\endcsname{\def\PY@tc##1{\textcolor[rgb]{0.10,0.09,0.49}{##1}}}
\expandafter\def\csname PY@tok@sr\endcsname{\def\PY@tc##1{\textcolor[rgb]{0.73,0.40,0.53}{##1}}}
\expandafter\def\csname PY@tok@mo\endcsname{\def\PY@tc##1{\textcolor[rgb]{0.40,0.40,0.40}{##1}}}
\expandafter\def\csname PY@tok@kn\endcsname{\let\PY@bf=\textbf\def\PY@tc##1{\textcolor[rgb]{0.00,0.50,0.00}{##1}}}
\expandafter\def\csname PY@tok@mi\endcsname{\def\PY@tc##1{\textcolor[rgb]{0.40,0.40,0.40}{##1}}}
\expandafter\def\csname PY@tok@gp\endcsname{\let\PY@bf=\textbf\def\PY@tc##1{\textcolor[rgb]{0.00,0.00,0.50}{##1}}}
\expandafter\def\csname PY@tok@o\endcsname{\def\PY@tc##1{\textcolor[rgb]{0.40,0.40,0.40}{##1}}}
\expandafter\def\csname PY@tok@kr\endcsname{\let\PY@bf=\textbf\def\PY@tc##1{\textcolor[rgb]{0.00,0.50,0.00}{##1}}}
\expandafter\def\csname PY@tok@s\endcsname{\def\PY@tc##1{\textcolor[rgb]{0.73,0.13,0.13}{##1}}}
\expandafter\def\csname PY@tok@kp\endcsname{\def\PY@tc##1{\textcolor[rgb]{0.00,0.50,0.00}{##1}}}
\expandafter\def\csname PY@tok@w\endcsname{\def\PY@tc##1{\textcolor[rgb]{0.73,0.73,0.73}{##1}}}
\expandafter\def\csname PY@tok@kt\endcsname{\def\PY@tc##1{\textcolor[rgb]{0.69,0.00,0.25}{##1}}}
\expandafter\def\csname PY@tok@ow\endcsname{\let\PY@bf=\textbf\def\PY@tc##1{\textcolor[rgb]{0.67,0.13,1.00}{##1}}}
\expandafter\def\csname PY@tok@sb\endcsname{\def\PY@tc##1{\textcolor[rgb]{0.73,0.13,0.13}{##1}}}
\expandafter\def\csname PY@tok@k\endcsname{\let\PY@bf=\textbf\def\PY@tc##1{\textcolor[rgb]{0.00,0.50,0.00}{##1}}}
\expandafter\def\csname PY@tok@se\endcsname{\let\PY@bf=\textbf\def\PY@tc##1{\textcolor[rgb]{0.73,0.40,0.13}{##1}}}
\expandafter\def\csname PY@tok@sd\endcsname{\let\PY@it=\textit\def\PY@tc##1{\textcolor[rgb]{0.73,0.13,0.13}{##1}}}

\def\PYZbs{\char`\\}
\def\PYZus{\char`\_}
\def\PYZob{\char`\{}
\def\PYZcb{\char`\}}
\def\PYZca{\char`\^}
\def\PYZam{\char`\&}
\def\PYZlt{\char`\<}
\def\PYZgt{\char`\>}
\def\PYZsh{\char`\#}
\def\PYZpc{\char`\%}
\def\PYZdl{\char`\$}
\def\PYZti{\char`\~}
% for compatibility with earlier versions
\def\PYZat{@}
\def\PYZlb{[}
\def\PYZrb{]}
\makeatother


%% The following metadata will show up in the PDF properties
\hypersetup{
  colorlinks = true,
  urlcolor = red,
  linkcolor = black,
  citecolor = black,
  pdfauthor = {\name},
  pdfkeywords = {CS461 ``Senior Capstone'' Technology Review},
  pdftitle = {CS461 ``Senior Capstone'' Technology Review},
  pdfsubject = {CS461 ``Senior Capstone'' Technology Review},
  pdfpagemode = UseNone
}
\usepackage{listings}
\usepackage{color}
\definecolor{lightgray}{rgb}{.9,.9,.9}
\definecolor{darkgray}{rgb}{.4,.4,.4}
\definecolor{purple}{rgb}{0.65, 0.12, 0.82}
\usepackage[margin=0.75in]{geometry}
\geometry{textheight=9.5in, textwidth=7in}
\lstdefinelanguage{JavaScript}{
    keywords={typeof, new, true, false, catch, function, return, null, catch, switch, var, if, in, while, do, else, case, break},
    keywordstyle=\color{blue}\bfseries,
    ndkeywords={class, export, boolean, throw, implements, import, this},
    ndkeywordstyle=\color{darkgray}\bfseries,
    identifierstyle=\color{black},
    sensitive=false,
    comment=[l]{//},
    morecomment=[s]{/*}{*/},
    commentstyle=\color{purple}\ttfamily,
    stringstyle=\color{red}\ttfamily,
    morestring=[b]',
    morestring=[b]"
  }
  \lstset{
    language=JavaScript,
    backgroundcolor=\color{lightgray},
    extendedchars=true,
    basicstyle=\footnotesize\ttfamily,
    linewidth=7in,
    xleftmargin=.25in,
    showstringspaces=false,
    showspaces=false,
    numbers=left,
    numberstyle=\footnotesize,
    numbersep=9pt,
    tabsize=2,
    breaklines=true,
    showtabs=false,
    captionpos=b
 }

% 1. Fill in these details
\def \CapstoneTeamName{		The Dream Team}
\def \CapstoneTeamNumber{		57}
\def \GroupMemberOne{			Daniel Schroeder}
\def \GroupMemberTwo{			Aubrey Thenell}
\def \GroupMemberThree{			Parker Bruni}
\def \CapstoneProjectName{		A Scalable Web Application Framework for Monitoring Energy Usage on Campus  }
\def \CapstoneSponsorCompany{	Oregon State Office of Sustainability}
\def \CapstoneSponsorPerson{		Jack Woods}

% 2. Uncomment the appropriate line below so that the document type works
\def \DocType{		%Problem Statement
        %Requirements Document
        Technology Review
        %Design Document
        %Progress Report
        }
    
\newcommand{\NameSigPair}[1]{\par
\makebox[2.75in][r]{#1} \hfil 	\makebox[3.25in]{\makebox[2.25in]{\hrulefill} \hfill		\makebox[.75in]{\hrulefill}}
\par\vspace{-12pt} \textit{\tiny\noindent
\makebox[2.75in]{} \hfil		\makebox[3.25in]{\makebox[2.25in][r]{Signature} \hfill	\makebox[.75in][r]{Date}}}}
% 3. If the document is not to be signed, uncomment the RENEWcommand below
%\renewcommand{\NameSigPair}[1]{#1}

%%%%%%%%%%%%%%%%%%%%%%%%%%%%%%%%%%%%%%%

\title{\textbf{\CapstoneProjectName} \linebreak \LARGE{Group 57} \linebreak \LARGE{Technology Review}}
\author{Daniel Schroeder}

\date{\today}

\begin{document}
\maketitle
\vfill
\noindent \textbf{Abstract} \\
            \indent 
            The purpose of this document is to provide comprehensive comparisons between different technologies that could be implemented during the development of our web application. This documents seeks to provide pros and cons based on various criteria for each component being reviewed, and selects the best candidate technology to satisfy our needs. Specifically this document focuses on different Visualization Libraries,Authentication Implementations, and Front-end Frameworks.
            
            

\newpage
\pagenumbering{arabic}
\tableofcontents
% 7. uncomment this (if applicable). Consider adding a page break.
%\listoffigures
%\listoftables
\clearpage

% 8. now you write!
\section{Introduction}
My role in this project will be integrating the front-end with the back-end framework and acting as a dev ops engineer. I want to take on the role of dev ops for this project to hone in on certain skills that have not previously been targeted in my past web-development internships. I have created UX/UI and built pages/forms that  manipulate database data. I have written back-end model functions to retrieve data for use in an application. Now I want to manage the full-fledged construction of a full stack web application. Another motive for taking a dev ops role is with a framework like NodeJS, there are endless amounts of libraries and frameworks that you can utilize in your application that can simplify every component. What I want to take away from this course is a well-rounded knowledge about how different components of an application work together and how to integrate each part efficiently.\\
By taking a dev ops role in this project, I will be able to:
\begin{itemize}
    \item Transition away from the tasks I've been previously trained on.
    \item Help my teammates understand things that I have previously learned by not implementing them myself.
    \item Develop a more comprehensive understanding of NodeJS, Angular, Express, and NOSQL Databases.
    \item Take lead in developing a high-end development environment for my team and I in order to create efficient workflow.
    \item Implement components of our web application that are new to me.
\end{itemize}

There are a lot of different components that go into making a web application work and it takes a high-level understanding to implement them all. Our web application will act like an admin dashboard that displays specific charts and graphs based on user preference.\\
Firstly, from a dev ops perspective, we need an architecture that can host a NodeJS application from a single VM. We will be using Amazon Elastic Compute Cloud (EC2) to host our web server. This EC2 instance will listen for inbound requests and transfer them to our Node server daemon. Our Node server will then have to communicate with our MongoDB to retrieve the necessary data.\\
Secondly, we will need a workflow architecture for continuous deployment that can be utilized by all members of the team. We will use Github to create a ``Git-flow'' environment that proceeds as follows:
\begin{itemize}
    \item The repository master branch is considered to be ``production.''
    \item Developers will create ``feature branches'' branched from master.
    \item Feature branches will be pull requested, reviewed, and merged into the master branch.
\end{itemize}
Git-flow allows team members to work on separate features while building off of the current working application. Feature branches make sure that the master branch is always in working condition and no new code gets merged without review and confirmation that it works.\\
Thirdly, our web application also has users with different roles, which adds a lot of overhead to the application. Our application needs to have user authentication, user sessions, user roles, and code to enforce it all. Using Google's oAuth 2.0 authentication, our application will force users to log in with their Google accounts and Google will handle most of the overhead. When a user logs in, ``the Google Authorization Server sends [the] application an access token'' which is then stored in the database instead of a user password \cite{oauth}. Then, through utilizing the Passport.js authentication middleware, we can create user sessions and manage authentication with this Google token and not have any security issues with passwords.\\
The final main component of our web application is the actual functionality of providing data visualizations to the user in a clean and dynamic way. This will include retrieving and parsing data from Acquisuite data acquisition servers, storing this information in the database, and creating dynamic graphs and charts using a visualization framework that can be manipulated by the user. This will involve developing filters and date ranges that can be applied to all the data to select sub-portions of data based on specific buildings, collections of certain buildings, or time periods. Having a visualization framework that works well with AngularJS and our MongoDB database is essential in creating a smooth, dynamic web application. 
%%%%%%%%%%%%%%%%%%%%%%%%%%%%%%%%%%%%%%%%%%%%%%%%%%%%%%%%%%%%%%%%%%
%%%%%%%%%%%%%%%%%%% Visualization Frameworks %%%%%%%%%%%%%%%%%%%%%
%%%%%%%%%%%%%%%%%%%%%%%%%%%%%%%%%%%%%%%%%%%%%%%%%%%%%%%%%%%%%%%%%%
\section{Visualization Frameworks}
Our web application will provide near-real time data visualizations for energy consumption by buildings on campus. This application will need to dynamically create charts and graphs based on energy data from the database. A key to choosing a visualization library will be to find one that can be dynamically created and changed as new data is received from the data acquisition servers, and the ability to create chart templates that can be reused on multiple pages with different input parameters.\\
\textbf{Basic Criteria}
\begin{itemize}
\item Handle dynamic data. 
\item Allow the user to manipulate data by applying filters or selecting data points. 
\item Works well with Angular framework.
\item Templatable.
\end{itemize}
\subsection{D3.js}
%https://www.dashingd3js.com/d3-resources/d3-and-angular
\textit{Repository Commits: 4,104}\\  
\textit{Contributors: 120}\\
D3 allows you to bind arbitrary data to a Document Object Model (DOM), and then apply data-driven transformations to the document \cite{d3_home}. It is extremely well documented and widely used. D3 has a high learning curve for beginners to take head-on, but has a wide array of different visualizations and customization.\\
\textbf{Pros}
\begin{itemize}
\item A lightweight, versatile javascript library that uses ``SVG to create the graphical elements'' and appends them to DOM elements \cite{Schwartz}. 
\item Makes use of javascript functions and DOM controlling functionality to dynamically change the content of the page. 
\item Provides a lot of variety and ability to customize graphics.
\item Widely used and there is a lot of documentation and resources available to assist the learning and development processes.
\end{itemize}
\textbf{Cons}
\begin{itemize}
\item D3 is essentially an API to to manipulate SVG, it is not a charting library in of itself \cite{Sun}.
\item You cannot easily pass a data set into a specified chart type like other libraries.
\item A large amount of ``up-front investment in time to get a handle on the D3 language''\cite{Jacobson}.
\item Angular and D3 both attempt to control the DOM and so you have to find a way to make the two work together which is counter intuitive to both framework's APIs. 
\end{itemize}
\subsection{Vis.js}
\textit{Repository Commits: 3,165}\\
\textit{Contributors: 137}\\
Vis.js is a lightweight charting library that allows users to create clean charts from dynamic data sets \cite{vis}. It is responsive and allows for interaction with the data on the page. Vis.js is praised for it's network chart capabilities but is limited in the number of different modules you can create.\\
\textbf{Pros}
\begin{itemize}
\item Easy to use and less of a learning curve than D3.
\item Allows for interaction and manipulation of data on the chart.
\item Able to handle large amounts of dynamic data.
\item Really clean and nice looking graphics.
\end{itemize}
\textbf{Cons}
\begin{itemize}
\item Limited amount of possible chart types.
\item Does not have built in heat map.
\end{itemize}
\subsection{Chart.js}
\textit{Repository Commits: 2,465}\\ 
\textit{Contributors: 236}\\
Chart js is a very lightweight library that provides 6 chart types and fully responsive designs. ChartJS is well documented and easy to use, but lacks in variety.\\
\textbf{Pros}
\begin{itemize}
\item Uses HTML5 canvas element.
\item Allows for easy creating based on chart type specification.
\item Library provides Line Charts, Bar Charts, Radar Charts, Pie Charts, Polar Area Charts, and Doughnut Charts.
\item Very responsive charts based on screen width.
\item Simple API, easy to use.
\end{itemize}
\textbf{Cons}
\begin{itemize}
\item Limited amount of possible chart types.
\item Does not have built in heat map.
\end{itemize}
\subsection{Conclusion}
In conclusion, despite the steep learning curve associated with D3.js we think it will be the best option for our web application. It has the widest range of available graphs to accommodate all the client's requirements and desired visualizations. There are also a number of wrapper libraries available for D3.js like DC.js and dimple.js to help create charts from D3. This is a great way to get around the inconveniences and downsides to D3.js and reap the benefits of all the other charting libraries. Another benefit to using D3 is the extensive amount of templates, examples, and documentation that exists to help guide the process and implementation of our application.\\
It is possible that our decision will change as we begin creating the web application and find other libraries to fit our needs. There are a lot of AngularJS libraries that specify unique directives for implementing different charting libraries and we might find one throughout the midst of our development that handles all of our needs swiftly and effectively. One of the main factors that guided our decision towards D3 was the ability to create heatmaps, which was a client requirement. If this requirement falls through, or we simply use D3 only for heatmaps, we may be able to implement a more lightweight charting library to achieve better simplicity.
%%%%%%%%%%%%%%%%%%%%%%%%%%%%%%%%%%%%%%%%%%%%%%%%%%%%%%%%%%%%%%%%%%
%%%%%%%%%%%%%%%%%%%%%%%% Authentication %%%%%%%%%%%%%%%%%%%%%%%%%%
%%%%%%%%%%%%%%%%%%%%%%%%%%%%%%%%%%%%%%%%%%%%%%%%%%%%%%%%%%%%%%%%%%
\section{Means of Incorporating Authentication}
Our web application will have an authentication layer which will allow users to register for our application and design their own ``stories.'' Our application will also authenticate user roles so that administrative users will have access to exclusive parts of the application. We want a simple way of authenticating users, while keeping personal information safe.\\
\textbf{Basic Criteria}
\begin{itemize}
\item Remember users across web pages (sessions). 
\item Allow users to be added to the database. 
\item Protect sensitive information like passwords.
\end{itemize}
\subsection{Building Our Own Authentication Layer}
Node.js has a lot of helpful modules and packages that allow you to create your own password hashing functions and generate a custom authentication layer. There is a ``crypto'' module that is included in Node.js that ``provides cryptographic functionality that includes a set of wrappers for OpenSSL's hash, HMAC, cipher, decipher, sign and verify functions''\cite{Crypto}. In addition to the crypto module, there are numerous Node.js extension modules that perform different hashing functions and provide the same functionality as the native crypto module. There are modules like \href{https://www.npmjs.com/package/bcrypt-Node.js}{\textit{bcrypt}} and \href{https://www.npmjs.com/package/scrypt}{\textit{scrypt}} that use different hashing algorithms to salt and hash password into a fixed length string to be stored into the database.
\begin{lstlisting}[
    caption={[An example of how to salt hash passwords using the Node.js crypto module]An example of how to salt hash passwords using the Node.js crypto module (Taken from \href{https://ciphertrick.com/2016/01/18/salt-hash-passwords-using-Node.js-crypto/}{\textit{Rahil Shaikh's example}})\cite{Rahil_Shaikh}}
    ]
    'use strict';
    var crypto = require('crypto');
    /**
     * generates random string of characters i.e salt
     */
    var genRandomString = function(length){
        return crypto.randomBytes(Math.ceil(length/2))
                .toString('hex') /** convert to hexadecimal format */
                .slice(0,length);   /** return required number of characters */
    };
    /** hash password with sha512.
     * @function
     * @param {string} password - List of required fields.
     * @param {string} salt - Data to be validated.
     */
    var sha512 = function(password, salt){
        var hash = crypto.createHmac('sha512', salt); /** Hashing algorithm sha512 */
        hash.update(password);
        var value = hash.digest('hex');
        return {
            salt:salt,
            passwordHash:value
        };
    };
    function saltHashPassword(userpassword) {
        var salt = genRandomString(16); /** Gives us salt of length 16 */
        var passwordData = sha512(userpassword, salt);
        console.log('UserPassword = '+userpassword);
        console.log('Passwordhash = '+passwordData.passwordHash);
        console.log('nSalt = '+passwordData.salt);
    }
\end{lstlisting}
\textbf{Pros}
\begin{itemize}
    \item Creating our own authentication layer would provide us full control over how passwords are hashed and stored into the database. 
    \item Provide understanding of every component that goes into our application's authentication.
    \item Do not have to rely on another API to authenticate users.
\end{itemize}
\textbf{Cons}
\begin{itemize}
    \item Laborious work and very time consuming.
    \item A lot of room for error and the possibility of data being compromised.
\end{itemize}
\subsection{Outsourcing Authentication to Google}
Google oAuth 2.0 is a Google API that authenticates users by signing in with their google accounts. There is a lot of documentation about how to integrate Google's authentication API with Node.js. \href{http://www.passportjs.org/}{\textit{Passport}} is authentication middleware for Node.js that provides simple authentication using ``A comprehensive set of strategies support authentication using a username and password, Facebook, Twitter,'' and Google \cite{passport}.
\begin{lstlisting}[
    caption={[Using passport to include Google oAuth 2.0 authentication]Using passport to include Google oAuth 2.0 authentication. (Taken from \href{http://www.passportjs.org/docs/username-password}{\textit{Passport Documentation}})\cite{passport_doc}}
    ]
    var passport = require('passport');
    var GoogleStrategy = require('passport-google-oauth').OAuth2Strategy;
    
    // Use the GoogleStrategy within Passport.
    //   Strategies in Passport require a `verify` function, which accept
    //   credentials (in this case, an accessToken, refreshToken, and Google
    //   profile), and invoke a callback with a user object.
    passport.use(new GoogleStrategy({
        clientID: GOOGLE_CLIENT_ID,
        clientSecret: GOOGLE_CLIENT_SECRET,
        callbackURL: "http://www.example.com/auth/google/callback"
      },
      function(accessToken, refreshToken, profile, done) {
           User.findOrCreate({ googleId: profile.id }, function (err, user) {
             return done(err, user);
           });
      }
    ));
\end{lstlisting}
\textbf{Pros}
\begin{itemize}
    \item Takes all the security risks out of creating our own authentication.
    \item Saves a lot of development time. 
    \item Widely used with a lot of documentation.
    \item Relieves the need to store user passwords in the database.
    \item Can use the passport authentication middleware to simplify authentication even further.
\end{itemize}
\textbf{Cons} 
\begin{itemize}
    \item Limits users to having a google account.
    \item Relies on Google API to be working and running.
    \item Adds dependencies to the project.
\end{itemize}
\subsection{Use CAS (Central Authentication Service)}
CAS is an authentication process that redirects users to a CAS login page like the one that Oregon State University uses for most things like Canvas, Box, and MyDegrees.\\

CAS uses sessions and a CAS Client server to authenticate users throughout a web application:
\begin{adjustwidth}{2cm}{2cm}
    \textit{\noindent When CAS redirects the authenticated user back to your application, it will append a \{ticket\} parameter to your url.\\
    \noindent The ticket returned to your application is opaque, meaning that it includes no useful information to anyone other than the CAS Server. The only thing that your application can do is send this ticket back to CAS for validation.\\
    \noindent CAS will then either respond that this ticket does not represent a valid user for this service, or will acknowledge that this ticket proves authentication. In the later case, CAS will also supply the user's NetID so that you know the identity of the user \cite{how_cas_works}.}
\end{adjustwidth} 
\textbf{Pros}
\begin{itemize}
    \item Can use the OSU official CAS to provide a nice theme.
    \item Keeps the same centralized CAS server as the rest of Oregon State University's application.
\end{itemize}
\textbf{Cons}
\begin{itemize}
    \item Relies on Oregon State University's CAS server to be operational. 
    \item Adds exteraneous code and implementation.
    \item Not as modular as the Node.js/Passport implementations of oAuth 2.0.
\end{itemize}
\subsection{Conclusion}
After reviewing the different means of including authentication for our web application, we think that using the Passport middleware to implement Google's oAuth 2.0 API would be the best option. It can be easily integrated into Node.js applications, everyone at Oregon State University has a Gmail account, and it removes the need to store any passwords into our database. The Google oAuth 2.0 authentication process uses a token that is received from the Google API that we will store along with the user ID and name into our database.\\
Another large benefit to using the Passport.js and Google oAuth 2.0 authentication process is that passport handles most of the overhead involved with user sessions and per-page authentication. Using a small cookie set that contains the user id, ``Passport will serialize and deserialize user instances to and from the session'' to handle login sessions throughout the application \cite{passport_session}.
\begin{lstlisting}[
    caption={[The user ID is serialized to the session, keeping the amount of data stored within the session small. When subsequent requests are received, this ID is used to find the user, which will be restored to req.user.]The user ID is serialized to the session, keeping the amount of data stored within the session small. When subsequent requests are received, this ID is used to find the user, which will be restored to req.user. (Taken from \href{http://www.passportjs.org/docs}{\textit{Passport Documentation}})\cite{passport_session}}
    ]
    passport.serializeUser(function(user, done) {
        done(null, user.id);
      });
      
      passport.deserializeUser(function(id, done) {
        User.findById(id, function(err, user) {
          done(err, user);
        });
      });
\end{lstlisting}

Using Google authentication and the Passport middleware will drastically simplify the authentication process of our application and keep the amount of sensitive data in our database to a minimum.
%%%%%%%%%%%%%%%%%%%%%%%%%%%%%%%%%%%%%%%%%%%%%%%%%%%%%%%%%%%%%%%%%%
%%%%%%%%%%%%%%%%%%%%% Front-end Framework %%%%%%%%%%%%%%%%%%%%%%%%
%%%%%%%%%%%%%%%%%%%%%%%%%%%%%%%%%%%%%%%%%%%%%%%%%%%%%%%%%%%%%%%%%%
\section{Front-end Framework}
Our web application will be a series of dashboards to display energy data based on different buildings and subsets of buildings on Oregon State University's campus. A key part to designing a clean dashboard is having a well-spaced grid-like layout with different charts and graphs to display a multitude of different data sets and trends. Rather than customizing classic html elements and using div containers to space our dashboard, we wanted to look into bootstrapped dashboard templates that allow easy customization and clean-looking results.\\ 
\textbf{Basic Criteria}
\begin{itemize}
\item Integrates well with Angular and the entire MEAN stack. 
\item Components can be easily integrated with Visualization Framework. 
\item Customizeable.
\end{itemize}
\subsection{CSS Bootstrap}
\textit{Repository Commits: 17,255}\\ 
\textit{Contributors: 953}\\
CSS Bootstrap is the most widely used CSS framework available. It has the most expansive component list and helps developers create clean and beautiful UX/UI in minimal time frames. Bootstrap allows developers to get their publications up and running quickly without having to worry about front-end styling. A major issue with Bootstrap is that it tends to look very similar across applications and leaves little room for simple customization.\\
\begin{lstlisting}[
    caption={[An example of using UI Bootstrap to create a datepicker object with an Angular controller.]An example of using UI Bootstrap to create a datepicker object with an Angular controller. (Taken from \href{https://codepen.io/joe-watkins/pen/KsAgp}{\textit{Angular - Bootstrap UI - Datepicker}})\cite{Watkins}}
    ]
    // HTML Declaration
    <input type="text" class="form-control" 
        datepicker-popup="{{format}}" 
        ng-model="dt" 
        is-open="opened" 
        min-date="minDate" 
        max-date="'2015-06-22'"
        datepicker-options="dateOptions" 
        date-disabled="disabled(date, mode)" 
        ng-required="true" 
        close-text="Close" 
        id="date-picker" 
        readonly
        ng-click="open($event)"
    />

    // Angular Controller
    var calPicker = angular.module("calPicker", ['ui.bootstrap']);
    
    calPicker.controller("DatepickerDemoCtrl", ["$scope", function($scope){
        
        // grab today and inject into field
        $scope.today = function() {
        $scope.dt = new Date();
        };
        
        // run today() function
        $scope.today();
    
        // setup clear
        $scope.clear = function () {
        $scope.dt = null;
        };
    
        // open min-cal
        $scope.open = function($event) {
        $event.preventDefault();
        $event.stopPropagation();
    
        $scope.opened = true;
        };
        
        // handle formats
        $scope.formats = ['dd-MMMM-yyyy', 'yyyy/MM/dd', 'dd.MM.yyyy', 'shortDate'];
        
        // assign custom format
        $scope.format = $scope.formats[0];
        
    }]);
\end{lstlisting}
\textbf{Pros}
\begin{itemize}
    \item Extensive component list, responsive design, and built-in Javascript functions.\cite{puranjay_2015}
    \item EFully responsive design.
    \item Huge developer contrbution and maintainence.
    \item Used by major companies like Lyft.com, Vogue.com, Vevo.com, and Newsweek.com. \cite{puranjay_2015}
\end{itemize}
\textbf{Cons}
\begin{itemize}
    \item Unsuitable for small scale projects. \cite{puranjay_2015}
    \item Not good if you want to have large control over UI.
\end{itemize}
\subsection{Pure CSS}
\textit{Repository Commits: 541}\\ 
\textit{Contributors: 51}\\
Pure CSS is known for its lightness and simplicity. Because it's small, it is very fast loading and makes for a lightweight web application. It is also unique in that it allows modules/components to be downloaded individually, reducing its size even more. Pure CSS is good for small projects that need to get up and running quickly and easily.\\
\textbf{Pros}
\begin{itemize}
    \item Meant for small project to get up and running quickly.
    \item Responsive design by default.
    \item Pure CSS is modular so you can download only the components you need.
    \item The complete module is very small so it is quick loading.
    \item Able to be used complimentary with other frameworks.
\end{itemize}
\textbf{Cons} 
\begin{itemize}
    \item Not as extensive component list as Bootstrap.
\end{itemize}
\subsection{Foundation}
\textit{Repository Commits: 15,094}\\ 
\textit{Contributors: 959}\\
Framework is the second most popular css framework on the market behind Bootstrap and tends to perform just as well. Unlike bootstrap's very prominent theme, Foundation allows for more customization with the look and feel of web pages \cite{Nick_Pettit}. Foundation also has a very good grid system to make the layout of components clean and responsive \cite{blankenship_2017}.\\ 
\textbf{Pros}
\begin{itemize}
    \item Responsive design by default.
    \item Easier to customize than Bootstrap.
    \item CSS classes are built in. \cite{team_2015}
    \item More unique look than the more-popular Bootstrap.
    \item Good grid implementations with customizable grid layouts.
\end{itemize}
\textbf{Cons}
\begin{itemize}
    \item Less maintained than Bootstrap.
    \item Lack of support.
    \item Higher learning curve than Bootstrap.
\end{itemize}
\subsection{Conclusion}
A lot of online resources acknowledged the fact that it is hard to make a website not look like Bootstrap when using Bootstrap css. Despite this, we are choosing to use Bootstrap as the main front-end framework for our application. Similar to our reasoning behind choosing D3.js, we would like to use a framework that is heavily supported and well-documented, as it will be easier to answer questions during development. We think that the time spent restyling Bootstrap to make it look unique will not compare to the time saved by utilizing a well-documented framework. There is also a project called \href{https://angular-ui.github.io/bootstrap/}{UI Bootstrap} that works with AngularJS to create directives for each of the Bootstrap components \cite{sevilayha_2015}.
\hypersetup{
    colorlinks = true,
    urlcolor = black,
    linkcolor = black,
    citecolor = black,
    pdfauthor = {\name},
    pdfkeywords = {CS461 ``Senior Capstone'' Technology Review},
    pdftitle = {CS461 ``Senior Capstone'' Technology Review},
    pdfsubject = {CS461 ``Senior Capstone'' Technology Review},
    pdfpagemode = UseNone
  }
\bibliographystyle{IEEEtran}
\bibliography{ref}
\end{document}





%%%%%%%%%%%%%%%%%%%%%%%%%%%%%%%%%%%%%%%%%%%%%%%%%%%%%%%%%%%%%%%%%%
%%%%%%%%%%%%%%%%%%%%% Old Section to Ignore %%%%%%%%%%%%%%%%%%%%%%
%%%%%%%%%%%%%%%%%%%%%%%%%%%%%%%%%%%%%%%%%%%%%%%%%%%%%%%%%%%%%%%%%%
\iffalse
\section{Password Hashing Algorithms}
Our web application will have an authentication layer which will allow administrative users to have access to exclusive parts of the application. Anytime user information is stored into a database, it is important to hash the passwords and keep user credentials encrypted.
\subsection{PBKDF2}
The PBKDF2 algorithm is widely accepted as it is published by the RSA (RSA Security LLC). It provides a pseudorandom function and a salt value over many iteartion to create a derived key which is used in conjunction with a known key like a password to effectively hash that value into the database. It's many iterations make it difficult for hackers to break.
\textbf{Pros}
\begin{itemize}
    \item PBKDF2 allows for multiple iterations, and adding salted random input on any number of iterations \cite{jarmoc_2015}.
    \item RSA standard.    
\end{itemize}
\textbf{Cons}
\begin{itemize}
    \item Unsafe because PBKDF2 can be thoroughly optimized with GPU \cite{pornin_2012}.
    \item Complex API and slow computationally.
\end{itemize}
\subsection{Bcrypt}
The Bcrypt algorithm performs quite the same way as PBKDF2 except it makes use of table altering in its algorithm that highlights a fault in PBKDF2 where GPU's can break the encryption. Bcrypt has been around for a long time and is widely accepted as an ``unbreakable'' hashing method.\\
\textbf{Pros}
\begin{itemize}
    \item Can provide multiple hash iterations to strengthen the security.
    \item Very secure hash that can hash the same password multiple times.
    \item Widely used today and remains unbroken.
    \item Vetted by the entire crypto community as it’s now 15 years old \cite{medium}.
\end{itemize}
\textbf{Cons} 
\begin{itemize}
    \item Slow and computationally expensive hashing.
    \item Only used for password hashing, not a key-derivation function. 
\end{itemize}
\subsection{Scrypt}
Script works the same way as Bcrypt and PBKDF2 by iteratively hashing and using a salt value to create a hashed password. Scrypt is unique in that it take exponential memory to decypher the hash as well as exponential time for every iteration over the hashing algorithm which makes attackers even more unlikely to decypher passwords. Although Scrypt has been proven effective, it is still fairly new to the market and has not been totally varified.\\
\textbf{Pros}
\begin{itemize}
    \item Can provide multiple hash iterations to strengthen the security.
    \item Newly developed based on focusing on the issues with BCrypt and PBKDF2 involving constant memory.
\end{itemize}
\textbf{Cons}
\begin{itemize}
    \item New and not widely accepted by security/cryptographic professionals.
    \item Uses exponential time AND exponential memory which may be overkill for our application.
\end{itemize}
\subsection{Conclusion}
There is a lot of discussion and documentation involving password hashing algorithms and which ones are the ``safest.'' For our web application, we will not be handling sensitive user data like credit card information or personal information, so we do not need the most comprehansive and computationally heavy hashing algorithm to encrypt our user passwords. For this, it would be best to use BCrypt because it is so widely used, has proven to be secure, and easy to implement. BCrypt has its own npm library and can be easily configure to run on a Node.js environment. It also provides secure, salted hashed that can be iterated over a configurable amount to increase the hashes effectiveness.
\fi