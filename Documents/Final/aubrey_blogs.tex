\subsubsection{Fall 2017}
\paragraph{Week 1}
\begin{adjustwidth}{2.5em}{0pt}
    \vspace{-0.5cm}\textbf{Plans:}
    \vspace{-0.5cm}
    \begin{itemize}
        \item Submit project preferences
    \end{itemize} 
    \vspace{-0.3cm}\textbf{Progress:}
    \vspace{-0.5cm}
    \begin{itemize}
        \item Decided on my top project choices
        \item Researched different projects to figure out what I wanted to do
    \end{itemize} 
    \vspace{-0.3cm}\textbf{Problems:}
    \vspace{-0.5cm}
    \noindent No problems to report   
    \vspace{-0.3cm}\noindent\textbf{Summary:}\\
    \noindent This week was an introductory period where we went over the course outline and details in lecture. We were told to select a few of our top choices for projects.
\end{adjustwidth}
\paragraph{Week 2}
\begin{adjustwidth}{2.5em}{0pt}
    \vspace{-0.5cm}\textbf{Plans:}
    \vspace{-0.5cm}
    \begin{itemize}
        \item Get together with team and talk with client
        \item Go over project scope 
    \end{itemize} 
    \vspace{-0.3cm}\textbf{Progress:}
    \vspace{-0.5cm}
    \begin{itemize}
        \item Got everything initialized between team and client
    \end{itemize} 
    \vspace{-0.3cm}\textbf{Problems:}
    \vspace{-0.5cm}
    \begin{itemize}
        \item Nobody seems to know how we are going to implement all of this
    \end{itemize}  
    \vspace{-0.3cm}\noindent\textbf{Summary:}\\
    \noindent This week our team set up our Slack channel for communication purposes.  We also were able to get a hold of our client to schedule a meeting for the next week. Got access to current implementation of web interface we will be replacing. 
	\end{adjustwidth} 
\paragraph{Week 3}
\begin{adjustwidth}{2.5em}{0pt}
    \vspace{-0.5cm}\textbf{Plans:}
    \vspace{-0.5cm}
    \begin{itemize}
        \item Work on problem statement assignment 
    \end{itemize} 
    \vspace{-0.3cm}\textbf{Progress:}
    \vspace{-0.5cm}
    \begin{itemize}
        \item It was good to finally meet our client and discuss our project in more detail.
    \end{itemize} 
    \vspace{-0.3cm}\textbf{Problems:}
    \vspace{-0.5cm}
    \begin{itemize}
        \item After our client meeting, we had a general sense of what would be delivered as a final product. However, we still had no idea how or what we were going to do to implement it. One of the main problems that arose from the meeting was the discussion of hardware for our project. We had no idea where we were going to store our database and website. As a solution, Jack told us he would talk to his supervisors and IT group to figure it out.
    \end{itemize}  
    \vspace{-0.3cm}\noindent\textbf{Summary:}\\
    \noindent This week involved a general setup of resources for the rest of the term including our Github repository and our OneNote Notebooks. We also met with our client, Jack Woods, at his office in the Office of Sustainability on campus to discuss details about the project and their visions for the web application.
	\end{adjustwidth} 
\paragraph{Week 4}
\begin{adjustwidth}{2.5em}{0pt}
    \vspace{-0.5cm}\textbf{Plans:}
    \vspace{-0.5cm}
    \begin{itemize}
		\item Start on requirements doc
    \end{itemize} 
    \vspace{-0.3cm}\textbf{Progress:}
    \vspace{-0.5cm}
    \begin{itemize}
        \item Finished the Problem Statement final draft and submitted on Github and OneNote. 
        \item Emailed our client requesting a list of requirements he and his team want to see in the final application to begin working on the Requirements Document assignment.
        \item This was the first week we were required to meet with our TA, Ben, where we discussed the structure and organization of directories in our group repository and how to create a logical labeling system. 
        \item Made a plan of action to begin researching the MEAN stack and its different frameworks.  
		\end{itemize} 
    \vspace{-0.3cm}\textbf{Problems:}
    \vspace{-0.5cm}
    \begin{itemize}
        \item Issues with LaTeX formatting and compiling, but was solved eventually. 
    \end{itemize}  
    \vspace{-0.3cm}\noindent\textbf{Summary:}\\
    \noindent Finishing the Problem Statement doc helped our team develop a general sense of the project. 
\end{adjustwidth} 
\paragraph{Week 5}
\begin{adjustwidth}{2.5em}{0pt}
    \vspace{-0.5cm}\textbf{Plans:}
    \vspace{-0.5cm}
    \begin{itemize}
        \item Finish up Final Draft of Requirements doc  
        \item Learn about more about MEAN dev
    \end{itemize} 
    \vspace{-0.3cm}\textbf{Progress:}
    \vspace{-0.5cm}
    \begin{itemize}
        \item Got a rough draft of the Requirements Document completed and submitted on Github and OneNote.
        \item Met with our client to discuss further details about the components listed in his requirements list.
        \item Our client also said he mailed his IT group about potential databases and servers that could be used for our application, but they had not responded yet. 
    \end{itemize} 
    \vspace{-0.3cm}\textbf{Problems:}
    \vspace{-0.5cm}
    \begin{itemize}
        \item We had some severe issues with creating a Gantt chart into our document. The Gantt LaTeX package just would not work properly. 
		\item We also had a few issues with Github merge conflicts. 
    \end{itemize}  
    \vspace{-0.3cm}\noindent\textbf{Summary:}\\
    \noindent Requirements doc further solidified our understanding of the project.
	\end{adjustwidth} 
\paragraph{Week 6}
\begin{adjustwidth}{2.5em}{0pt}
    \vspace{-0.5cm}\textbf{Plans:}
    \vspace{-0.5cm}
    \begin{itemize}
        \item Create and mess with simple MEAN app 
        \item Research Database and website hosting. 
    \end{itemize} 
    \vspace{-0.3cm}\textbf{Progress:}
    \vspace{-0.5cm}
    \begin{itemize}
        \item We got a Latex Gantt package to work in our Requirements Document and submitted final draft to Github and OneNote. 
        \item Contacted our client to review and approve our Requirements Document and send a confirmation email to Kirsten Winters and Kevin McGrath.
        \item Our group assigned three components to each member to research for our individual Technology Review documents.
		\item TA meeting revolved around clearing up the final questions about the Requirements Document and discussing our current actions required model.
    \end{itemize} 
    \vspace{-0.3cm}\textbf{Problems:}
    \vspace{-0.5cm}
    \begin{itemize}
        \item No problems to report
    \end{itemize}  
    \vspace{-0.3cm}\noindent\textbf{Summary:}\\
    \noindent  We were very content with the communication and rate of response from our client; it simplified our workflow and made accomplishing our tasks very trivial. Finishing requirements doc helped us immensely in figuring out what needed to be done for rest of project. 
	\end{adjustwidth} 
\paragraph{Week 7}
\begin{adjustwidth}{2.5em}{0pt}
    \vspace{-0.5cm}\textbf{Plans:}
    \vspace{-0.5cm}
    \begin{itemize}
        \item Meet with client to go over hosting services
    \end{itemize} 
    \vspace{-0.3cm}\textbf{Progress:}
    \vspace{-0.5cm}
    \begin{itemize}
        \item Did heavy research into webhosting and database options for the project.
        \item This was all preliminary research done ahead of time to present to our client for a meeting next week. I specifically focused on finding information on Amazon AWS. 
    \end{itemize} 
    \vspace{-0.3cm}\textbf{Problems:}
    \vspace{-0.5cm}
    \begin{itemize}
        \item We have had trouble getting our client and his office to set up an AWS account to host our application. 
		\item He ensured us that he would get it taken care of, but we are currently stuck with only local development strategies in the meantime. 
	\end{itemize}  
    \vspace{-0.3cm}\noindent\textbf{Summary:}\\
    \noindent Doing research into AWS will help us quite bit, both for our meeting next week and tech doc. 
\end{adjustwidth} 
\paragraph{Week 8}
\begin{adjustwidth}{2.5em}{0pt}
    \vspace{-0.5cm}\textbf{Plans:}
    \vspace{-0.5cm}
    \begin{itemize}
        \item Begin design doc
    \end{itemize} 
    \vspace{-0.3cm}\textbf{Progress:}
    \vspace{-0.5cm}
    \begin{itemize}
        \item This week our group finished the individual Technology Review rough drafts and began working on the final drafts for next week. 
		\item Underwent extensive research on specific technologies that would be incorporated in our application and generated good documentation of examples and resources to help drive future development.  
    \end{itemize} 
	
	
	
	
    \vspace{-0.3cm}\textbf{Problems:}
    \vspace{-0.5cm}
    \begin{itemize}
        \item We had some issues trying to figure out how to handle security. 
		\item After talking to Kevin and Ben, we decided to outsource it to a third party rather than making it ourselves.
    \end{itemize}  
    \vspace{-0.3cm}\noindent\textbf{Summary:}\\
    \noindent This week I finished the tech review rough draft and started really researching nitty gritty details about the dev ops behind this project. I spoke to my boss from my last job and got 3 page of notes about dev ops and total web application architecture. (I will up load this notes next week). I did not have very much time to work on personal MEAN stack projects this week as I was very busy with other work.
\end{adjustwidth} 
\paragraph{Week 9}
\begin{adjustwidth}{2.5em}{0pt}
    \vspace{-0.5cm}\textbf{Plans:}
    \vspace{-0.5cm}
    \begin{itemize}
        \item Complete design doc
    \end{itemize} 
    \vspace{-0.3cm}\textbf{Progress:}
    \vspace{-0.5cm}
    \begin{itemize}
        \item Submitted Final Draft of Tech Review 
        \item Made template for Design Doc 
    \end{itemize} 
    \vspace{-0.3cm}\textbf{Problems:}
    \vspace{-0.5cm}
    \begin{itemize}
        \item The IEEE format for the design doc was hard to understand and we needed to talk to Kirsten about it. 
    \end{itemize}  
    \vspace{-0.3cm}\noindent\textbf{Summary:}\\
    \noindent Very short week due to holiday.
	\end{adjustwidth} 
\paragraph{Week 10}
\begin{adjustwidth}{2.5em}{0pt}
    \vspace{-0.5cm}\textbf{Plans:}
    \vspace{-0.5cm}
    \begin{itemize}
        \item Design and Code project over break. 
    \end{itemize} 
    \vspace{-0.3cm}\textbf{Progress:}
    \vspace{-0.5cm}
    \begin{itemize}
        \item  Wrote and submitted design doc 
    \end{itemize} 
    \vspace{-0.3cm}\textbf{Problems:}\\
	\vspace{-0.5cm}
	\begin{itemize}
		\item We had some troubles with explicit definitions of terms in our requirements from our client. 
		\item We had multiple conversation and messages from our client where he used the terms ``dashboard'' and ``page'' interchangeably. After meeting with him, we cleared up the confusion and were able to establish a finalized dictionary of terms. 
		\item Another issue we faced was the design of data models that presented themselves in the Design Document and determining how we were going to efficiently store and manage our different data and entities. 
	\end{itemize} 
	\vspace{-0.3cm}\noindent\textbf{Summary:}\\
    \noindent Design doc finalized our understanding of project 
\end{adjustwidth} 
\subsubsection{Winter 2018}
\paragraph{Week 1}
\begin{adjustwidth}{2.5em}{0pt}
    \vspace{-0.5cm}\textbf{Plans:}
    \vspace{-0.5cm}
    \begin{itemize}
        \item Set up times for TA meeting 
		\item Set up Client meeting
    \end{itemize} 
    \vspace{-0.3cm}\textbf{Progress:}
    \vspace{-0.5cm}
    \begin{itemize}
        \item None
    \end{itemize} 
    \vspace{-0.3cm}\textbf{Problems:}
    \vspace{-0.5cm}
    \begin{itemize}
        \item None
    \end{itemize}  
    \vspace{-0.3cm}\noindent\textbf{Summary:}\\
    \noindent Getting back on our feet after break. 
\end{adjustwidth}
\paragraph{Week 2}
\begin{adjustwidth}{2.5em}{0pt}
    \vspace{-0.5cm}\textbf{Plans:}
    \vspace{-0.5cm}
    \begin{itemize}
        \item Make sure our EC2 instance works
    \end{itemize} 
    \vspace{-0.3cm}\textbf{Progress:}
    \vspace{-0.5cm}
    \begin{itemize}
        \item Had first TA meeting of term
		\item Briefly went over break
		\item Had Client Meeting 
    \end{itemize} 
    \vspace{-0.3cm}\textbf{Problems:}
    \vspace{-0.5cm}
    \begin{itemize}
        \item None
    \end{itemize}  
    \vspace{-0.3cm}\noindent\textbf{Summary:}\\
    \noindent Setting plan of action.
	\end{adjustwidth} 
\paragraph{Week 3}
\begin{adjustwidth}{2.5em}{0pt}
    \vspace{-0.5cm}\textbf{Plans:}
    \vspace{-0.5cm}
    \begin{itemize}
        \item Wait for client to get sample data to begin backend 
    \end{itemize} 
    \vspace{-0.3cm}\textbf{Progress:}
    \vspace{-0.5cm}
    \begin{itemize}
        \item Found packages to use on backend 
	\end{itemize} 
    \vspace{-0.3cm}\textbf{Problems:}
    \vspace{-0.5cm}
    \begin{itemize}
        \item A lot of hardware problems with my computer set me behind schedule quite a bit. 
    \end{itemize}  
    \vspace{-0.3cm}\noindent\textbf{Summary:}\\
    \noindent Continuing to build plan of action. Waiting patiently for client to get back to me so I can start on backend data collection.
\end{adjustwidth} 
\paragraph{Week 4}
\begin{adjustwidth}{2.5em}{0pt}
    \vspace{-0.5cm}\textbf{Plans:}
    \vspace{-0.5cm}
    \begin{itemize}
        \item Work on backend parsing
    \end{itemize} 
    \vspace{-0.3cm}\textbf{Progress:}
    \vspace{-0.5cm}
    \begin{itemize}
        \item Working on buildings. 
		\item Got templates working fully  
    \end{itemize} 
    \vspace{-0.3cm}\textbf{Problems:}
    \vspace{-0.5cm}
    \begin{itemize}
        \item Very bad health issues left me incapacitated 
	\end{itemize}  
    \vspace{-0.3cm}\noindent\textbf{Summary:}\\
    \noindent Finally could get some work done to server.
\end{adjustwidth} 
\paragraph{Week 5}
\begin{adjustwidth}{2.5em}{0pt}
    \vspace{-0.5cm}\textbf{Plans:}
    \vspace{-0.5cm}
    \begin{itemize}
        \item Continue implementing backend of our application.
		\item Implement duplicate prevention.
    \end{itemize} 
    \vspace{-0.3cm}\textbf{Progress:}
    \vspace{-0.5cm}
    \begin{itemize}
        \item Pushed out a bunch of updates to the backend. 
		\item Our software can now read and add entries to database. 
    \end{itemize} 
    \vspace{-0.3cm}\textbf{Problems:}
    \vspace{-0.5cm}
    \begin{itemize}
        \item Had some conflicts with team dynamics, but we resolved those quickly.
    \end{itemize}  
    \vspace{-0.3cm}\noindent\textbf{Summary:}\\
    \noindent Huge server changes made.
\end{adjustwidth} 
\paragraph{Week 6}
\begin{adjustwidth}{2.5em}{0pt}
    \vspace{-0.5cm}\textbf{Plans:}
    \vspace{-0.5cm}
    \begin{itemize}
        \item Fix duplicate prevention
    \end{itemize} 
    \vspace{-0.3cm}\textbf{Progress:}
    \vspace{-0.5cm}
    \begin{itemize}
        \item Pushed out a bunch of updates to the backend. 
		\item Finished progress report. 
    \end{itemize} 
    \vspace{-0.3cm}\textbf{Problems:}
    \vspace{-0.5cm}
    \begin{itemize}
        \item Having issues with my duplicate data check not comparing format correctly.
    \end{itemize}  
    \vspace{-0.3cm}\noindent\textbf{Summary:}\\
    \noindent Continued implementing backend of our application.
\end{adjustwidth} 
\paragraph{Week 7}
\begin{adjustwidth}{2.5em}{0pt}
    \vspace{-0.5cm}\textbf{Plans:}
    \vspace{-0.5cm}
    \begin{itemize}
        \item Implement addMeters 
		\item See if I can get addBuildings to work 
    \end{itemize} 
    \vspace{-0.3cm}\textbf{Progress:}
	\noindent This week I worked a lot on routing to make it easier to access schema data after POST data comes in.
    \vspace{-0.5cm}
    \begin{itemize}
        \item Implemented functional routes to add data to new schema
		\begin{itemize}
			\item This involved changing the structure of the DB access order
			\item Used to look for building->meter->entry 
			\item Now looks for meter->building->entry 
		\end{itemize}
		\item Changes made makes it easier to find necessary data without having to make another call later 
		\item Changed HTML/controller files to match new schemas 
    \end{itemize} 
    \vspace{-0.3cm}\textbf{Problems:}
    \vspace{-0.5cm}
    \begin{itemize}
        \item None
    \end{itemize}  
    \vspace{-0.3cm}\noindent\textbf{Summary:}\\
    \noindent Decently busy week. Not much code went out, but it sure does a lot.
	\end{adjustwidth} 
\paragraph{Week 8}
\begin{adjustwidth}{2.5em}{0pt}
    \vspace{-0.5cm}\textbf{Plans:}
    \vspace{-0.5cm}
    \begin{itemize}
        \item Get LIVE data working on EC2 instance 
		\item Finish implementing addMeter from XML POST route 
		\item Change 'Edit Meter' controller to add data entries to new building when added. 
    \end{itemize} 
    \vspace{-0.3cm}\textbf{Progress:}
    \vspace{-0.5cm}
    \begin{itemize}
		\item Changed backend functionality in how it adds data entries
		\item Changed building html pages to reflect above changes
		\item Implemented controllers for edit buildings (work in progress) 
		\item Fixed 'Add Building' to remove meters from old buildings when added to new.
		\item Moved XML POST route to dataServer.js
		\item Fixed duplicate prevention 
    \end{itemize} 
    \vspace{-0.3cm}\textbf{Problems:}
    \vspace{-0.5cm}
	\begin{itemize}
    \item The `addMeter()` function I was implementing will require some tweaking of the meter schema.
	\item As the flow of the `receiveXML()` works right now, it finds: 
	\begin{itemize}
		\item `meter->building with that meter->pushes data entry to that building` 
		\item `addMeter()` will have a building set to null, so referencing it will crash program. There's no way for us to be able tell which building a meter goes to (at least I think) from just the XML data 
	\end{itemize}
	\item  My solution will be to add a data entry array to the meter schema. I would just add a check: 
	\begin{itemize}
		\item If the meter's building is null, then the meter data entry array will push the entry
	\end{itemize}
	\item Then, once the meter gets added to a building (via edit meter or edit building), then all the contents of a meter's data entries would be pushed into that building and popped out of the meter's data entries (to prevent duplicate entries if it were to change buildings) 
  \end{itemize}
  \vspace{-0.3cm}\noindent\textbf{Summary:}\\
    \noindent A lot of work was completed this week and we now have AcquiSuites posting to our EC2 Instance.
\end{adjustwidth} 
\paragraph{Week 9}
\begin{adjustwidth}{2.5em}{0pt}
    \vspace{-0.5cm}\textbf{Plans:}
    \vspace{-0.5cm}
    \begin{itemize}
        \item Plans for next week is to finish up add/delete different components. Each of those should take a while. 
    \end{itemize} 
    \vspace{-0.3cm}\textbf{Progress:}
    \vspace{-0.5cm}
    \begin{itemize}
     \item Met with client and debugged connection to EC2 instance and discussed current affairs
	 \item Controllers for buildings/meters 
	 \item Completely finished data parsing and functionality of adding entries to entries DB 
	 \item Updated DBSample.js to push dummy data to local DB for graph testing by creating function in DBSample.js that uses RNG to generate timestamps and readings then adds them to DB. This allows Dan and Parker to work on graphs.
	 \item Implemented functionality to remove meters from old buildings when it is added to new buildings. This is a very important fix because it prevents data from going to the wrong building!
	 \item Finished 'addMeter()' in dataServer.js. This function adds a meter if it doesn't currently exist in the database. It uses data from XML to do so.
	 \item Modularized XML function with async function calls to do various thing. Implemented this in case server gets hit with two POST requests at the same time. If the function blocks completely, it might reject data, which in turn could cause data loss. 
	 \item split XML function into addMeter() and addEntry()
	 \item addEntry() is what actually pushes and updates 
	 \item Fixed error that occurred when buildings were null when trying to add entries 
	 \item Implemented functionality to let us keep track of data when it comes to our server before a building/meter is added to it. 
	 \begin{itemize}
		\item Creates meter for data entry. Then it sets that meter as the DataEntries' Meter.
		\item This function is called when a building is added when a meter.building is null
		\item Pushes all the the matching data entries to the building array 
	 \end{itemize} 
	 \item Implemented functionality to handle different types of XML requests. Essentially just makes sure it's the log file and appropriately formatted before trying to do anything with it. 
	\end{itemize} 
    \vspace{-0.3cm}\textbf{Problems:}
    \vspace{-0.5cm}
    \begin{itemize}
        \item I ran into a LOT of problems this week after we migrated our project to the EC2 instance.  
		\item XML POST requests weren't always formatted the same. 
		\item XML POST data was just completely different than the sample data I've had all term. 
		\item Specifically there was multiple <record> entries, which I've only seen the XML contain one at a time This threw how addEntry() worked completely out the window. 
		\item These two problems along with some minor things took about a good 10 hours to resolve. 
		\item To resolve these issues, I added a check mentioned above. This was an easy one. 
		\item The second one was not so great to fix. I ended up having to wrap everything in a promise, create a DataEntry array and add each <record> to that. Then it was as simple as iterating through the array and adding to the database. A big time factor was figuring out exactly why it was crashing in the first place. I was able to pull some logs from the server, which helped. It also didn't help that if I had one <record>, it wouldn't be considered an array of <record>. 
    \end{itemize}  
    \vspace{-0.3cm}\noindent\textbf{Summary:}\\
    \noindent I made a ton of progress this week and look forward to wrapping it up.
\end{adjustwidth} 
\paragraph{Week 10}
\begin{adjustwidth}{2.5em}{0pt} 
    \vspace{-0.5cm}\textbf{Plans:}
    \vspace{-0.5cm}
    \begin{itemize}
        \item Finish progress report 
        \item Fix any features that don't work 100% 
        \item Start debugging some of the more high priority bugs
    \end{itemize} 
    \vspace{-0.3cm}\textbf{Progress:}
    \vspace{-0.5cm}
    \begin{itemize}
        \item Removed and fixed merge conflicts someone pushed that caused data retrieval to return some interesting results. 
        \item Implemented a functions that resets the building variable of DataEntry to null when that building is deleted. This will allow past/stored data entries to push correctly to that building if buildings were added incorrectly. 
        \item Related to above, implemented a function that set the building variable of DataEntry to the building id  when being pushed from a meter that was storing data entries. Previously, only new data entries whose meters were already added to a building would be set as such. 
        \item Fixed functionality so when meters were deleted, they would remove themselves from a Buildings array of Meters. 
        \item Implemented and fixed edit/delete functionality of Block, Building, Meter, and Dashboard components. 
        \item Worked on application optimization in some of the functions (they were causing application to be quite slow). 
        \item Implemented main feature that allows users to enter an email address and receive a link from our application that would allow them register. 
        \item Furthermore, implemented the front side of this as well. Also implemented it to only show when user is not Logged in. 
        \item Found temporary fix for our 'null' building bug that keeps popping up. We simply don't load that entry. 
        \item Updated all backend dataServer log statements to provide much more useful. 
        \item Fixed async functions that were causing data to populate strangely. 
        \item Fixed issue that was causing app to crash when any of the components were interacted with but only had one or fewer buildings. 
        \item Implemented main feature that prevents malicious code to be injected into our database. 
        \item Formatted all of our code docs so they had consistent styling. 
        \item Implemented main feature for the upcoming Kilowatt Crackdown event. 


    \end{itemize} 
    \vspace{-0.3cm}\textbf{Problems:}
    \vspace{-0.5cm}
    \begin{itemize}
        \item Null building bug seems to reoccur randomly and the cause of it isn't apparent. 
        \item Visual bug when using date selector for Kilowatt Crackdown 
        \item Registration only works with AWS verified accounts until our AWS is moved out of sandbox mode. 
        \item A lot of front end crashes caused by code that wasn't tested with pre-existed components. 
        \item Data in backend was messed up for a while when client used the MongoDB Compass app to add entries instead of our application. This prevented a lot of the necessary data and functions to fire and make things work correctly. 
    \end{itemize} 
    \vspace{-0.3cm}\noindent\textbf{Summary:}\\
    \noindent Very busy two weeks after an extremely busy week 9. I got a lot of the main features in our requirements doc complete.
\end{adjustwidth} 
\subsubsection{Spring 2018}
\paragraph{Week 1}
\begin{adjustwidth}{2.5em}{0pt}
    \vspace{-0.5cm}\textbf{Plans:}
    \vspace{-0.5cm}
    \begin{itemize}
        \item Figure out why McNary is acting so strangely.
    \end{itemize} 
    \vspace{-0.3cm}\textbf{Progress:}
    \vspace{-0.5cm}
    \begin{itemize}
        \item Set TA schedule time
        \item Over the break, it seemed our server went haywire. We received even more extremely weird entries. 
		\item I spent most of the week debugging the backend of the server to figure out the issue. This really just involved a lot of trial and error, with plenty of conditionals and log statements.
    \end{itemize} 
    \vspace{-0.3cm}\textbf{Problems:}
    \vspace{-0.5cm}
    \begin{itemize}
        \item Lots of problems with only some of the buildings reporting weird issues. This is much worse than if ALL the buildings were reporting errors.
    \end{itemize}  
    \vspace{-0.3cm}\noindent\textbf{Summary:}\\
    \noindent Good progress for first week of the term. Some buildings are acting weird. 
\end{adjustwidth}
\paragraph{Week 2}
\begin{adjustwidth}{2.5em}{0pt}
    \vspace{-0.5cm}\textbf{Plans:}
    \vspace{-0.5cm}
    \begin{itemize}
        \item Fix the email alert functions.
        \item Continue to debug AcquiSuite data and large spikes that are occurring.
    \end{itemize} 
    \vspace{-0.3cm}\textbf{Progress:}
    \vspace{-0.5cm}
    \begin{itemize}
        \item During this week, I noticed that some of the entries were showing negative values. After talking with our client, he determined that this is a hardware issue caused by some of the wires crossing. To fix this, I simply added a statement that would take the absolute value of each entry. 
		\item Added a function that would retroactively change all the negative values to positive. There were a few more 'one time use' functions added. 
		\item Also resolved an issue with an async function causing EC2 to crash.
    \end{itemize} 
    \vspace{-0.3cm}\textbf{Problems:}
    \vspace{-0.5cm}
    \begin{itemize}
        \item Negative values showing where positive values expected.
    \end{itemize}  
    \vspace{-0.3cm}\noindent\textbf{Summary:}\\
    \noindent Bug fixing most of the week. 
\end{adjustwidth} 
\paragraph{Week 3}
\begin{adjustwidth}{2.5em}{0pt}
    \vspace{-0.5cm}\textbf{Plans:}
    \vspace{-0.5cm}
    \begin{itemize}
        \item Keep adding to email functions as well as start on user account stuff.
        \item Finish WIRED Article
    \end{itemize} 
    \vspace{-0.3cm}\textbf{Progress:}
    \vspace{-0.5cm}
    \begin{itemize}
        \item Began working on the WIRED article for class.
        \item Interviewed friend from The Board Room group
        \item This week involved moving some functions around to make more sense. Specifically, I moved our email alert functions to our data retrieval server, as this is where the bulk of the interaction would take place. 
		\item Added more robust functionality to all of the email functions. 
		\item Added functionality so all of these email alerts would occur automatically. This was implemented by checking the timestamp then checking if alert parameters were true. This prevents an email going out every time a data entry comes in.
    
    \end{itemize} 
    \vspace{-0.3cm}\textbf{Problems:}
    \vspace{-0.5cm}
    \begin{itemize}
        \item Troubleshooting AcquiSuite still
    \end{itemize}  
    \vspace{-0.3cm}\noindent\textbf{Summary:}\\
    \noindent A lot of work put into email functionality. 
\end{adjustwidth} 
\paragraph{Week 4}
\begin{adjustwidth}{2.5em}{0pt}
    \vspace{-0.5cm}\textbf{Plans:}
    \vspace{-0.5cm}
    \begin{itemize}
        \item Debug above functionality and ensure it works before midterm presentation. 
		\item Update schema and login functionality to accept new users permissions. 
    \end{itemize} 
    \vspace{-0.3cm}\textbf{Progress:}
    \vspace{-0.5cm}
    \begin{itemize}
        \item Finished WIRED
        \item This week I mainly focused on starting the user access levels. Specifically, this was for the email registration page that allowed admins to add new users with various permissions.
        \item Took group photo for poster
    \end{itemize} 
    \vspace{-0.3cm}\textbf{Problems:}
    \vspace{-0.5cm}
    \begin{itemize}
        \item Troubleshooting AcquiSuite still
    \end{itemize}  
    \vspace{-0.3cm}\noindent\textbf{Summary:}\\
    \noindent User permission levels implemented. 
\end{adjustwidth} 
\paragraph{Week 5}
\begin{adjustwidth}{2.5em}{0pt}
    \vspace{-0.5cm}\textbf{Plans:}
    \vspace{-0.5cm}
    \begin{itemize}
        \item Debug above functionality and ensure it works before expo.
    \end{itemize} 
    \vspace{-0.3cm}\textbf{Progress:}
    \vspace{-0.5cm}
    \begin{itemize}
        \item This week I mainly focused on updating the schema and functionality for login. Specifically this was for accounts invited through the email invitation link. 
		\item After a user was invited, nothing really happened besides them getting an email. After this update, users will be added to the database. 
		\item Added minor fix to form input blacklist. There was also a lot of work done for midterm presentation.  

    \end{itemize} 
    \vspace{-0.3cm}\textbf{Problems:}
    \vspace{-0.5cm}
    \begin{itemize}
        \item None
    \end{itemize}  
    \vspace{-0.3cm}\noindent\textbf{Summary:}\\
    \noindent 
	Good functionality updates and worked on progress report.
\end{adjustwidth} 
\paragraph{Week 6}
\begin{adjustwidth}{2.5em}{0pt}
    \vspace{-0.5cm}\textbf{Plans:}
    \vspace{-0.5cm}
    \begin{itemize}
        \item Debug login functionality further. 
		\item Prepare for expo.
        \item Prepare for code freeze.
    \end{itemize} 
    \vspace{-0.3cm}\textbf{Progress:}
    \vspace{-0.5cm}
    \begin{itemize}
        \item This week I mainly focused on figuring out syntax for JSDocs. This is going to be our main code documentation, so I really wanted to focus on learning the ins and outs of it. 
		\item I also added JSDocs for basic typedefs. 
		\item This was for both regular JS objects and custom created objects we use in our codebase. 
    \end{itemize} 
    \vspace{-0.3cm}\textbf{Problems:}
    \vspace{-0.5cm}
    \begin{itemize}
        \item JSDocs seems a little finicky at first, but after some tweaking, I can see it being a powerful documentation asset.
    \end{itemize}  
    \vspace{-0.3cm}\noindent\textbf{Summary:}\\
    \noindent Documentation for final started. 
\end{adjustwidth} 
\paragraph{Week 7}
\begin{adjustwidth}{2.5em}{0pt}
    \vspace{-0.5cm}\textbf{Plans:}
    \vspace{-0.5cm}
    \begin{itemize}
        \item Continue on documentation. 
    \end{itemize} 
    \vspace{-0.3cm}\textbf{Progress:}
    \vspace{-0.5cm}
    \begin{itemize}
        \item This week I mainly focused on cleaning up our site so it would look better for expo. 
		\item The most important update I implemented was sitewide security. I found this to be a huge priority because we would be demoing our project on our live website. If someone were to wander to that site on their own device, they could wreak havoc on our databases.
    \end{itemize} 
    \vspace{-0.3cm}\textbf{Problems:}
    \vspace{-0.5cm}
    \begin{itemize}
        \item None. 
    \end{itemize}  
    \vspace{-0.3cm}\noindent\textbf{Summary:}\\
    \noindent Expo was great! 
\end{adjustwidth} 
\paragraph{Week 8}
\begin{adjustwidth}{2.5em}{0pt}
    \vspace{-0.5cm}\textbf{Plans:}
    \vspace{-0.5cm}
    \begin{itemize}
        \item Keep working on documentation 
    \end{itemize} 
    \vspace{-0.3cm}\textbf{Progress:}
    \vspace{-0.5cm}
    \begin{itemize}
        \item This week I mainly focused on working on documentation. I will have a large amount of code to push in next couple of weeks.
    \end{itemize} 
    \vspace{-0.3cm}\textbf{Problems:}
    \vspace{-0.5cm}
    \begin{itemize}
        \item None.
    \end{itemize}  
    \vspace{-0.3cm}\noindent\textbf{Summary:}\\
    \noindent Documenting everything.  
\end{adjustwidth} 
\paragraph{Week 9}
\begin{adjustwidth}{2.5em}{0pt}
    \vspace{-0.5cm}\textbf{Plans:}
    \vspace{-0.5cm}
    \begin{itemize}
        \item Finish presentation. 
    \end{itemize} 
    \vspace{-0.3cm}\textbf{Progress:}
    \vspace{-0.5cm}
    \begin{itemize}
        \item This week I mainly focused on getting my blog posts up to speed with progress. I also started on some of the final presentation and report stuff.
    \end{itemize} 
    \vspace{-0.3cm}\textbf{Problems:}
    \vspace{-0.5cm}
    \begin{itemize}
        \item None
    \end{itemize}  
    \vspace{-0.3cm}\noindent\textbf{Summary:}\\
    \noindent Documenting everything. 
\end{adjustwidth} 
\paragraph{Week 10}
\begin{adjustwidth}{2.5em}{0pt}
    \vspace{-0.5cm}\textbf{Plans:}
    \vspace{-0.5cm}
    \begin{itemize}
        \item Gather things together and finish documentation. 
    \end{itemize} 
    \vspace{-0.3cm}\textbf{Progress:}
    \vspace{-0.5cm}
    \begin{itemize}
        \item This week I mainly focused on more code documentation. I also wanted to start getting all of my documents together for the final report. 
    \end{itemize} 
    \vspace{-0.3cm}\textbf{Problems:}
    \vspace{-0.5cm}
    \begin{itemize}
        \item None
    \end{itemize}  
    \vspace{-0.3cm}\noindent\textbf{Summary:}\\
    \noindent Finishing everything up.
\end{adjustwidth} 