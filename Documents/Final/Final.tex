\documentclass[journal,10pt,onecolumn,compsoc]{IEEEtran} 
\usepackage[margin=0.75in]{geometry} 
\usepackage{pdfpages} 
\usepackage{graphicx}  

\graphicspath{/images} 
\usepackage[english]{babel}
\usepackage{blindtext}

\makeatletter
\renewcommand{\paragraph}{\@startsection{paragraph}{4}{0ex}%
   {-3.25ex plus -1ex minus -0.2ex}%
   {1.5ex plus 0.2ex}%
   {\normalfont\normalsize\itshape}}
\makeatother
\stepcounter{secnumdepth}
\stepcounter{tocdepth}

\usepackage{tikz}
\usetikzlibrary{shapes,arrows}
\tikzstyle{block} = [rectangle, draw, fill=blue!20, 
text width=5.5em, text centered, rounded corners, minimum height=4em]
\tikzstyle{subblock} = [rectangle, draw, fill=blue!20, 
text width=7em, text centered, rounded corners, minimum height=2em]
\tikzstyle{datablock} = [rectangle, draw, fill=red!20, 
text width=7em, text centered, rounded corners, minimum height=2em]
\tikzstyle{line} = [draw, -latex']  

\newenvironment{paddedtikzpicture}{\vspace*{0.5cm}
\begin{center}\begin{tikzpicture}}
{\
\end{tikzpicture}\end{center}
}  
\setlength{\parskip}{\baselineskip} \setlength\parindent{24pt}
\usepackage{url}
\usepackage{setspace}
\usepackage{geometry}
\geometry{textheight=9.5in, textwidth=7in}
\usepackage{hyperref}
\usepackage{caption} 
\usepackage{float} 
\usepackage{rotating} 
\usepackage{ragged2e} % provides \RaggedLeft
\hypersetup{
    colorlinks,
    citecolor=black,
    filecolor=black,
    linkcolor=black,
    urlcolor=black
}
\geometry{textheight=9.5in, textwidth=7in}
\usepackage{listings}
\usepackage{color}
\usepackage{cite}
\lstdefinelanguage{JavaScript}{
    keywords={typeof, new, true, false, catch, function, return, null, catch, switch, var, if, in, while, do, else, case, break},
    keywordstyle=\color{blue}\bfseries,
    ndkeywords={class, export, boolean, throw, implements, import, this},
    ndkeywordstyle=\color{darkgray}\bfseries,
    identifierstyle=\color{black},
    sensitive=false,
    comment=[l]{//},
    morecomment=[s]{/*}{*/},
    commentstyle=\color{purple}\ttfamily,
    stringstyle=\color{red}\ttfamily,
    morestring=[b]',
    morestring=[b]"
  }
  \lstset{
    language=JavaScript,
    backgroundcolor=\color{lightgray},
    extendedchars=true,
    basicstyle=\footnotesize\ttfamily,
    linewidth=7in,
    xleftmargin=.25in,
    showstringspaces=false,
    showspaces=false,
    numbers=left,
    numberstyle=\footnotesize,
    numbersep=9pt,
    tabsize=2,
    breaklines=true,
    showtabs=false,
    captionpos=b
 }

% 1. Fill in these details
\def \CapstoneTeamName{		The Dream Team}
\def \CapstoneTeamNumber{		57}
\def \GroupMemberOne{			Daniel Schroeder}
\def \GroupMemberTwo{			Aubrey Thenell}
\def \GroupMemberThree{			Parker Bruni}
\def \CapstoneProjectName{		A Scalable Web Application Framework for Monitoring Energy Usage on Campus  }
\def \CapstoneSponsorCompany{	Oregon State Office of Sustainability}
\def \CapstoneSponsorPerson{		Jack Woods}

% 2. Uncomment the appropriate line below so that the document type works
\def \DocType{		%Problem Statement
        %Requirements Document
        %Technology Review
        Design Document
        %Progress Report
        }
        
\newcommand{\NameSigPair}[1]{\par
\makebox[2.75in][r]{#1} \hfil 	\makebox[3.25in]{\makebox[2.25in]{\hrulefill} \hfill		\makebox[.75in]{\hrulefill}}
\par\vspace{-12pt} \textit{\tiny\noindent
\makebox[2.75in]{} \hfil		\makebox[3.25in]{\makebox[2.25in][r]{Signature} \hfill	\makebox[.75in][r]{Date}}}}
% 3. If the document is not to be signed, uncomment the RENEWcommand below
%\renewcommand{\NameSigPair}[1]{#1}

%%%%%%%%%%%%%%%%%%%%%%%%%%%%%%%%%%%%%%%
\title{Design Document for: \linebreak Scalable Web Application Framework for Monitoring Energy Usage on Campus}
\author{Daniel Schroeder, Aubrey Thenell, Parker Bruni}
\date{\today}

\begin{document}
\maketitle
\vspace{2cm}
\begin{center}
\noindent \textbf{Abstract} \\
            \indent The purpose of this document is to outline the architecture and component design for a Scalable Web Application Framework for Monitoring Energy Usage on Campus. The designs for each modular component are discussed in detail, outlining the data residing in each object and the functionality it holds in the application. Additionally, this document aims to discuss the rationale for each design decision and the interactions between individual components. 
\end{center}         

\newpage
\pagenumbering{arabic}
\tableofcontents
    
% 8. now you write! 
\section{Introduction to Project} 
(Hint, refer to project proposal and problem statement)
\subsection{Who requested it?}
\subsection{Why was it requested?}
\subsection{What is its importance?}
\subsection{Who was/were your client(s)?}
\subsection{Who are the members of your team?}
\subsection{What were their roles?}
\subsection{What was the role of the client(s)?} (I.e., did they supervise only, or did they participate in doing development)
\section{Requirements Document}
%%%%%%% ---------------------------------
% Insert the requirements document
%%%%%%% ---------------------------------
\subsection{Introduction}

    \subsubsection{Purpose}
    The purpose of this document is to outline the project and all associated information. Furthermore, it will outline how the end product will be used, how it was developed, and all resources and documentation referenced and used during development. This document will also describe the application’s target audience, user interface, and any hardware/software requirements. Finally, it will define how our client, team, and users will see the final product and related functionality.
    \subsubsection{Scope}
    This project will provide a web application that displays different collections of energy use reports for campus buildings. Each of these collections will be customizable, with the ability to add, remove, and filter building data according to user preference. The web application will allow users to visually comprehend energy data in a way that can raise awareness about consumption and reduce the campus-wide carbon footprint. The web application will also have three different permission levels: Guest, Registered Users, and Admin Users. Each one will provide permissions to what can be viewed and edited.

    \subsubsection{Definitions, acronyms, and abbreviations} \label{definition}
    \begin{table}[h]
    \centering
    \begin{tabular}{ll}
    \textbf{Term} & \textbf{Definition} \\
    OSU & Oregon State University \\
    MEAN stack & MongoDB, Express, AngularJS, Node.js \\
    JSON & JavaScript Object Notation \\
    AcquiSuite & Data Acquisition Servers responsible for metering energy usage \\
    Collections & User Interface dashboard that displays assortment of blocks \\
    Blocks & Charts/graphs of building data \\
    Stories & Collections of dashboards 
    \end{tabular}
    \end{table}
    \subsubsection{References} 
    The references are:
    \begin{itemize}
        \item AcquiSuite and Modbus Information \cite{aquisuite_emb} \cite{modbus_practices} \cite{modbus} \cite{AcquiSuite}
        \item D3.js \cite{d3_home}
        \item Vis.js \cite{vis}
        \item The M.E.A.N. Stack \cite{mean}
        \item JSON \cite{json}
        \item BuildingOS \cite{BuildingOS}
        \item Mocha.js \cite{mochajs}
    \end{itemize}
    
    \normalsize
    \subsubsection{Document Overview}
	The remaining sections of this document provide a more elaborate description of product functionality, assumptions, and specific requirements. Section 2 provides information about functional requirements, data requirements, and assumptions made for designing the \textit{Scalable Web Application Framework for Monitoring Energy Usage on Campus}. Section 3 outlines the specific requirements for the final product, the external interfaces that communicate with the software, and the functional requirements of the system.
	
    \subsection{Overall description}
    This section will provide an overview of the web application as a whole, including:
        \begin{enumerate} 
            \setlength\itemsep{1mm}
            \item Details about the user interface and expected user interactions.
            \item An outline of specific constraints and dependences included in development. 
            \item Background information about the specific software requirements.
        \end{enumerate}
    \subsubsection{Product perspective}
    %A block diagram showing the major components of the larger system, interconnections, and external interfaces can be helpful.
    %5.2.1.1 System interfaces
    %This should list each system interface and identify the functionality of the software to accomplish the system
    %requirement and the interface description to match the system.

    This product will replace the current web application that is used by The Oregon State Sustainability Office. The product will display energy usage information about Oregon State University buildings through an intuitive user interface. The application will gather data from energy meters by connecting to AcquiSuite\texttrademark data acquisition servers. The energy data from the meters will be transferred to a database that the product will access directly. Users will interact with the application interface from an internet browser application. 
    
    \subsubsection{Product functions}
    The product will allow users to create accounts that may be given either administrative or user permissions. Users have the ability to personalize their accounts. Administrators accounts will have special capabilities that user accounts will not. \\
    The application will allow users to create customizable dashboards that will contain easily adjustable “blocks” of campus building data that will contain the campus building energy usage data. These “blocks” of data will be the basic building blocks for the dashboard and will provide an intuitive view of the data. They will feature various graph types, building energy efficiency rankings, and data trends.\\
    Each OSU building that contains the energy monitoring meter(s) will have a specific, non-customizable page that will display general information. The product will also have a public interface and a private administrative interface.\\
    Administrators of the application will have the ability to add, remove, or edit entire buildings profiles, building subspaces, or individual meters.
    
    \subsubsection{User characteristics}
    A user that will be using the general public UI will not need to know any specific information about the application to navigate the various energy data presentations. A public user will be able to intuitively navigate the UI at their discretion. \\ 
    An administrative level user will need a basic understanding of the tools of the application because they will be allowed the freedom to control parts of the website as well as have access to more specific energy data within the application. An administrator will likely not need extensive training to use this application for more specific purposes as the administrator UI will be designed to be intuitive to navigate.
    \subsubsection{Constraints}
    Data updates will be limited to a granularity of 15 minute intervals. The data acquisition server is capable of providing a granularity of up to 15 second intervals but it is not necessary for the purposes of the application. Any other constraints on the product are subject to the internet browsing interface that is accessing our product. The UI will be intuitive and able to navigate data presentations, but will not allow manipulation of the data or underlying structure of the application.
    
    \subsubsection{Assumptions and Dependencies}
    The application will be dependant on the data acquisition server. If the data acquisition server is drastically changed or removed, the application will not be functional. Meter data within the application will also be limited by the functionality of the meters themselves. Should a meter malfunction, the energy data will not be gathered which may cause some of the data presentations to deviate from expected data. The use of the application will require a compatible web browsing interface, which may be limited to browsers such as Firefox, Chrome, and Internet Explorer. The functionality of the product will depend on those internet browsers performing as expected with regards to their intended functionality and internet access. 

    \subsubsection{Apportioning of Requirements}
    Future versions of the application may include features such as cost tables, automated electronic invoice generation, energy billing analysis capabilities, budget analysis capabilities, and mobile energy data entry.
    
    \subsection{Specific Requirements}
    
    \subsubsection{External Interfaces}
    AcquiSuite data acquisition servers [meters] made by Obvious
    \begin{itemize}
        \setlength\itemsep{1mm}
        \item Used for collecting electric, water, gas, steam, and other energy parameters over the web.
        \item Data is received through IP-based connection.
        \item Data can be reached anywhere with an internet connection, as long as the AcquiSuite is online.
        \item Data will be collected every 15 minutes.
        \item Data from the AcquiSuite servers will be stored into a database.
    \end{itemize}

    \subsubsection{Functions}
    This section defines how the software system should behave with regards to input and output.
    \begin{itemize}
        \setlength\itemsep{1mm}
        \item The system shall receive and store data from AcquiSuite data acquisition servers into the database. 
        \item The system shall have input validation measures in place to monitor incoming data and protect from malicious injections.
        \item The system shall retrieve data from the database and populate webpages with filtered datasets when prompted.
        \item The system shall have permission based restrictions for accessing certain data.
        \item The system shall generate alerts for offline buildings and high energy usage.
        \item The system shall generate emails for users to access a sign-up form and create an account.
        \item The system shall be able to create arbitrary combinations of datasets given user filters.
        \item The system shall be able to calculate rankings for all the buildings based on certain metrics like ``kilowatt-hour consumption''.
    \end{itemize}

    \subsubsection{Performance Requirements}
    This section describes the functionality requirements for the software as well as what a user should be able to accomplish when using the web application.

    \paragraph{Users}
    The system will have 3 types of users: admin level users, registered users, and generic users. An administrative user will be able to add new meters, buildings, and other objects into the database. A registered user, for example the head of a department, can create their own stories with information and collections that are unique to their own interests or departments. Lastly, a generic user can simply view the publically facing web application and see public dashboards and browse public stories. 
    \begin{itemize}
        \setlength\itemsep{1mm}
        \item A registered user should be able to customize dashboard layouts in a grid-based orientation. A \hyperref[definition]{\textit{story}} page should have a customizable layout where a user can add different \hyperref[definition]{\textit{blocks}} with information relevant to their personal needs.
        \item An administrative user should be able to add buildings to the database through a web form.
        \item An administrative user should be able to add data aquisitions servers.
        \item An administrative user should be able to download specific datasets as a .csv file.
        \item An administrative user should be able to customize public stories.
    \end{itemize}

    \paragraph{Data Management}
    \begin{itemize}
        \setlength\itemsep{1mm}
        \item The web application should update data blocks every 15 minutes as new data is received into the database.
        \item The web application should allow users to create  which are collections of dashboards. These 
        \hyperref[definition]{\textit{stories}} are meant to bring related buildings and datasets together into intuitive groups, for instance ``Residence Halls'' or ``Engineering Buildings.''
        \item The application should be able to filter building data by date range specifications.
        \item The web application should be able to scale up to as many buildings are on campus.
        \item The web application should properly create and store new entities (buildings, users, meters) into the database.
        
    \end{itemize}    

    \paragraph{Visualization}
    \begin{itemize}
        \setlength\itemsep{1mm}
        \item The web application should be able to generate different types of graphs for different \hyperref[definition]{\textit{blocks}}. For example load-profile charts, comparative line charts, and heat maps.
        \item The web application should have generic pages for each building in the database. This page will display a series of graphs and charts that outline energy usage for a particular building.
        \item The \hyperref[definition]{\textit{blocks}} on each page should automatically update as new data is received by the database.
    \end{itemize}
    \paragraph{Objects}
        The system should have three main entities: 
        \begin{itemize}
            \setlength\itemsep{1mm}
            \item \textbf{Building} - A building on campus that is connected to a data acquisition server.
            \item \textbf{Meter} - A specific data acquisition server.
            \item \textbf{User} - A user with a specific role.
        \end{itemize}
        
    \subsubsection{Logical Database Requirements}
    Most of the calls to the database will be requests for meter data from a specific building or group of buildings. Access to the database should be hidden from the user and only accessed from the back-end of the application itself. Any input into the database should be validated and encrypted, if applicable. 
    
    \subsubsection{Design Constraints}
    Design constraints may include server availability which could harm scalability.
    \paragraph{Standards Compliance}
    There may be standards when storing energy data based on The Office of Sustainability Standards.
    [Check with Client]

    \subsubsection{Software System Attributes}
    
    \paragraph{Reliability}
    The system will be reliable at the time of deployment if all data displayed in graphs and charts is correct.
    \paragraph{Availability}
    The system should be consistently available as long as the servers are up and running. As a web application, the system will be available via URL.
    \paragraph{Security}
    The system will validate input for duplicate and malicious content.
    The system will also have back-end functionality to protect user passwords through proper hashing.
    The web application will restrict access to certain pages and content based on user roles and permissions.

    \paragraph{Maintainability}
    Maintainability for this application should be simple for anyone with web development experience. The application code will be organized into logical directories and sub directories that should mimic a full stack web application. There will be documentation that explains how to connect to the AcquiSuite servers and how to use our API for data collection.
    \paragraph{Portability}
    Our web application should be visible and accessible from most web browsers. There are not any host-dependent constraints since it will be hosted from a central server and available to the public via the internet. The source code should stay on Github so modifications can be made easily.
    


include the original document, showing what you thought, at the time, was the project definition with the original Gantt chart
Add (your client should have okay'd): What new requirements were added? What existing requirements were changed? What existing requirements were deleted? Why? 
Add: Final Gantt Chart as a record of what happened when.

\section{Design Document (original)}
\subsection{Introduction}
\subsubsection{Purpose}

The purpose of this Software Design Document (SDD) is to provide details about the architecture of the web application as well
as details about each component. It will describe the underlying design of each component and their purpose within the context
of the application that will allow Oregon State affiliates to effectively monitor energy usage on campus.

\subsubsection{Scope}

The software outlined in this document will function as an interface by which Oregon State students, faculty, and affiliates
may monitor energy usage of various buildings located on the Oregon State Campus. It will act as a tool for users to 
make informed infrastructure decisions and adjustments or act as a display piece to be presented within buildings on campus.
It will serve the OSU Office of Sustainability as a replacement to an outdated and costly implementation. 

\subsubsection{Overview}
Oregon State University is constantly making strides to reduce its carbon footprint and reduce its energy consumption. There is a carbon neutrality goal for 2025 where the university is trying to generate as much energy as it uses and have a net carbon footprint of zero. Our web application aims to monitor the energy use of buildings on campus in order to create a visual representation of each building's consumption and incentivize people to monitor their consumption habits and reduce the campus's overall consumption.

\noindent Our web application will resemble an administrative dashboard with charts and graphs of energy use over time for individual buildings and subgroups of buildings on Oregon State's campus. 
Our application will have a series of public facing pages that will show energy usage for all the buildings on campus that have AcquiSuite meters as well as display user generated stories of personalized dashboards. Each page will be a grid-based dashboard with personalized blocks for displaying data through time-series charts or graphs. A user will be able to add different blocks to create unique dashboards for their own unique interests and subsets of buildings. An example of this would be a dashboard designated to only residence halls where each block shows a usage over time graph for energy consumption of each residence hall. Users will be able to apply date filters to their blocks to generate different data sets and more explicit visualizations.

\noindent Our web application will be constructed using a MEAN Stack framework hosted on AWS. Our application will have a MongoDB database server and a Node.js application server hosted on a single virtual EC2 instance to make deployment easy and reliable.

\subsubsection{Reference Material} 
    \begin{itemize}
        \item Bootstrap4 Dashboards \cite{dashboard_bootstrap} 
        \item Using OAuth 2.0 to Access Google APIs \cite{oauth} 
        \item Data-Driven Documents \cite{data_driven}
        \item AngularJS \cite{ng}
        \item AcquiSuite by Obvius \cite{AcquiSuite}
    \end{itemize} 
\subsubsection{Definitions and Acronyms}
This subsection will define terms and acronyms that are specific to the application that may otherwise be misunderstood or poorly interpreted.
\begin{itemize}
    \item \textbf{Dashboard:}
        A grid-based information management tool for visually tracking and displaying metrics and data through graphs and charts.
    \item \textbf{Block:} 
        An individual graph or chart depicting time-series data. More specifically, it is an HTML element containing one or more buildings, with filter buttons, informative text, and graphs.
    \item \textbf{Story:} 
        A user generated collection of dashboards. What each story displays is based solely on user preferences. With multiple stories, the user can select between ``presets'' that can show different dashboards.
    \item \textbf{MEAN Stack:} 
        An acronym used to define a full stack application engineered from the MongoDB, Express.js, Angular.js, and Node.js frameworks. 
    \item \textbf{Bootstrap:} 
        Bootstrap CSS is a front-end framework that uses component templates to easily generate different HTML elements like buttons, navigation, or forms. 
    \item \textbf{OAuth 2.0:} 
        OAuth 2.0 is an authorization protocol that grants authentication through tokens rather than credentials. 
    \item \textbf{Filter:} 
        Filters are essentially parameters for fetching data from that constrain the subset of data being received to the specifications of the filter parameters (i.e. a date range). 
    \item \textbf{D3:} 
        D3.js is a visualization framework that appends charts and graphs to DOM elements on a webpage. 
    \item \textbf{AcquiSuite:} 
        AcquiSuites are data acquisition servers made by a company called Obvius that post building meter data to a designated IP address.
    \item \textbf{AWS:} 
        AWS is an acronym for Amazon Web Services which offers reliable cloud computing services for building and hosting web applications.
   \item \textbf{Passport:} 
        Passport is authentication middleware for Node.js. Extremely flexible and modular, Passport can be unobtrusively dropped in to any Express-based web application. A comprehensive set of strategies support authentication using a username and password, Facebook, Twitter, etc.
\end{itemize}
\subsection{System Overview} 
Our web application will provide users with the functionality to create unique dashboard-style collections of charts and graphs to analyze energy use from Oregon State's campus buildings. Our system will be highly modular, containing multiple smaller components that provide their own individual functionality while sharing information across other components. Designing a modular architecture will simplify the code base and allow the developers to encapsulate specific data models as their own entities and keep abstractions in the system differentiable.

\noindent Our design also stems from the main constructs behind AngularJS and data injection. Most of our components and subsystems are, at large, data models and objects that hold all the necessary information to create graph and charts. With this design, AngularJS services will extract data from the session user and rendered components and generate render the desired graphs and views to the content container. This allows the application to break up back-end services as well as UI components to simplify how data is gathered and processed.
\subsection{Design Viewpoints and Concerns}
\subsubsection{Stakeholders and Concerns}
Developers concerns consist of simplifying complex model functions, replicable components, modular subsystems, and simple data access functionality. Addressing and capitalizing on these concerns will create an application design that will enhanced understanding, simplify implementation, and produce a high-end product. From a developer's standpoint, it is also essential to design an application that is ideal for both implementation and usability.

\noindent Client concerns revolve around satisfying the requirements of the application and providing all the necessary user functionality. The implementation should adhere to the client's requests and specifications while creating a user-friendly interface to perform all the necessary actions.
\subsubsection{Design Viewpoints}
\paragraph{Context Viewpoint}
The design of this web application will greatly depend upon satisfying the user actions outlined in the requirements document. With that in mind, all components and subsystems should cater towards simplifying the user's experience while performing actions and navigating through the application. This viewpoint provides design constraints for individual component functionality, as well as the overall layout and structure of the application as a whole. Our design implementations should cater to a wide range of users while creating an elegant way for the user to create, edit, and remove content.
\paragraph{Composition Viewpoint}
This application contains a number of components that are recurrently used throughout the application and across users. These include the basic data blocks, dashboards, and stories that a user can create. The design of this application should embrace the modularity of these components and provide a consistent model for individual functionality while maintaining the ability to communicate and pass data across different components or objects. 

\paragraph{Dependency Viewpoint}
With a design focused around modularity and individualized components, there needs to be designs for services and dependencies so each component can interact. Some examples of these are: how a block component depends on multiple entities contained inside the block object like the building objects stored within the block, the filter's current values, the graph template, and the graphing service. The design of this application should encapsulate the relationships between different modular components and provide intuitive means for inter-component communication.

\paragraph{Information Viewpoint}
Our applications design should provide logical ways of storing data and receiving data from building meters. This involves an explicit declaration of individual data models and how how data is created and stored. Our design will attempt to create a modular data structure where data specific to certain entities reside in the entity objects themselves. This allows the application to access data from an individual component as it gets created or moved throughout the application.

\paragraph{Logical Viewpoint}  
The application should maintain a consistent model for each component which can be used in different ways for different purposes. An entity should contain and provide the necessary information as it is needed throughout various areas of the application. The design of the application as a whole should consider where certain data is needed, and what data models must be present to satisfy the system's requests.
\begin{figure}[H]
    \centering
    \includegraphics[width=14cm]{images/CapstoneUML.eps}
    \caption{A UML diagram to show component relationships.}
\end{figure}
\paragraph{Interface Viewpoint}  
Users will require a front end interface to interact with the application and utilize its functionality. The design of these interfaces should focus on satisfying all user actions in a simple and appealing manner. Users should be able to intuitively navigate the application and perform essential tasks without confusion or error.
\paragraph{Structure Viewpoint} 
The design of this application describes the structure of independent modular components and their relationships to other entities. From a developer's standpoint, organizing the structure and interdependence of different components is essential when implementing a complex system. This application's design highlights key individual components along with their individual functionality and relationships with other entities. 
\paragraph{Interaction Viewpoint}  
Each object component in the application should not need any logic to handle its interactions with other components. This functionality should be performed by services and model functions implemented in the back-end of the application. The design of this application aims to keep data and entities separate from the logical functionality required to extract and share data across different components.
\subsection{System Architecture}
\subsubsection{Architectural Design}

\begin{figure}[H] 
\begin{paddedtikzpicture}[node distance = 2cm, auto]
    % Place nodes
    \node (ui)[block] {User Interface}; 
    \path (ui.east)+(2,5) node (nav) [block] {Navigation}; 
    \path [line] (ui)+(-0.3,0.7) -- (nav);
    \path (ui.east)+(2,3) node (block) [block] {Create Block}; 
    \path [line] (ui) -- (block);
    \path (ui.east)+(2,1) node (dash) [block] {Create Dashboard}; 
    \path [line] (ui) -- (dash);
    \path (ui.east)+(2,-1) node (story) [block] {Create Story}; 
    \path [line] (ui) -- (story);
    \path (ui.east)+(2.35,-3) node (add) [block,  text width=7.5em] {Add Building/Meter};
    \path [line] (ui) -- (add);
    \path (ui.east)+(2,-5) node (log) [block] {Log In}; 
    \path [line] (ui)+(-0.3,-0.7) -- (log);
    
    %Create Dashboard
    \path (nav.east)+(2,0) node (select) [subblock] {Get User Selection};
    \path [line] (nav) -- (select);
    \path (select.east)+(2,0) node (route)[datablock] {Pass Selected Route to View};
    \path [line] (select) -- (route); 
     
    %Create Block
    \path (block.east)+(2,0.5) node (ctype) [subblock] {Get Chart Type};
    \path [line] (block) -- (ctype);
    \path (block.east)+(2,-0.5) node (bdata)[subblock] {Get Building Data};
    \path [line] (block) -- (bdata); 
    \path (block.east)+(5,0) node (template)[subblock] {Retrieve Graph Template}; 
    \path [line] (bdata) -- (template); 
    \path [line] (ctype) -- (template); 
    \path (template.east)+(2,0) node (d3)[subblock] {Build D3}; 
    \path [line] (template) -- (d3); 
    %Create Dashboard
    \path (dash.east)+(2,0) node (getblocks) [datablock] {Get Blocks};
    \path [line] (dash) -- (getblocks);
    \path (getblocks.east)+(2,0) node (saveblock)[datablock] {Add Blocks to Dashboards};
    \path [line] (getblocks) -- (saveblock); 
    %Create Story
    \path (story.east)+(2,0) node (createdash)[datablock] {Get Dashboards};
    \path [line] (story) -- (createdash); 
    \path (createdash.east)+(2,0) node (adddash)[datablock] {Add Dashboards to Story};
    \path [line] (createdash) -- (adddash); 
    
    %Add Building
    \path (add.east)+(2,0) node (input)[subblock] {Get Input Data};
    \path [line] (add) -- (input); 
    \path (input.east)+(2,0) node (storeinput)[subblock] {Store Input Data};
    \path [line] (input) -- (storeinput); 

    %Log In
    \path (log.east)+(2,0) node (auth) [subblock]  {Google oAuth 2.0};
    \path [line] (log) -- (auth);
    \path (auth.east)+(2,0.5) node (store) [subblock] {Store User};
    \path [line] (auth) -- (store);
    \path (auth.east)+(2,-0.5) node (retrieve)[subblock] {Retrieve User}; 
    \path [line] (auth) -- (retrieve);
\end{paddedtikzpicture}
\caption{An overview of the major application systems.} 
\end{figure}
\subsubsection{Decomposition Description}

All database operations (retrieving data and storing data) are denoted with red fill and will be carried out by an AngularJS ``Service'' within the component's controller. 
\paragraph{Log In}
\begin{figure}[H] 
\begin{paddedtikzpicture}[node distance = 2cm, auto]
    % Place nodes
    \node (login)[block] {Log In}; 
    \path (login.east)+(2,0) node (googleauth) [block] {Contact Google oAuth API}; 
    \path [line] (login) -- (googleauth);
    \path (googleauth.east)+(2,0.5) node (store) [datablock] {Store User in Database}; 
    \path [line] (googleauth) -- (store);
    \path (googleauth.east)+(2,-0.5) node (retrieve) [datablock] {Retrieve User From Database}; 
    \path [line] (googleauth) -- (retrieve);
\end{paddedtikzpicture} 
\caption{A functional representation of the log in subsystem.} 
\end{figure}
\paragraph{Navigation} 
\begin{figure}[H] 
\begin{paddedtikzpicture}[node distance = 2cm, auto]
    % Place nodes
    \node (nav)[block] {Navigation Bar}; 
    \path (nav.east)+(2,0) node (route) [block] {Get Route from Nav Item}; 
    \path [line] (nav) -- (route);
    \path (route.east)+(2,0.5) node (pass) [subblock] {Pass Route to Parent Scope}; 
    \path [line] (route) -- (pass);
    \path (route.east)+(2,-0.5) node (view) [datablock] {Call Route's Get Function}; 
    \path [line] (route) -- (view);
    \path (route.east)+(5.5,0) node (render) [subblock, text width=8em] {Render New View HTML}; 
    \path [line] (pass) -- (render);
    \path [line] (view) -- (render);
\end{paddedtikzpicture} 
\caption{A functional representation of the navigation subsystem.} 
\end{figure}
\paragraph{Create Block} 
\begin{figure}[H] 
\begin{paddedtikzpicture}[node distance = 2cm]
    % Place nodes
    \node (create)[block] {Create Block}; 
    \path (create.east)+(2,0) node (view) [block] {Serve Web Form in View};
    \path [line] (create) -- (view);
    \path (view.east)+(2,0.5) node (charts) [datablock] {Pull Charts from Database}; 
    \path [line] (view) -- (charts);
    \path (view.east)+(2,-0.5) node (buildings) [datablock] {Pull Buildings from Database}; 
    \path [line] (view) -- (buildings);
    \path (view.east)+(5.5,0) node (submit) [subblock, text width=8em] {Get User from Data}; 
    \path [line] (charts) -- (submit);
    \path [line] (buildings) -- (submit);
    \path (submit.east)+(2,0) node (store) [datablock, text width=8em] {Store Block in User Object}; 
    \path [line] (submit) -- (store);
\end{paddedtikzpicture} 
\caption{A functional representation of the block creation subsystem.} 
\end{figure}
\paragraph{View Blocks}
\begin{figure}[H] 
\begin{paddedtikzpicture}[node distance = 2cm]
    % Place nodes
    \node (view)[block] {View Blocks}; 
    \path (view.east)+(2,0) node (retrieve) [datablock] {Retrieve Blocks from User Object};
    \path [line] (view) -- (retrieve);
    \path (retrieve.east)+(2,0) node (render) [subblock] {Render Blocks to Content View}; 
    \path [line] (retrieve) -- (render);
    \path (render.east)+(2,0.5) node (edit) [datablock, text width=8em] {Edit Block Fields}; 
    \path (render.east)+(2,-0.5) node (delete) [datablock, text width=8em] {Delete Block};
    \path [line] (render) -- (edit);
    \path [line] (render) -- (delete); 
\end{paddedtikzpicture}  
\caption{A functional representation of the block view subsystem.} 
\end{figure}
User created blocks will be stored in their respective ``Blocks'' array This allows a user to keep track of the specific blocks they create and reuse them to create stories.
\paragraph{Create Dashboard}
\begin{figure}[H] 
\begin{paddedtikzpicture}[node distance = 2cm]
    % Place nodes
    \node (view)[block] {Create Dashboard}; 
    \path (view.east)+(2,0) node (retrieve) [datablock] {Retrieve Blocks from User Object};
    \path [line] (view) -- (retrieve);
    \path (retrieve.east)+(2,0) node (render) [datablock] {Store Selected Blocks in Dashboard Object}; 
    \path [line] (retrieve) -- (render);
    \path (render.east)+(2,0) node (edit) [datablock, text width=8em] {Store Dashboard in User Object};  
    \path [line] (render) -- (edit); 
\end{paddedtikzpicture}  
\caption{A functional representation of creating a dashboard object from user created block.} 
\end{figure}
A dashboard is a collection of blocks. The dashboard component provides a way to group blocks together based on user preference and display them in the content view by simply rendering each block stored.
\paragraph{Create Story}
\begin{figure}[H] 
\begin{paddedtikzpicture}[node distance = 2cm]
    % Place nodes
    \node (story)[block] {Create Story}; 
    \path (story.east)+(2,0) node (retrieve) [datablock] {Retrieve User Dashboards};
    \path [line] (story) -- (retrieve);
    \path (retrieve.east)+(2,0) node (add) [datablock, text width=8em] {Save Selected Dashboards to Story};
    \path [line] (retrieve) -- (add);
    \path (add.east)+(2,0) node (store) [datablock, text width=8em] {Store Block in Story Object};
    \path [line] (add) -- (store); 
\end{paddedtikzpicture}   
\caption{A functional representation of the story creation subsystem.} 
\end{figure}
Stories are stored in the User object who created them in an array ``Stories.'' This allows the application to render stories created by the session user easily and will eliminate the use of SQL style querying to retrieve stories based on a user key.
\paragraph{Add Building/Meter}
\begin{figure}[H] 
    \begin{paddedtikzpicture}[node distance = 2cm]
        % Place nodes
        \node (add)[block, text width=8em] {Add Building/Meter}; 
        \path (add.east)+(2,0) node (render) [subblock] {Render Input Form};
        \path [line] (add) -- (render);
        \path (render.east)+(2,0) node (store) [datablock, text width=8em] {Store Form Values As New Object};
        \path [line] (render) -- (store);
    \end{paddedtikzpicture}   
    \caption{A functional representation for storing new building and meter objects to the database.} 
\end{figure}

\paragraph{Create Graph}
\begin{figure}[H] 
    \begin{paddedtikzpicture}[node distance = 2cm]
        % Place nodes
        \node (graph)[block] {Create Graph}; 
        \path (graph.east)+(2,0) node (get) [datablock] {Get Data from Parent Block};
        \path [line] (graph) -- (get);
        \path (get.east)+(2,0) node (fetch) [datablock, text width=8em] {Fetch Graph Template};
        \path [line] (get) -- (fetch);
        \path (fetch.east)+(2,0) node (render) [subblock, text width=8em] {Serve Graph to HTML};
        \path [line] (fetch) -- (render);
    \end{paddedtikzpicture}   
    \caption{A functional representation for building a graph in a block.} 
\end{figure}
To render a chart or graph to the webpage, our application will gather all the parameters from the block component that contains the graph (the building(s), the chart type, and the date range). The the graph component will then fetch a D3 template for its specific chart type, input the data from the block, and serve it to the dashboard. 

\paragraph{Get AcquiSuite Data}
\begin{figure}[H] 
    \begin{paddedtikzpicture}[node distance = 2cm]
        % Place nodes
        \node (data)[datablock] {Receive AcquiSuite Data}; 
        \path (data.east)+(2,0) node (parse) [subblock] {Parse Data to JSON Object};
        \path [line] (data) -- (parse);
        \path (parse.east)+(2,0) node (store) [datablock, text width=8em] {Store Data in Building Object};
        \path [line] (parse) -- (store);
    \end{paddedtikzpicture}   
    \caption{A functional representation for storing AcquiSuite data to a building object.} 
\end{figure}
Timestamped data entries from AcquiSuite Servers will be contained in their respective building objects from which the data was sent. Similar to the rational behind storing blocks in the user object that created them, each building will have an array of meter data sorted by timestamp. This will allow our application to use a service to retrieve the data for any graphs or charts directly from the buildings objects in the relative block. A graph service can then fetch the building data, apply filters (like date ranges), and pass it to a D3 graph to render. This minimizes the amount of total data to be parsed and collected by narrowing the data to the relative block.

\noindent [\textbf{Note:} Subject to change in later revisions. The methodology of how to receive data from  AcquiSuite data acquisition servers is still being designed.]

\subsubsection{Design Rationale}
The system's architecture revolves around a heavily modular system where individual components have their own functionality and are able to work cohesively with other components. The modular architecture also breaks up the back-end components into smaller services which simplifies functionality across the entire application.

\subsection{Component Design}
This subsection is designated to describing individual components in greater detail, including the type of element, designated functionality, and the reasoning behind their design. 
Overall, this subsection aims to provide an in-depth view of the application components and describe their individual roles in the application as a whole.
    \subsubsection{Block}
        A Block is a component in our application that will act as a container for a graph, filter options, and provide information about the data in the graph.
        \paragraph{Element} 
            A Block is an HTML element that uses the Bootstrap 4 ``card'' component for its design. It will have a title bar with the Block's name, then a content container split into a grid. 
            On the left, the Block will display the building(s) being graphed, contain a column of filter buttons, and display the D3 graph in the right column. 
        \paragraph{Functionality}
            The Block component neatly organizes data for the user by encapsulating the data, filters, and graph components into one container. 
            A Block needs to pull data from the ``building'' objects that it contains, gather data from any filters, and pass all this data into the correct graph template to generate a D3 graph. 
            There will be a service within the Block's AngularJS controller that will gather this data and pass the desired output to the dashboard.
        \paragraph{Rationale} 
            The Block component is an efficient way to create a reusable HTML template that can be reproduced throughout the application. 
            This provides a simple and consistent process for displaying graphs and data. Additionally, the block component yields a compact solution for organizing data from multiple sources. 
    
    \subsubsection{Dashboard}
        The dashboard component acts a way to store collections of block objects based on user preference.
        \paragraph{Element} 
            A dashboard is an object in the database that holds a collection of block objects.
        \paragraph{Functionality}
            The dashboard object is a way to conveniently package blocks together into groups so users can reuse the same groups or create unique collections of different blocks. 
            When ``viewing'' a dashboard, the application will simply iterate through the block objects and render them to the view container in a vertical list.
        \paragraph{Rationale} 
            The dashboard components provides a simple way to organize data keep track of user generated collections. 
            The dashboard component will work nicely with AngularJS to render an HTML element for every object in an array, which in this case are the blocks in the dashboard.

    \subsubsection{Login}
        The Login component is an HTML button that will allow users to login with a Gmail account by redirecting the user to Google's oAuth 2.0 API.
        \paragraph{Element}
            The Login function will be a button in the corner of the screen or somewhere in the navbar. The button will change to ``logout'' if the user is already logged in.
        \paragraph{Functionality}
            Clicking the Login component will redirect the user to our authentication service. 
            The application will use ``PassportJS'' as an authentication middleware which authenticates the user with Google oAuth 2.0 tokens and sets the req.user variable upon redirecting to the specified route. 
            Once a user is logged in, they will have access to additional application functionality and navigation items based on role.
        \paragraph{Rationale}
            From a developer standpoint, requiring users to login for certain features will make separation of privileges much easier. With authentication, users will also be able to have the convenience of Google SSO.

    \subsubsection{Navigation Bar}
        The Navigation bar is a commonly used UI feature that provides users the ability to navigate through different webpages of the application.
        \paragraph{Element}
            The navigation bar will be an HTML element with differentiable ``nav-items'' that render different views to the content view space. There will be a fixed-top navigation bar for generic item and a side bar navigation for more explicit functionalities such as admin controls. 
        \paragraph{Functionality}
            Clicking on a nav-item has one of two functionalities: 
            \begin{itemize}
                \item Redirects the user to a new webpage by having the route's get function perform a redirect.
                \item Renders a new view to the webpage by having the route's get function render a new HTML view.
            \end{itemize}
            The navigation bar shows which item is currently being served by changing the style of the active nav-item to be different than the rest.
        \paragraph{Rationale}
            From a user perspective, having a navigation bar to navigate around the main features is an extremely useful UI design decision. 
            From a design perspective, having a dedicated area to insert useful links is much more streamline than listing them. 
            The navigation bar is a very useful tool for serving different content to the view space statically.
    
    \subsubsection{Graph Component}
        Our application will contain a graph component that accepts a graph type and data as input and constructs a graph using the D3 visualization framework.
        \paragraph{Element}
            A graph component will be an SVG element, created with D3, that is served to the webpage inside of a block. 
        \paragraph{Functionality}
            The graph component's functionality involves gathering different input parameters from the parent block component, fetching a template based on the graph type, and creating a unique graph. 
            The graph component should have its own service to handle all the data organization and create a list of uniform parameters than can be passed to a D3 graph template.
        \paragraph{Rationale}
            Having a structure like this simplifies the act of generating graphs by having a series of templates that can be fed data in a specific format. 
            The graph service can take all the data and different filter parameters from the block object and format it into the right structure. 
            All the computationally heavy functionality is black-boxed in the graph service and keeps the functionality handled by the user and front-end at a minimal. 
            Another benefit provided by the graph component is that it can change the graph dynamically. 
            Our application may include filters in the block component like a date range or a filter to change the graphed variable, which can trigger the graph service to reconstruct the data and create a new graph to serve based on the applied filters.
    
        \subsubsection{Story}
        A story is a container for user dashboards to be stored and retrieved. A story is an abstract object that allows the user to create unique collections of dashboards with general theme or purpose.
        \paragraph{Element}
            A story will be an database object that contains collections of dashboards.
        \paragraph{Functionality}
            Each story will have a collections of dashboards. The number of dashboards in the collection will vary from story to story based on user preference.
    
    \subsubsection{Content View}
        The content view is a portion of the webpage where different views can be rendered to provide different content.
        \paragraph{Element} 
            The content view will be an HTML container that resides within the context of the webpage. The user will be able to change the content of the container by selecting different navigation items.
        \paragraph{Functionality}
            When the user clicks on a button or list on the side bar, it will pass a route to the scope variable in the view container. Once that route is called, the Angular function ``ng-include'' will inject the rendered HTML into the context of the webpage.
        \paragraph{Rationale} 
            From a developer standpoint, having a content view will dramatically reduce clutter both in the code and in the UI. The content view allows the application to serve dynamic content without reloading the webpage and creates a simple way to perform navigation services.



\subsection{Human Interface Design}
\subsubsection{Overview of User Interface}

\paragraph{Public User}
A public user is a user without a registered account or a user that is not logged in to their registered account. The authentication of a user is based on if 
they have a registered Oregon State Gmail domain. Public users will have limited access to features and data on the website. For example, the left side navigation
bar will be hidden from a public user. Public users will not be able to create personalized dashboards or stories. Public users will be allowed to view public 
user generated dashboards as well as general building dashboards. Essentially, public users will not be able to create personalized data sets or modify elements of the site but will be 
allowed to view general public data. 
\paragraph{Authorized User}
An authorized user is a user with a registered Oregon State Gmail account but does not have administrative privileges. When an authorized user is logged into their account, 
they have greater privileges than a public user as well as all privileges granted to a public user. They are allowed access to a left side navigation 
bar that has special functionality. Unique functionality granted by the vertical navigation bar include: viewing of personal and public dashboards, viewing of personal and public
stories, and access to personal settings. Authorized users may create personalized dashboards and stories for unique purposes.

\paragraph{Administrative User}
Administrative users will have the highest privileges and control of the website. They have all privileges as a public or authorized user. Administrative users
will be able to add or remove buildings, meters, public dashboards, and public stories as well as perform general administration tasks.


\subsubsection{Screen Images}

\paragraph{Home Webpage (Public Access)} 

\textbf{\textit{\hyperref[sec:home_public]{\textcolor{red}{link: See Appendix B for image}}}}\\ 

If a general public user does not have an account but wishes to access energy monitoring data, they will
have a unique experience with the site that is different than those who are logged in to authorized accounts. 
For instance, they will not have the ability to create personal dashboards or stories, but will be allowed to view
public stories, public dashboards, and general building dashboards. They will not have access to the same navigation 
bar on the left side of the screen as a logged in user. 

\paragraph{Home Webpage (Logged in Access)}

\textbf{\textit{\hyperref[sec:home_logged]{\textcolor{red}{link: See Appendix B for image}}}}\\

If an authorized user wishes to access energy monitoring data, they will be able to have special privileges.
For instance, they will have the ability to create personal dashboards or stories which they may specify as private or public.
They will have access to a navigation bar on the left side of the screen which allows for unique data viewing and privileges. 

\paragraph{Buildings Webpage} 

\textbf{\textit{\hyperref[sec:buildings]{\textcolor{red}{link: See Appendix B for image}}}}\\ 

Users may select the ``Buildings'' tab in the navigation bar to bring up an array of building cards to choose from.
Every building on campus that is gathering metering data associated with this application will have a card in this array.
A user may select a building card to bring up a generalized dashboard of information for each specific building. 
The user will be able to search for buildings by name in a search bar. Administrators will be able to add buildings as the metering system expands to new buildings.

\paragraph{Selected Building Page}

\textbf{\textit{\hyperref[sec:selected_buildings]{\textcolor{red}{link: See Appendix B for image}}}}\\  

Once a user has selected a building card, a generalized dashboard will be displayed to them.
All buildings associated with this application will contain the same generalized dashboard to be viewed.
These dashboards will display various cards of data detailing the most general information about their respective buildings.
A user may full screen the application to remove menu clutter and allow for a more presentable display of the data. 
Users will have the option to modify personal dashboards or public dashboards by adding a card, adjusting the organization of cards, or editing the information on the cards.

\paragraph{Dashboards Page}

\textbf{\textit{\hyperref[sec:dashboards]{\textcolor{red}{link: See Appendix B for image}}}}\\ 

Users may select the ``Dashboards'' tab in the navigation bar to bring up a list of user-generated dashboards.
The dashboards will contain user specified arrangements of data cards. 
The list of dashboards will detail the name of each dashboard, the user that created each dashboard, and the date and time of when each was created.
The user will be able to search for dashboards by name in a search bar. 
A user may select a dashboard from the list to view or use the options menu for more specific tasks.

\paragraph{Selected Dashboard Webpage}

\textbf{\textit{\hyperref[sec:selected_dashboards]{\textcolor{red}{link: See Appendix B for image}}}}\\

Once a user has selected a dashboard, a user-generated dashboard will be displayed to them.
The dashboard will display various cards of data presenting information about various buildings energy usage.
A user may full screen the application to remove menu clutter and allow for a more presentable display of the data.

\paragraph{Story Webage}

\textbf{\textit{\hyperref[sec:story]{\textcolor{red}{link: See Appendix B for image}}}}\\ 

Users may select the ``My Stories'' tab in the navigation bar to bring up a drop down list of their personal stories in the left side navigation bar.
Each story will contain a list of dashboards that was generated by the current user to their preference. 
The list of stories will include both their private and public stories.
A user may select a story from the list to view or use the options menu for more specific tasks.
include the original document, showing what you thought, at the time, was the project design
Add (your client should have okay'd): What design aspects were changed, deleted or added? Why? 
\section{Tech Review (original)}
\subsection{Daniel Technology Review Overview}
\subsubsection{Introduction}
My role in this project will be integrating the front-end with the back-end framework and acting as a dev ops engineer. I want to take on the role of dev ops for this project to hone in on certain skills that have not previously been targeted in my past web-development internships. I have created UX/UI and built pages/forms that  manipulate database data. I have written back-end model functions to retrieve data for use in an application. Now I want to manage the full-fledged construction of a full stack web application. Another motive for taking a dev ops role is with a framework like NodeJS, there are endless amounts of libraries and frameworks that you can utilize in your application that can simplify every component. What I want to take away from this course is a well-rounded knowledge about how different components of an application work together and how to integrate each part efficiently.\\
By taking a dev ops role in this project, I will be able to:
\begin{itemize}
    \item Transition away from the tasks I've been previously trained on.
    \item Help my teammates understand things that I have previously learned by not implementing them myself.
    \item Develop a more comprehensive understanding of NodeJS, Angular, Express, and NOSQL Databases.
    \item Take lead in developing a high-end development environment for my team and I in order to create efficient workflow.
    \item Implement components of our web application that are new to me.
\end{itemize}

There are a lot of different components that go into making a web application work and it takes a high-level understanding to implement them all. Our web application will act like an admin dashboard that displays specific charts and graphs based on user preference.\\
Firstly, from a dev ops perspective, we need an architecture that can host a NodeJS application from a single VM. We will be using Amazon Elastic Compute Cloud (EC2) to host our web server. This EC2 instance will listen for inbound requests and transfer them to our Node server daemon. Our Node server will then have to communicate with our MongoDB to retrieve the necessary data.\\
Secondly, we will need a workflow architecture for continuous deployment that can be utilized by all members of the team. We will use Github to create a ``Git-flow'' environment that proceeds as follows:
\begin{itemize}
    \item The repository master branch is considered to be ``production.''
    \item Developers will create ``feature branches'' branched from master.
    \item Feature branches will be pull requested, reviewed, and merged into the master branch.
\end{itemize}
Git-flow allows team members to work on separate features while building off of the current working application. Feature branches make sure that the master branch is always in working condition and no new code gets merged without review and confirmation that it works.\\
Thirdly, our web application also has users with different roles, which adds a lot of overhead to the application. Our application needs to have user authentication, user sessions, user roles, and code to enforce it all. Using Google's oAuth 2.0 authentication, our application will force users to log in with their Google accounts and Google will handle most of the overhead. When a user logs in, ``the Google Authorization Server sends [the] application an access token'' which is then stored in the database instead of a user password \cite{oauth}. Then, through utilizing the Passport.js authentication middleware, we can create user sessions and manage authentication with this Google token and not have any security issues with passwords.\\
The final main component of our web application is the actual functionality of providing data visualizations to the user in a clean and dynamic way. This will include retrieving and parsing data from Acquisuite data acquisition servers, storing this information in the database, and creating dynamic graphs and charts using a visualization framework that can be manipulated by the user. This will involve developing filters and date ranges that can be applied to all the data to select sub-portions of data based on specific buildings, collections of certain buildings, or time periods. Having a visualization framework that works well with AngularJS and our MongoDB database is essential in creating a smooth, dynamic web application. 
%%%%%%%%%%%%%%%%%%%%%%%%%%%%%%%%%%%%%%%%%%%%%%%%%%%%%%%%%%%%%%%%%%
%%%%%%%%%%%%%%%%%%% Visualization Frameworks %%%%%%%%%%%%%%%%%%%%%
%%%%%%%%%%%%%%%%%%%%%%%%%%%%%%%%%%%%%%%%%%%%%%%%%%%%%%%%%%%%%%%%%%
\subsubsection{Visualization Frameworks}
Our web application will provide near-real time data visualizations for energy consumption by buildings on campus. This application will need to dynamically create charts and graphs based on energy data from the database. A key to choosing a visualization library will be to find one that can be dynamically created and changed as new data is received from the data acquisition servers, and the ability to create chart templates that can be reused on multiple pages with different input parameters.\\
\textbf{Basic Criteria}
\begin{itemize}
\item Handle dynamic data. 
\item Allow the user to manipulate data by applying filters or selecting data points. 
\item Works well with Angular framework.
\item Templatable.
\end{itemize}
\paragraph{D3.js}
%https://www.dashingd3js.com/d3-resources/d3-and-angular
\textit{Repository Commits: 4,104}\\  
\textit{Contributors: 120}\\
D3 allows you to bind arbitrary data to a Document Object Model (DOM), and then apply data-driven transformations to the document \cite{d3_home}. It is extremely well documented and widely used. D3 has a high learning curve for beginners to take head-on, but has a wide array of different visualizations and customization.\\
\textbf{Pros}
\begin{itemize}
\item A lightweight, versatile javascript library that uses ``SVG to create the graphical elements'' and appends them to DOM elements \cite{Schwartz}. 
\item Makes use of javascript functions and DOM controlling functionality to dynamically change the content of the page. 
\item Provides a lot of variety and ability to customize graphics.
\item Widely used and there is a lot of documentation and resources available to assist the learning and development processes.
\end{itemize}
\textbf{Cons}
\begin{itemize}
\item D3 is essentially an API to to manipulate SVG, it is not a charting library in of itself \cite{Sun}.
\item You cannot easily pass a data set into a specified chart type like other libraries.
\item A large amount of ``up-front investment in time to get a handle on the D3 language''\cite{Jacobson}.
\item Angular and D3 both attempt to control the DOM and so you have to find a way to make the two work together which is counter intuitive to both framework's APIs. 
\end{itemize}
\paragraph{Vis.js}
\textit{Repository Commits: 3,165}\\
\textit{Contributors: 137}\\
Vis.js is a lightweight charting library that allows users to create clean charts from dynamic data sets \cite{vis}. It is responsive and allows for interaction with the data on the page. Vis.js is praised for it's network chart capabilities but is limited in the number of different modules you can create.\\
\textbf{Pros}
\begin{itemize}
\item Easy to use and less of a learning curve than D3.
\item Allows for interaction and manipulation of data on the chart.
\item Able to handle large amounts of dynamic data.
\item Really clean and nice looking graphics.
\end{itemize}
\textbf{Cons}
\begin{itemize}
\item Limited amount of possible chart types.
\item Does not have built in heat map.
\end{itemize}
\paragraph{Chart.js}
\textit{Repository Commits: 2,465}\\ 
\textit{Contributors: 236}\\
Chart js is a very lightweight library that provides 6 chart types and fully responsive designs. ChartJS is well documented and easy to use, but lacks in variety.\\
\textbf{Pros}
\begin{itemize}
\item Uses HTML5 canvas element.
\item Allows for easy creating based on chart type specification.
\item Library provides Line Charts, Bar Charts, Radar Charts, Pie Charts, Polar Area Charts, and Doughnut Charts.
\item Very responsive charts based on screen width.
\item Simple API, easy to use.
\end{itemize}
\textbf{Cons}
\begin{itemize}
\item Limited amount of possible chart types.
\item Does not have built in heat map.
\end{itemize}
\paragraph{Conclusion}
In conclusion, despite the steep learning curve associated with D3.js we think it will be the best option for our web application. It has the widest range of available graphs to accommodate all the client's requirements and desired visualizations. There are also a number of wrapper libraries available for D3.js like DC.js and dimple.js to help create charts from D3. This is a great way to get around the inconveniences and downsides to D3.js and reap the benefits of all the other charting libraries. Another benefit to using D3 is the extensive amount of templates, examples, and documentation that exists to help guide the process and implementation of our application.\\
It is possible that our decision will change as we begin creating the web application and find other libraries to fit our needs. There are a lot of AngularJS libraries that specify unique directives for implementing different charting libraries and we might find one throughout the midst of our development that handles all of our needs swiftly and effectively. One of the main factors that guided our decision towards D3 was the ability to create heatmaps, which was a client requirement. If this requirement falls through, or we simply use D3 only for heatmaps, we may be able to implement a more lightweight charting library to achieve better simplicity.
%%%%%%%%%%%%%%%%%%%%%%%%%%%%%%%%%%%%%%%%%%%%%%%%%%%%%%%%%%%%%%%%%%
%%%%%%%%%%%%%%%%%%%%%%%% Authentication %%%%%%%%%%%%%%%%%%%%%%%%%%
%%%%%%%%%%%%%%%%%%%%%%%%%%%%%%%%%%%%%%%%%%%%%%%%%%%%%%%%%%%%%%%%%%
\subsubsection{Means of Incorporating Authentication}
Our web application will have an authentication layer which will allow users to register for our application and design their own ``stories.'' Our application will also authenticate user roles so that administrative users will have access to exclusive parts of the application. We want a simple way of authenticating users, while keeping personal information safe.\\
\textbf{Basic Criteria}
\begin{itemize}
\item Remember users across web pages (sessions). 
\item Allow users to be added to the database. 
\item Protect sensitive information like passwords.
\end{itemize}
\paragraph{Building Our Own Authentication Layer}
Node.js has a lot of helpful modules and packages that allow you to create your own password hashing functions and generate a custom authentication layer. There is a ``crypto'' module that is included in Node.js that ``provides cryptographic functionality that includes a set of wrappers for OpenSSL's hash, HMAC, cipher, decipher, sign and verify functions''\cite{Crypto}. In addition to the crypto module, there are numerous Node.js extension modules that perform different hashing functions and provide the same functionality as the native crypto module. There are modules like \href{https://www.npmjs.com/package/bcrypt-Node.js}{\textit{bcrypt}} and \href{https://www.npmjs.com/package/scrypt}{\textit{scrypt}} that use different hashing algorithms to salt and hash password into a fixed length string to be stored into the database.
\begin{lstlisting}[
    caption={[An example of how to salt hash passwords using the Node.js crypto module]An example of how to salt hash passwords using the Node.js crypto module (Taken from \href{https://ciphertrick.com/2016/01/18/salt-hash-passwords-using-Node.js-crypto/}{\textit{Rahil Shaikh's example}})\cite{Rahil_Shaikh}}
    ]
    'use strict';
    var crypto = require('crypto');
    /**
     * generates random string of characters i.e salt
     */
    var genRandomString = function(length){
        return crypto.randomBytes(Math.ceil(length/2))
                .toString('hex') /** convert to hexadecimal format */
                .slice(0,length);   /** return required number of characters */
    };
    /** hash password with sha512.
     * @function
     * @param {string} password - List of required fields.
     * @param {string} salt - Data to be validated.
     */
    var sha512 = function(password, salt){
        var hash = crypto.createHmac('sha512', salt); /** Hashing algorithm sha512 */
        hash.update(password);
        var value = hash.digest('hex');
        return {
            salt:salt,
            passwordHash:value
        };
    };
    function saltHashPassword(userpassword) {
        var salt = genRandomString(16); /** Gives us salt of length 16 */
        var passwordData = sha512(userpassword, salt);
        console.log('UserPassword = '+userpassword);
        console.log('Passwordhash = '+passwordData.passwordHash);
        console.log('nSalt = '+passwordData.salt);
    }
\end{lstlisting}
\textbf{Pros}
\begin{itemize}
    \item Creating our own authentication layer would provide us full control over how passwords are hashed and stored into the database. 
    \item Provide understanding of every component that goes into our application's authentication.
    \item Do not have to rely on another API to authenticate users.
\end{itemize}
\textbf{Cons}
\begin{itemize}
    \item Laborious work and very time consuming.
    \item A lot of room for error and the possibility of data being compromised.
\end{itemize}
\paragraph{Outsourcing Authentication to Google}
Google oAuth 2.0 is a Google API that authenticates users by signing in with their google accounts. There is a lot of documentation about how to integrate Google's authentication API with Node.js. \href{http://www.passportjs.org/}{\textit{Passport}} is authentication middleware for Node.js that provides simple authentication using ``A comprehensive set of strategies support authentication using a username and password, Facebook, Twitter,'' and Google \cite{passport}.
\begin{lstlisting}[
    caption={[Using passport to include Google oAuth 2.0 authentication]Using passport to include Google oAuth 2.0 authentication. (Taken from \href{http://www.passportjs.org/docs/username-password}{\textit{Passport Documentation}})\cite{passport_doc}}
    ]
    var passport = require('passport');
    var GoogleStrategy = require('passport-google-oauth').OAuth2Strategy;
    
    // Use the GoogleStrategy within Passport.
    //   Strategies in Passport require a `verify` function, which accept
    //   credentials (in this case, an accessToken, refreshToken, and Google
    //   profile), and invoke a callback with a user object.
    passport.use(new GoogleStrategy({
        clientID: GOOGLE_CLIENT_ID,
        clientSecret: GOOGLE_CLIENT_SECRET,
        callbackURL: "http://www.example.com/auth/google/callback"
      },
      function(accessToken, refreshToken, profile, done) {
           User.findOrCreate({ googleId: profile.id }, function (err, user) {
             return done(err, user);
           });
      }
    ));
\end{lstlisting}
\textbf{Pros}
\begin{itemize}
    \item Takes all the security risks out of creating our own authentication.
    \item Saves a lot of development time. 
    \item Widely used with a lot of documentation.
    \item Relieves the need to store user passwords in the database.
    \item Can use the passport authentication middleware to simplify authentication even further.
\end{itemize}
\textbf{Cons} 
\begin{itemize}
    \item Limits users to having a google account.
    \item Relies on Google API to be working and running.
    \item Adds dependencies to the project.
\end{itemize}
\paragraph{Use CAS (Central Authentication Service)}
CAS is an authentication process that redirects users to a CAS login page like the one that Oregon State University uses for most things like Canvas, Box, and MyDegrees.\\

CAS uses sessions and a CAS Client server to authenticate users throughout a web application:
\textit{When CAS redirects the authenticated user back to your application, it will append a \{ticket\} parameter to your url.}
The ticket returned to your application is opaque, meaning that it includes no useful information to anyone other than the CAS Server. The only thing that your application can do is send this ticket back to CAS for validation.
CAS will then either respond that this ticket does not represent a valid user for this service, or will acknowledge that this ticket proves authentication. In the later case, CAS will also supply the user's NetID so that you know the identity of the user \cite{how_cas_works}.

\textbf{Pros}
\begin{itemize}
    \item Can use the OSU official CAS to provide a nice theme.
    \item Keeps the same centralized CAS server as the rest of Oregon State University's application.
\end{itemize}
\textbf{Cons}
\begin{itemize}
    \item Relies on Oregon State University's CAS server to be operational. 
    \item Adds exteraneous code and implementation.
    \item Not as modular as the Node.js/Passport implementations of oAuth 2.0.
\end{itemize}
\paragraph{Conclusion}
After reviewing the different means of including authentication for our web application, we think that using the Passport middleware to implement Google's oAuth 2.0 API would be the best option. It can be easily integrated into Node.js applications, everyone at Oregon State University has a Gmail account, and it removes the need to store any passwords into our database. The Google oAuth 2.0 authentication process uses a token that is received from the Google API that we will store along with the user ID and name into our database.\\
Another large benefit to using the Passport.js and Google oAuth 2.0 authentication process is that passport handles most of the overhead involved with user sessions and per-page authentication. Using a small cookie set that contains the user id, ``Passport will serialize and deserialize user instances to and from the session'' to handle login sessions throughout the application \cite{passport_session}.
\begin{lstlisting}[
    caption={[The user ID is serialized to the session, keeping the amount of data stored within the session small. When subsequent requests are received, this ID is used to find the user, which will be restored to req.user.]The user ID is serialized to the session, keeping the amount of data stored within the session small. When subsequent requests are received, this ID is used to find the user, which will be restored to req.user. (Taken from \href{http://www.passportjs.org/docs}{\textit{Passport Documentation}})\cite{passport_session}}
    ]
    passport.serializeUser(function(user, done) {
        done(null, user.id);
      });
      
      passport.deserializeUser(function(id, done) {
        User.findById(id, function(err, user) {
          done(err, user);
        });
      });
\end{lstlisting}

Using Google authentication and the Passport middleware will drastically simplify the authentication process of our application and keep the amount of sensitive data in our database to a minimum.
%%%%%%%%%%%%%%%%%%%%%%%%%%%%%%%%%%%%%%%%%%%%%%%%%%%%%%%%%%%%%%%%%%
%%%%%%%%%%%%%%%%%%%%% Front-end Framework %%%%%%%%%%%%%%%%%%%%%%%%
%%%%%%%%%%%%%%%%%%%%%%%%%%%%%%%%%%%%%%%%%%%%%%%%%%%%%%%%%%%%%%%%%%
\subsubsection{Front-end Framework}
Our web application will be a series of dashboards to display energy data based on different buildings and subsets of buildings on Oregon State University's campus. A key part to designing a clean dashboard is having a well-spaced grid-like layout with different charts and graphs to display a multitude of different data sets and trends. Rather than customizing classic html elements and using div containers to space our dashboard, we wanted to look into bootstrapped dashboard templates that allow easy customization and clean-looking results.\\ 
\textbf{Basic Criteria}
\begin{itemize}
\item Integrates well with Angular and the entire MEAN stack. 
\item Components can be easily integrated with Visualization Framework. 
\item Customizeable.
\end{itemize}
\paragraph{CSS Bootstrap}
\textit{Repository Commits: 17,255}\\ 
\textit{Contributors: 953}\\
CSS Bootstrap is the most widely used CSS framework available. It has the most expansive component list and helps developers create clean and beautiful UX/UI in minimal time frames. Bootstrap allows developers to get their publications up and running quickly without having to worry about front-end styling. A major issue with Bootstrap is that it tends to look very similar across applications and leaves little room for simple customization.\\
\begin{lstlisting}[
    caption={[An example of using UI Bootstrap to create a datepicker object with an Angular controller.]An example of using UI Bootstrap to create a datepicker object with an Angular controller. (Taken from \href{https://codepen.io/joe-watkins/pen/KsAgp}{\textit{Angular - Bootstrap UI - Datepicker}})\cite{Watkins}}
    ]
    // HTML Declaration
    <input type="text" class="form-control" 
        datepicker-popup="{{format}}" 
        ng-model="dt" 
        is-open="opened" 
        min-date="minDate" 
        max-date="'2015-06-22'"
        datepicker-options="dateOptions" 
        date-disabled="disabled(date, mode)" 
        ng-required="true" 
        close-text="Close" 
        id="date-picker" 
        readonly
        ng-click="open($event)"
    />

    // Angular Controller
    var calPicker = angular.module("calPicker", ['ui.bootstrap']);
    
    calPicker.controller("DatepickerDemoCtrl", ["$scope", function($scope){
        
        // grab today and inject into field
        $scope.today = function() {
        $scope.dt = new Date();
        };
        
        // run today() function
        $scope.today();
    
        // setup clear
        $scope.clear = function () {
        $scope.dt = null;
        };
    
        // open min-cal
        $scope.open = function($event) {
        $event.preventDefault();
        $event.stopPropagation();
    
        $scope.opened = true;
        };
        
        // handle formats
        $scope.formats = ['dd-MMMM-yyyy', 'yyyy/MM/dd', 'dd.MM.yyyy', 'shortDate'];
        
        // assign custom format
        $scope.format = $scope.formats[0];
        
    }]);
\end{lstlisting}
\textbf{Pros}
\begin{itemize}
    \item Extensive component list, responsive design, and built-in Javascript functions.\cite{puranjay_2015}
    \item EFully responsive design.
    \item Huge developer contrbution and maintainence.
    \item Used by major companies like Lyft.com, Vogue.com, Vevo.com, and Newsweek.com. \cite{puranjay_2015}
\end{itemize}
\textbf{Cons}
\begin{itemize}
    \item Unsuitable for small scale projects. \cite{puranjay_2015}
    \item Not good if you want to have large control over UI.
\end{itemize}
\paragraph{Pure CSS}
\textit{Repository Commits: 541}\\ 
\textit{Contributors: 51}\\
Pure CSS is known for its lightness and simplicity. Because it's small, it is very fast loading and makes for a lightweight web application. It is also unique in that it allows modules/components to be downloaded individually, reducing its size even more. Pure CSS is good for small projects that need to get up and running quickly and easily.\\
\textbf{Pros}
\begin{itemize}
    \item Meant for small project to get up and running quickly.
    \item Responsive design by default.
    \item Pure CSS is modular so you can download only the components you need.
    \item The complete module is very small so it is quick loading.
    \item Able to be used complimentary with other frameworks.
\end{itemize}
\textbf{Cons} 
\begin{itemize}
    \item Not as extensive component list as Bootstrap.
\end{itemize}
\paragraph{Foundation}
\textit{Repository Commits: 15,094}\\ 
\textit{Contributors: 959}\\
Framework is the second most popular css framework on the market behind Bootstrap and tends to perform just as well. Unlike bootstrap's very prominent theme, Foundation allows for more customization with the look and feel of web pages \cite{Nick_Pettit}. Foundation also has a very good grid system to make the layout of components clean and responsive \cite{blankenship_2017}.\\ 
\textbf{Pros}
\begin{itemize}
    \item Responsive design by default.
    \item Easier to customize than Bootstrap.
    \item CSS classes are built in. \cite{team_2015}
    \item More unique look than the more-popular Bootstrap.
    \item Good grid implementations with customizable grid layouts.
\end{itemize}
\textbf{Cons}
\begin{itemize}
    \item Less maintained than Bootstrap.
    \item Lack of support.
    \item Higher learning curve than Bootstrap.
\end{itemize}
\paragraph{Conclusion}
A lot of online resources acknowledged the fact that it is hard to make a website not look like Bootstrap when using Bootstrap css. Despite this, we are choosing to use Bootstrap as the main front-end framework for our application. Similar to our reasoning behind choosing D3.js, we would like to use a framework that is heavily supported and well-documented, as it will be easier to answer questions during development. We think that the time spent restyling Bootstrap to make it look unique will not compare to the time saved by utilizing a well-documented framework. There is also a project called \href{https://angular-ui.github.io/bootstrap/}{UI Bootstrap} that works with AngularJS to create directives for each of the Bootstrap components \cite{sevilayha_2015}.

\section{Weekly Blog Posts (all team members' posts)}
formatted nicely and clearly distinct from one another
include all members posts, clearly distinct, clearly organized
\section{Final Poster}
scaled to fit on a color-printed on a single 8.5"x11" paper.
\section{Project Documentation}
\subsection{How does your project work? (Could include the following...)}
\begin{itemize}
\item What is its structure?
\item What is its Theory of Operation?
\item Block and flow diagrams are good here.
\end{itemize}
\subsection{How does one install your software, if any?}
\subsection{How does one run it?}
\subsection{Are there any special hardware, OS, or runtime requirements to run your software?}
\subsection{Any user guides, API documentation, etc.}
This needs to be detailed enough to recreate and/or use your project!
\section{Recommended Technical Resources for Learning More}
What web sites were helpful? (Listed in order of helpfulness.)
What, if any, reference books really helped?
Were there any people on campus that were really helpful?
\section{Conclusions and Reflections (each team member answers all questions individually)}
What technical information did you learn?
What non-technical information did you learn?
What have you learned about project work?
What have you learned about project management?
What have you learned about working in teams?
If you could do it all over, what would you do differently?
Be honest here -- no B.S.

\section{Appendix 1: Essential Code Listings.}
You don't have to include absolutely everything, but if someone wants to understand your project, there should be enough here to learn from. If you worked within a larger project, something like a patch file might be a good way to go.
\section{Appendix 2: Anything else you want to include.}
Photos, etc.
\subsection{Design Document Images}
\subsubsection{Home Webpage (Public Access)}
\label{sec:home_public}
\begin{figure}[H]
    \centering
    \includegraphics[width=14cm]{images/Home_NOT_Logged_In.eps}
    \caption{A mock-up of a home page from a general public user access perspective (not logged in to an account).}
\end{figure}

\subsubsection{Home Webpage (Logged in Access)}
\label{sec:home_logged}
\begin{figure}[H]
    \centering
    \includegraphics[width=14cm]{images/Home_Logged_In.eps}
    \caption{A mock-up of a home page from an authorized user access perspective (logged in to an account).}
\end{figure}

\subsubsection{Buildings Webpage}
\label{sec:buildings}
\begin{figure}[H]
    \centering
    \includegraphics[width=14cm]{images/buildings.eps}
    \caption{A mock-up of the webpage that shows all buildings.}
\end{figure}

\subsubsection{Selected Building Page}
\label{sec:selected_buildings}
\begin{figure}[H]
    \centering
    \includegraphics[width=14cm]{images/Building_Select.eps}
    \caption{A mock-up of a selected building page.}
\end{figure}


\subsubsection{Dashboards Page}
\label{sec:dashboards}
\begin{figure}[H]
    \centering
    \includegraphics[width=14cm]{images/Dashboards.eps}
    \caption{A mock-up page of a list of user generated dashboards.}
\end{figure}

\subsubsection{Selected Dashboard Webpage}
\label{sec:selected_dashboards}
\begin{figure}[H]
    \centering
    \includegraphics[width=14cm]{images/Dashboard_Select.eps}
    \caption{A mock-up of a selected dashboard webpage.}
\end{figure}

\subsubsection{Stories Webpage}
\label{sec:story}
\begin{figure}[H]
    \centering
    \includegraphics[width=14cm]{images/Stories.eps}
    \caption{A mock-up of a list of user generated story Webpages.}
\end{figure}

\subsection{Technology Review Images}

\bibliographystyle{IEEEtran}
\bibliography{refs}
\end{document}