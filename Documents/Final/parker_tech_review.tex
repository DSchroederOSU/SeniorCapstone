\subsection{Parker Technology Review Overview}


\subsubsection{Introduction}
	
	Our scalable web application for monitoring energy usage on campus is an application that utilizes various technologies to accurately store, process, and display
	energy usage of buildings on the Oregon State campus. There are many different components and frameworks by which to build an application for this purpose and
	selection the right technologies for is proper functionality is crucial. Choosing technologies that may be outdated, have flaws specific to the needs of the application, contain
	compatibility issues, etc., could prove to be detrimental to the project as a whole. It is necessary to analyse the various technologies and select
	the technology that will provide the best performance and functionality prior to the creation of the application so that the development process will be fluid and efficient. 
	
	This document will discuss the pros and cons of various specific technologies that may be applicable to the components of our web application. Specifically, this
	document will compare technologies for the database framework of the application, the database host of the application, and the server framework that will be used.
	For the database framework, comparisons will be made between mongoDB, mySQL, and Apache Cassandra. As for our database host, comparisons will be made between AWS EC2, Microsoft Azure Cosmos DB, and Google CE.
	And finally, for our back-end framework, comparisons will be made between Node.js, PHP, and Golang.


\subsubsection{Technology 1: Database Framework}

	For any web application that needs to utilize compilations of data for its processes it is critical to have a reliable and 
	powerful framework for that data. Specifically, for our application we need to select a database that can handle a large amount
	of raw text based data that is presented to us from our energy metering systems. Having a database with proper functionality and
	compatibility with the rest of our technologies is necessary so that our application can request and store data in meaningful ways. 
	
	\paragraph{mongoDB}
	
		MongoDB is a noSQL scalable open source JavaScript based database system. It is a non-relational database and it's documents are stored in json format essentially making all JavaScript data types compatible with this
		database. It is often used in conjunction with Node.js projects because of this relationship with JavaScript. It is widely used because it is a "document" based
		database which essentially allows the data that it stores to be interpreted as objects of data, where you can have levels and sub-levels of data in one data entry.
		For example, you could have 1 entry in mongoDB that represents a store, and within that store object it may have lower levels of data such as products provided that store 
		or employees that work there. It is widely used as the database for many web based applications because it is designed to have useful models for data structures
		such as logs, graphs, account and user profiles, form data, etc. It has been used by many major industries and companies such as MetLife, government websites, Expedia, and more.
		MongoDB is often referred to as new user friendly because it is widely documented and has many tutorials online. 
		
	\paragraph{mySQL}	
	
		MySQL is perhaps one of the most popular open-source relational database management systems, written in c and c++. It is an extremely well documented and mature database system and 
		is used widely by many applications, include web based applications. It is compatible with nearly all operating systems and supports many storage engines. Data 
		using mySQL is stored in tables that may be accessed or manipulated using query requests. It is not a document oriented database system so data may not have 
		various levels of hierarchy within each entry, but each entry may be logically bound to other parts of the database which would be managed by those who
		are designing the database. MySQL is proven reliable, high performance, and easy to use and is used by major company applications such as Facebook, Twitter, YouTube, and much more.

	\paragraph{Apache Cassandra}
	
		Apache Cassandra is an open source linearly scalable storage system that is build to handle large amounts of data. It is build using two core technologies: Google Big Table
		and Amazon Dynamo. It is a noSQL class of database system developed to improve consistency, availability, and partition tolerance of the database. It is designed
		to handle arbitrary data types that may have interconnected relationships to other pieces of data within that database. Because of the way that Cassandra's architecture 
		is designed, it is incapable of having a single point of failure among its nodes. Each node within the database is connected and running concurrently, there is no 
		"master-slave" relationship between nodes and nodes can be added to an existing cluster without taking down the server. This works well for applications that need to 
		have access to a database at all times. Cassandra excels for real world uses such as messaging, fraud detection, recommendation engines, playlists or collections, and more.
		An issue that presents itself is that it is not as widely used as either mongoDB or mySQL, and as a consequence is lacking the resources to learn about it as efficiently
		as either mongoDB or mySQL.
		
	\paragraph{Conclusion}
	
		Our choice for the database framework for our application is mongoDB. Not only is it non-relational, provides us with scalability, and provides adequate functionality, it is
		a JavaScript based framework that is very compatible with the rest of our applications technologies. It is widely popular, relatively easy to learn, and allows us
		to access and store data in ways that will suite our needs for this assignment. While the other database framework technologies that were listed are potentially valid options,
		mySQL is not non-relational and they both are not JavaScript based technologies that fall under the "MEAN stack" development method, which is ultimately what our project will be utilizing. 
		

\subsubsection{Technology 2: Database Host}

	Web based applications and their associated databases are required to be hosted from some machine that is handling all of the processes that are required by the application for it 
	to function and provide dynamic usability. A modern and reliable way to host a website and its database is from cloud services that will host your site with varying criteria
	based on various payment options or plans. We prefer to host our application on a cloud based platform so as to reduce any complications that may arise 
	from hosting our site on a local machine that may not be nearly as reliable, secure, or structurally dynamic as cloud based services. Specifically, we will
	compare and contrast the cloud service providers AWS EC2, Microsoft Azure Cosmos DB, and Google CE and decide which is best to suite our needs.
	
	\paragraph{AWS EC2}
	
		Amazon Web Services (AWS) provides a secure, scalable, and non-constricting web based service for applications which may use the non-relational database that we chose mongoDB. It allows complete control of the application (just as root access 
		to computing capabilities) and scales elastically to the needs of the application. Along with control and scalability, the operating system environment in which the application
		is to be ran on may also be chosen. It is reliably available, secure, and has access to many other AWS computing applications that may be utilized by the project, offering the widest
		range of services to its users. It is designed to be simple to learn and cost effective. Because of AWS many features, it has been an excellent choice for a huge number of applications to be ran on, including
		Netflix, AirBnB, and Lamborghini. Companies have opted to use AWS because of its many great features, specifically cost of use, freedom of control, reliability, and ease to learn. 
	
	\paragraph{Microsoft Azure Cosmos DB}
	
		Microsoft Azure Cosmos DB accomplishes a similar functionality as AWS EC2 but is not nearly as widely used as AWS. It has support for applications based in
		non-relational database services such as mongoDB. Essentially it offers all the functionality of a privately owned server but based in the "cloud", similar to AWS.
		It is open source and scalable, as well as provides security, good pay rates, and access to many of Microsoft Azures cloud computing applications. It has low latency and fast
		I/O performance, which may be scaled up for a larger price. Notable users that rely on this service include Pearson, Ford, and NBC News.  
		In a lot of ways it is very similar to the other two technologies compared here, but Azure is perhaps less reliable than AWS and Google CE, having had a series of outages over the years.
	
	\paragraph{Google CE}
	
		Google Compute Engine (CE) is Google cloud computing equivalent to Microsoft Azure and AWS. It is similar to other cloud computing services in that it supports all of 
		the quality attributes of a good web hosting service, including scalability, security, control, area of distribution, reliability, support for non-relational databases
		such as mongoDB, etc., but it is not nearly as widely used or distributed as AWS. It may have an advantage over the other services with regards to machine-learning capability that an application
		may take advantage of, but in our case we do not require that unique resource. Google CE may be more appealing to niche markets, as it focuses on smaller scale companies.
		Essentially, Google CE is very similar to AWS but is much less widely adopted and does not offer as many resources to use as AWS EC2.
		
	\paragraph{Conclusion}
	
		The database hosting technology that we have opted for is the Amazon based service "AWS EC2". The main reasoning for this decision is that it has a massive
		selection of other applications that may be utilized for our project, more so than either of the other two options. As well as more provided services, AWS EC2 is
		without question the dominant hosting service in the market because of its reliability, security, and scalability. It offers exceptionable payment plans and is
		widely used. Other database hosting services may work fine for our application, but AWS EC2 provides us with more potential services and is more widely trusted by large scale companies. 
	
	
\subsubsection{Technology 3: Back-End Framework}

	A solid back-end framework is a necessity for any scalable application that requires dynamic and fluid processing and presentation of information. While a solid
	database is necessary, it is useless without a proper tool for accessing and manipulating that data efficiently. Our  web application will require present dynamically 
	changing informative entities within the page that will need to neatly and quickly display our data in formats that we desire, so a proper back-end framework technology
	will be crucial to accomplish this. Comparisons will be made between node.js, PHP, and Golang to be used for our application.

	\paragraph{Node.js}
	
		Node.js is an open source cross-platform runtime environment for developing applications using JavaScript. It can be run on various operating systems such as Linux, Windows, or OS X.
		It has a massive open source library of JavaScript libraries that may be easily installed and used by an application. It is extremely fast, scalable, asynchronous and
		event driven. Essentially, this framework can make multiple calls to a server without needing to be hung up on one specific call, it can reference a database dynamically. 
		It is very popular and widely used by many large companies such as eBay, General Electric, PayPal, Yahoo!, and more. It performs exceptionally for applications that need to produce data
		functions quickly and dynamically so as not to refresh a browser to display data. 
	
	\paragraph{PHP}
	
		PHP is a widely used and matured general purpose scripting language. It is perhaps more widely used than JavaScript as it is an older developed scripting language. PHP is perhaps
		simpler than node.js as it does not require a server, simply a php file that may be interpreted by a web server. PHP uses a multi-threaded blocking I/O to achieve parallel task 
		completion whereas node.js uses a single threaded non-blocking I/O model. Many large companies also use PHP for their products, including Facebook, Wikipedia, Flickr, and many
		more big names. PHP is reliable and simple to use, but lacks modern functionality, organization, and resources that node.js has. As stated before, our application is JavaScript
		dependant so it is wise to keep as many technologies as possible within this language.
	
	\paragraph{Golang}
	
		Golang, commonly referred to as "Go", is a more modern back-end framework developed by Google. It is not as widely used as node.js, but it is still and effective choice 
		for many web based applications. Because it is in its early days of development use, it is still a relatively immature framework, but that is not necessarily a bad thing.
		It outperforms node.js with speed and performance of tasks because it is not based on an interpreted high language such as JavaScript. It is used by companies such as Google, 
		Docker, and Dropbox because of its excellence in scalability and concurrency. Large scale data can be effectively managed using Go but it may lack the broad scope of 
		tools that node.js could provide. A major con of utilizing Golang is because of its relative youth. It lacks the wide range of tools that we may
		utilize for our application, it lacks the simplicity of node.js, and it is not a JavaScript based environment such as node.js. 
	

	\paragraph{Conclusion}
	
		Our application is very dependant on our back-end framework being able to dynamically and concurrently manipulate data on our web pages. 
		While PHP and Golang are capable of doing this, they do not do it as simply as node.js. All of the technologies that are compared here are
		capable of accomplishing the simpler tasks that our application needs, such as registering user data and doing simple read and writes to our database,
		but node.js provides a more simplified way to accomplish the more complex tasks and is compatible with the rest of our technologies. Having a JavaScript
		based back-end framework is also a wise choice as a large portion of our technologies used for our application are also JavaScript, reducing complexity of component interaction. 
	
\subsubsection{Conclusion}

	We have made a decision to base our application off of the JavaScript language, and as such we have chosen many of our technologies based on this
	so that they may be compatible and perform without complications. Reducing the number of languages involved in a full stack application reduces the complexity of
	the project and allows development to happen more fluidly. We have opted for JavaScript based technologies because of it's wide use, large amount of
	applications and libraries to use, and overall high-level simplicity. Many of our technologies used, specifically our back-end framework and database framework, 
	will utilize JavaScript based technologies, such as mongoDB and Node.js. Not only do these technologies have sufficient performance, scalability, and scope for our project, it allows the 
	project to be easily compatible within itself. As a matter of our database host, all we need is a sufficient web service that can provide support for a 
	non-relational database system such as mongoDB. The best choice to us is using AWS EC2, which supports non-relational database systems 
	and allows us a vast array of tools provided by Amazon Web Services. 


	







