\subsubsection{Fall 2017}
\paragraph{Week 1}
\begin{adjustwidth}{2.5em}{0pt}
    \vspace{-0.5cm}\textbf{Plans:}
    \vspace{-0.5cm}
    \begin{itemize}
        \item Get a usable MEAN stack app working
        \item Contact and meet with potential client
        \item Submit project preferences
    \end{itemize} 
    \vspace{-0.3cm}\textbf{Progress:}
    \vspace{-0.5cm}
    \begin{itemize}
        \item Met with Jack Woods, got him to email a request for me 
    \end{itemize} 
    \vspace{-0.3cm}\textbf{Problems:}
    \vspace{-0.5cm}
    \begin{itemize}
        \item MongoDB is a pain
        \item Need to learn more about package managers
        \item Need to set up db
        \item Node for routes? Angular has routes too.
    \end{itemize}  
    \vspace{-0.3cm}\noindent\textbf{Summary:}\\
    \noindent This week was an introductory period where we went over the course outline and details in lecture.
    I began to pursue a project that I really wanted which is the Scalable Web Application for Energy on Campus.
    I met with Jack Woods and learned more about the project and requested that he submit a request for me to be on the team.
    I submitted project preferences with the MEAN stack application as my first choice.
\end{adjustwidth}
\paragraph{Week 2}
\begin{adjustwidth}{2.5em}{0pt}
    \vspace{-0.5cm}\textbf{Plans:}
    \vspace{-0.5cm}
    \begin{itemize}
        \item Get mocha js and d3 js incorporated in app
        \item Create archive of helpful/related resources 
    \end{itemize} 
    \vspace{-0.3cm}\textbf{Progress:}
    \vspace{-0.5cm}
    \begin{itemize}
        \item Got mean stack working completely
    \end{itemize} 
    \vspace{-0.3cm}\textbf{Problems:}
    \vspace{-0.5cm}
    \begin{itemize}
        \item Team needs to get up to speed on MEAN stack web development
    \end{itemize}  
    \vspace{-0.3cm}\noindent\textbf{Summary:}\\
    \noindent This week we found out our groups for the project. We met and created a slack group for communication. We are making pushes in the right direction and sharing resources to get everyone on the same page. I built a template mean stack application that is a simple to do list from a tutorial online, it works.
\end{adjustwidth} 
\paragraph{Week 3}
\begin{adjustwidth}{2.5em}{0pt}
    \vspace{-0.5cm}\textbf{Plans:}
    \vspace{-0.5cm}
    \begin{itemize}
        \item Get mocha js and d3 js incorporated in app
        \item Create archive of helpful/related resources 
    \end{itemize} 
    \vspace{-0.3cm}\textbf{Progress:}
    \vspace{-0.5cm}
    \begin{itemize}
        \item Got mean stack working completely
    \end{itemize} 
    \vspace{-0.3cm}\textbf{Problems:}
    \vspace{-0.5cm}
    \begin{itemize}
        \item Team needs to get up to speed on MEAN stack web development
    \end{itemize}  
    \vspace{-0.3cm}\noindent\textbf{Summary:}\\
    \noindent This week was a good push in the right direction. We met with our clients on Tuesday and went over the general idea for the web application.
    I also set up the git hub repo and got everyone invited. Made directories and README. I didn't have much time to work on the test application but have been really dying to build up a template angular app with some d3 incorporated.
\end{adjustwidth} 
\paragraph{Week 4}
\begin{adjustwidth}{2.5em}{0pt}
    \vspace{-0.5cm}\textbf{Plans:}
    \vspace{-0.5cm}
    \begin{itemize}
        \item Get mocha js and d3 js incorporated in app
        \item Finish Final Draft of the problem statement 
        \item Get all requirements from client
        \item LEARN SOME REACT.js
        \item Have Kevin look over requirements
    \end{itemize} 
    \vspace{-0.3cm}\textbf{Progress:}
    \vspace{-0.5cm}
    \begin{itemize}
        \item Kevin accepted the requirements list from Jack 
        \item Almost finished with Problem Statement
    \end{itemize} 
    \vspace{-0.3cm}\textbf{Problems:}
    \vspace{-0.5cm}
    \begin{itemize}
        \item Client is seemingly slow at responding to requests and emails -> need to address him and send him requests sooner so he has more time to respond.
    \end{itemize}  
    \vspace{-0.3cm}\noindent\textbf{Summary:}\\
    \noindent This week we finished the problem statement and sent them to our client Jack. We also requested that jack send us a rough draft of the requirements and features his team is expecting from us by the end of the term. We received these requirements and I met with Kevin to look them over. He said they seemed very reasonable and that we should go forth with minimal consultation/negotiation. I also created a mock up for the requirements document in LaTex.
\end{adjustwidth} 
\paragraph{Week 5}
\begin{adjustwidth}{2.5em}{0pt}
    \vspace{-0.5cm}\textbf{Plans:}
    \vspace{-0.5cm}
    \begin{itemize}
        \item Get mocha js and d3 js incorporated in app
        \item Finish Rough draft of requirements
        \item Possibly make plan of action 
    \end{itemize} 
    \vspace{-0.3cm}\textbf{Progress:}
    \vspace{-0.5cm}
    \begin{itemize}
        \item Meeting with client at 5:30 on Tuesday
        \item Rough draft done
        \item Did not work on template with mocha and d3 
    \end{itemize} 
    \vspace{-0.3cm}\textbf{Problems:}
    \vspace{-0.5cm}
    \begin{itemize}
        \item I couldn't seem to get the Gantt chart latex package to work on os-class so I just added an eps. I will try to figure out how to use the package.
    \end{itemize}  
    \vspace{-0.3cm}\noindent\textbf{Summary:}\\
    \noindent This week we finished the rough draft of the requirements document and began working on the final draft. I plan to mess around with some MEAN stack stuff next week and try to get a d3 example web app going.
\end{adjustwidth} 
\paragraph{Week 6}
\begin{adjustwidth}{2.5em}{0pt}
    \vspace{-0.5cm}\textbf{Plans:}
    \vspace{-0.5cm}
    \begin{itemize}
        \item Get mocha js and d3 js incorporated in app
        \item Finish Final draft of requirements 
        \item Possibly make plan of action 
        \item Start tech review? At least split up components
    \end{itemize} 
    \vspace{-0.3cm}\textbf{Progress:}
    \vspace{-0.5cm}
    \begin{itemize}
        \item Finished final draft of requirements document
        \item Made plan of action for work throughout the term
        \item Split up components for technology review
    \end{itemize} 
    \vspace{-0.3cm}\textbf{Problems:}
    \vspace{-0.5cm}
    \begin{itemize}
        \item Finally found a latex package that worked to create the Gantt chart our requirements document.
    \end{itemize}  
    \vspace{-0.3cm}\noindent\textbf{Summary:}\\
    \noindent This week we finished the requirements document and got it approved by our client. We had a helpful discussion in our TA meeting about generating a plan of action for assignments to follow and how we want to set up our workflow. We like communicating over Slack and have been doing fine working remote. We also began writing our technology reviews.
\end{adjustwidth} 
\paragraph{Week 7}
\begin{adjustwidth}{2.5em}{0pt}
    \vspace{-0.5cm}\textbf{Plans:}
    \vspace{-0.5cm}
    \begin{itemize}
        \item Get mocha js and d3 js incorporated in app
        \item Do the tech review
        \item Review web sockets
        \item Make an app with web sockets that JSONifies?
        \item Get access from client to see the meter data from the AcquiSuites
    \end{itemize} 
    \vspace{-0.3cm}\textbf{Progress:}
    \vspace{-0.5cm}
    \begin{itemize}
        \item Completed websocket personal application
        \item Implemented D3 dynamic charts with websocket application
    \end{itemize} 
    \vspace{-0.3cm}\textbf{Problems:}
    \vspace{-0.5cm}
    \begin{itemize}
        \item I still want to figure out how we are going to get access to the data from the Acquisuite servers, and if we will be able to have it push to our EC2 instance once it is set up. 
        \item I want to get this project started, but I think it all begins with the back end and handling data, which is still black boxed to us right now.
        \item Our client Jack was nice enough to look into what AWS package will work best and handle setting up an EC2 instance with the Office of Sustainability.
    \end{itemize}  
    \vspace{-0.3cm}\noindent\textbf{Summary:}\\
    \noindent This week we had a very helpful client meeting about the data acquisition servers and setting up AWS. On a personal accomplishment, I got D3 working on my web socket application, now I just need to make the d3 bar chart I found update based on the web socket data. We also began the tech review to be finished by Tuesday. 
\end{adjustwidth} 
\paragraph{Week 8}
\begin{adjustwidth}{2.5em}{0pt}
    \vspace{-0.5cm}\textbf{Plans:}
    \vspace{-0.5cm}
    \begin{itemize}
        \item Get mocha js and d3 js incorporated in app
        \item Get websocket data into d3 bar graph
        \item Finish the tech review 
    \end{itemize} 
    \vspace{-0.3cm}\textbf{Progress:}
    \vspace{-0.5cm}
    \begin{itemize}
        \item Completed tech review rough draft
        \item Implemented D3 dynamic charts with websocket application
    \end{itemize} 
    \vspace{-0.3cm}\textbf{Problems:}
    \vspace{-0.5cm}
    \begin{itemize}
        \item I was researching hashing algorithms and authentication measures, but talked to Kevin and my TA who said I should just outsource authentication to google.
        \item Next week I will change my tech review from hashing algorithms to means if authentications.
    \end{itemize}  
    \vspace{-0.3cm}\noindent\textbf{Summary:}\\
    \noindent This week I finished the tech review rough draft and started really researching nitty gritty details about the dev ops behind this project. I spoke to my boss from my last job and got 3 page of notes about dev ops and total web application architecture. (I will up load this notes next week). I did not have very much time to work on personal MEAN stack projects this week as I was very busy with other work.
\end{adjustwidth} 
\paragraph{Week 9}
\begin{adjustwidth}{2.5em}{0pt}
    \vspace{-0.5cm}\textbf{Plans:}
    \vspace{-0.5cm}
    \begin{itemize}
        \item Get mocha js and d3 js incorporated in app
        \item Get websocket data into d3 bar graph
        \item Finish the tech review  
        \item Change authentication section to ``means of authentication'' in tech review
        \item Finish tech review final draft (early?)
    \end{itemize} 
    \vspace{-0.3cm}\textbf{Progress:}
    \vspace{-0.5cm}
    \begin{itemize}
        \item Completed tech review final draft
        \item Changed authentication section to ``means of authentication'' in tech review
    \end{itemize} 
    \vspace{-0.3cm}\textbf{Problems:}
    \vspace{-0.5cm}
    \begin{itemize}
        \item I was researching hashing algorithms and authentication measures, but talked to Kevin and my TA who said I should just outsource authentication to google.
        \item Next week I will change my tech review from hashing algorithms to means if authentications.
    \end{itemize}  
    \vspace{-0.3cm}\noindent\textbf{Summary:}\\
    \noindent This week I finished the tech review final draft early (by 1:00 pm) and emailed to Kirsten. We are still waiting on Jack, our client, to set up EC2 with the IT guys in his office. I worked on a lot of work for other classes to prepare for the design document this upcoming week. My group mates have been nonchalant with the work load and the amount of work that is required for the rest of the term's documents. I had no time to do the personal projects I wanted to do, but freed up time for next week to work on stuff.
\end{adjustwidth} 
\paragraph{Week 10}
\begin{adjustwidth}{2.5em}{0pt}
    \vspace{-0.5cm}\textbf{Plans:}
    \vspace{-0.5cm}
    \begin{itemize}
        \item Get mocha js and d3 js incorporated in app
        \item Finish the Design Document
        \item Make a template for Google auth with passport.js 
    \end{itemize} 
    \vspace{-0.3cm}\textbf{Progress:}
    \vspace{-0.5cm}
    \begin{itemize}
        \item Completed the design document
        \item Made a sample application with Google oAuth2.0 with Passport.js
    \end{itemize} 
    \vspace{-0.3cm}\textbf{Problems:}\\
    Began to think about the full architecture of the application brought up some design constraints that will need to be addressed before implementation. A lot of components and individual model objects will need to share data in order to render the correct graph and content to the page which will require explicit data model designs that are consistent and robust. Creating Angular services to process the data from multiple data models and components will be essential, but our application needs to ensure that these services receive consistent data in the forms it is expecting. An example of this will be the timestamped data entries logged in each building will need to be parameterized the same way for every building and type checking will need to be done in the service. \\

    \vspace{-0.3cm}\noindent\textbf{Summary:}\\
    \noindent This week was a grind through the design document and the cataclysm of other projects and assignments due for other classes. I was able to get a very functional MEAN stack application running that includes Google oAuth 2.0 authentication with passport.js middleware and effectively stores user profiles into user objects and maintains user sessions. I was also able to dive into the realm od new Bootstrap 4 components and design an HTML template for our block components using bootstrap 4 ``Cards'' which are the new take on ``well'' elements. Cards provide a container with a header row and content row that can flex to the size of its elements. Along with the bootstrap grid layout system, the two mingle very well to create the basis of what we need for our dashboards. In all, creating a design template application really allowed me to grasp the different components of the entire application and develop design strategies to make these modular subsystems work together seamlessly.
\end{adjustwidth} 
\subsubsection{Winter 2018}
\paragraph{Week 1}
\begin{adjustwidth}{2.5em}{0pt}
    \vspace{-0.5cm}\textbf{Plans:}
    \vspace{-0.5cm}
    \begin{itemize}
        \item Make plans for the term 
    \end{itemize} 
    \vspace{-0.3cm}\textbf{Progress:}
    \vspace{-0.5cm}
    \begin{itemize}
        \item None
    \end{itemize} 
    \vspace{-0.3cm}\textbf{Problems:}
    \vspace{-0.5cm}
    \begin{itemize}
        \item None
    \end{itemize}  
    \vspace{-0.3cm}\noindent\textbf{Summary:}\\
    \noindent Introductory week. We got settled back into the groove of things and began planning our attack on the project. 
\end{adjustwidth}
\paragraph{Week 2}
\begin{adjustwidth}{2.5em}{0pt}
    \vspace{-0.5cm}\textbf{Plans:}
    \vspace{-0.5cm}
    \begin{itemize}
        \item Set schedule for work
        \item Get repo on Ec2
        \item Continue to work on weekly blogs 
    \end{itemize} 
    \vspace{-0.3cm}\textbf{Progress:}
    \vspace{-0.5cm}
    \begin{itemize}
        \item Made substantial progress on the application. Most progress since break. 
        \item I got authentication working and users stored in the DB based of Google tokens. 
        \item I got the block ``create'' form to auto-populate and move building objects from the drop down to the list view of ``Selected Building.''
        \item Added a demo of create mongoose functions to add buildings to the database and the building schema in our models directory. 
        \item I assigned issues on github and got them synced with our waffle.io agile workflow program.
        \item Began using a .env file for sensitive tokens and id's. Using the dotenv npm library to use them in the code source.
    \end{itemize} 
    \vspace{-0.3cm}\textbf{Problems:}
    \vspace{-0.5cm}
    \begin{itemize}
        \item Would like more confidence from teammates with web development proficiency.
    \end{itemize}  
    \vspace{-0.3cm}\noindent\textbf{Summary:}\\
    \noindent This week was much better than week 1 in terms of progress. We met with our client and set up a good plan for moving forward including using waffle.io for issue tracking and swim lanes. I pushed some good functionality that demonstrates how to go from HTML to controller to service to database and populate the view with the necessary data. Hoping this push will be used as a platform for moving forward. I got Google authentication to work and had our client create a google account for the Office of Sustainability to get a GOOGLE API token to use. Also got secrets and ID's protected in a .env file, good practice. Moving forward I plan to do the same thing I did with create block to create dashboard and create story to lay the foundation for the main components of our application. 
\end{adjustwidth} 
\paragraph{Week 3}
\begin{adjustwidth}{2.5em}{0pt}
    \vspace{-0.5cm}\textbf{Plans:}
    \vspace{-0.5cm}
    \begin{itemize}
        \item Set schedule for work
        \item Get repo on Ec2
        \item D3
        \item Dashboard functionality
        \item Story functionality
        \item Acquisuite data (websocket server)
        \item EC2 establishment.
    \end{itemize} 
    \vspace{-0.3cm}\textbf{Progress:}
    \vspace{-0.5cm}
    \begin{itemize}
        \item Imported building images.
        \item Made mongoose functions for buildings.
        \item Finally figured out the populate mongoose method which gathers documents based on ID in a relational database.
        \item Was able to completely populate block components with array attribute fields and accomplished a nested ng-repeat angular data bind to display all buildings within a block within a users block array. 
    \end{itemize} 
    \vspace{-0.3cm}\textbf{Problems:}
    \vspace{-0.5cm}
    \begin{itemize}
        \item Making MongoDB function as a relation database was more difficult than expected. 
        \item Finally figured it out with hours of reading documentation.
    \end{itemize}  
    \vspace{-0.3cm}\noindent\textbf{Summary:}\\
    \noindent I made great progress on a personal note to our application. Finally figured out the ``populate'' mongoose.js method which gathers subdocuments based on ID. I got building objects to populate in `blocks' finally. It required the mongoose.js function ``populate'' to get the document when stored by id. Functionality includes a nested ng-repeat directive in blocks.html and some database queries in the route handler.
\end{adjustwidth} 
\paragraph{Week 4}
\begin{adjustwidth}{2.5em}{0pt}
    \vspace{-0.5cm}\textbf{Plans:}
    \vspace{-0.5cm}
    \begin{itemize}
        \item Set schedule for work
        \item Get repo on Ec2
        \item D3
        \item Dashboard functionality
        \item Story functionality
        \item Acquisuite data (websocket server)
        \item EC2 establishment.
    \end{itemize} 
    \vspace{-0.3cm}\textbf{Progress:}
    \vspace{-0.5cm}
    \begin{itemize}
        \item Dashboard functionality is working
        \item Our repo is on EC2 and we have a data server receiving POST requests from Acquisuites
        \item Got our client to set up a data server and input our EC2 IP address on a couple of meters around campus. 
        \item We are currently receiving meter data every 15 minutes. 
    \end{itemize} 
    \vspace{-0.3cm}\textbf{Problems:}
    \vspace{-0.5cm}
    \begin{itemize}
        \item Waiting on teammate to create data model for data timestamp model. 
        \item Client is seemingly slow at responding to requests and emails -> need to address him and send him requests sooner so he has more time to respond.
    \end{itemize}  
    \vspace{-0.3cm}\noindent\textbf{Summary:}\\
    \noindent I continued pushing functionality for the major components of the application. Blocks and dashboards are implemented. I'm getting a good grasp on mongoose.js and AngularJS.
\end{adjustwidth} 
\paragraph{Week 5}
\begin{adjustwidth}{2.5em}{0pt}
    \vspace{-0.5cm}\textbf{Plans:}
    \vspace{-0.5cm}
    \begin{itemize}
        \item D3
        \item Dashboard functionality
        \item Story functionality
    \end{itemize} 
    \vspace{-0.3cm}\textbf{Progress:}
    \vspace{-0.5cm}
    \begin{itemize}
        \item Kind of a slow implementation week.
        \item Began on poster and started assigning presentation tasks.
        \item Worked on dashboards and stories. 
        \item Met with Kirsten.
    \end{itemize} 
    \vspace{-0.3cm}\textbf{Problems:}
    \vspace{-0.5cm}
    \begin{itemize}
        \item None
    \end{itemize}  
    \vspace{-0.3cm}\noindent\textbf{Summary:}\\
    \noindent Made good progress towards required assignments this week (i.e. poster and presentation); full confidence we will finish by due dates. Would really like to just get D3 over with. Might assign a day next week to sit down and just DO IT. Almost finished with dashboards (still need a view/delete). Going to begin on stories next week. 
\end{adjustwidth} 
\paragraph{Week 6}
\begin{adjustwidth}{2.5em}{0pt}
    \vspace{-0.5cm}\textbf{Plans:}
    \vspace{-0.5cm}
    \begin{itemize}
        \item D3 
        \item Story functionality
        \item Finish Poster
        \item Finish Presentation
    \end{itemize} 
    \vspace{-0.3cm}\textbf{Progress:}
    \vspace{-0.5cm}
    \begin{itemize}
        \item Did the progress report and presentation. 
        \item Assigned turn in to other partners as I have been handling a lot of the work lately. 
    \end{itemize} 
    \vspace{-0.3cm}\textbf{Problems:}
    \vspace{-0.5cm}
    \begin{itemize}
        \item Video concatenation took longer than expected. Progress report was not uploaded by midnight of due date.
    \end{itemize}  
    \vspace{-0.3cm}\noindent\textbf{Summary:}\\
    \noindent This week was one of the most frustrating weeks of my life. No more to report. 
\end{adjustwidth} 
\paragraph{Week 7}
\begin{adjustwidth}{2.5em}{0pt}
    \vspace{-0.5cm}\textbf{Plans:}
    \vspace{-0.5cm}
    \begin{itemize}
        \item D3
        \item Story functionality
        \item Edit/Update functionality for all forms
    \end{itemize} 
    \vspace{-0.3cm}\textbf{Progress:}
    \vspace{-0.5cm}
    \begin{itemize}
        \item Minor tweaks to application.
    \end{itemize} 
    \vspace{-0.3cm}\textbf{Problems:}
    \vspace{-0.5cm}
    \begin{itemize}
        \item Still waiting for data collection to be completed by partner.
        \item Tried to get edit/create dynamics to work for a long time this week but could not figure out UI router/resolves.
    \end{itemize}  
    \vspace{-0.3cm}\noindent\textbf{Summary:}\\
    \noindent This was a slow week, I began to set deadlines for partners and myself so that we begin pushing out minimum viable features. Need to figure out how to pre-populate web forms when a user wants to ``edit'' a particular component. I looked at AngularJS routing with resolves, but nothing seems to work. 
\end{adjustwidth} 
\paragraph{Week 8}
\begin{adjustwidth}{2.5em}{0pt}
    \vspace{-0.5cm}\textbf{Plans:}
    \vspace{-0.5cm}
    \begin{itemize}
        \item D3
        \item Story functionality
        \item Edit/Update functionality for all forms
    \end{itemize} 
    \vspace{-0.3cm}\textbf{Progress:}
    \vspace{-0.5cm}
    \begin{itemize}
        \item All main components are implemented except for graphs.
        \item I took meters and building relationships into my own hands because they were not getting done. 
    \end{itemize} 
    \vspace{-0.3cm}\textbf{Problems:}
    \vspace{-0.5cm}
    \begin{itemize}
        \item Lack of contributions.
    \end{itemize}  
    \vspace{-0.3cm}\noindent\textbf{Summary:}\\
    \noindent This week was very, very productive! Completed a lot, made plans to complete more, trying to assign simple helpful tasks to partners to just get things done. 
\end{adjustwidth} 
\paragraph{Week 9}
\begin{adjustwidth}{2.5em}{0pt}
    \vspace{-0.5cm}\textbf{Plans:}
    \vspace{-0.5cm}
    \begin{itemize}
        \item D3 --> Chart.js?
        \item Finish data retrieval
        \item Begin Charting
    \end{itemize} 
    \vspace{-0.3cm}\textbf{Progress:}
    \vspace{-0.5cm}
    \begin{itemize}
        \item Data retrieval from Acquisuites seems to be working.
        \item Parker got a chart template into the application using chart.js
        \item Authentication hiding objects on home screen working with ng-hide directive
    \end{itemize} 
    \vspace{-0.3cm}\textbf{Problems:}
    \vspace{-0.5cm}
    \begin{itemize}
        \item Data retrieval has had a lot of hidden bugs and unknowns that has made data collection and aggregation very difficult.
    \end{itemize}  
    \vspace{-0.3cm}\noindent\textbf{Summary:}\\
    \noindent At least we made progress. A lot of Parker and I's work on charts has been hindered by the lack of data retrieval. Data retrieval seems to be in order and we have tons of data in the database, so I'm excited to keep building. The format and process of meter data being sent is largely black boxed to us and our clients, so we have been incrementally fixing problems that arise.
\end{adjustwidth} 
\paragraph{Week 10}
\begin{adjustwidth}{2.5em}{0pt} 
    \vspace{-0.5cm}\textbf{Plans:}
    \vspace{-0.5cm}
    \begin{itemize}
        \item Finish charting
        \item Set up email service
        \item Filter data in graphs
        \item Finish building pages
        \item Export to csv
        \item Finish other small requirements
    \end{itemize} 
    \vspace{-0.3cm}\textbf{Progress:}
    \vspace{-0.5cm}
    \begin{itemize}
        \item I have finally found a solution for view/edit functionality using AngularJS ng-init directive
        \item This allows me to call scope functions before the form and inputs are rendered to DOM.
        \item I wrote my essay early so I plan to turn that In per request by Kirsten
    \end{itemize} 
    \vspace{-0.3cm}\textbf{Problems:}
    \vspace{-0.5cm}
    \begin{itemize}
        \item Data retrieval has had a lot of hidden bugs and unknowns that has made data collection and aggregation very difficult.
    \end{itemize} 
    \vspace{-0.3cm}\noindent\textbf{Summary:}\\
    \noindent We are very close to being feature complete. I'm almost finished on my paper and need to finish features and make demo video. I really hope that Ben (TA) was truthful when he said winter reports will be evaluated and contributions will be taken seriously. This has been a very difficult term.
\end{adjustwidth} 
\subsubsection{Spring 2018}
\paragraph{Week 1}
\begin{adjustwidth}{2.5em}{0pt}
    \vspace{-0.5cm}\textbf{Plans:}
    \vspace{-0.5cm}
    \begin{itemize}
        \item New Data Model for Data Entries
        \item Reformat Design Doc
        \item Buffer date values with 0's
        \item Fix Date Filters for csv
        \item Complete fixes for date filters
    \end{itemize} 
    \vspace{-0.3cm}\textbf{Progress:}
    \vspace{-0.5cm}
    \begin{itemize}
        \item Set TA schedule time
        \item Filtered chart data to get consumption PER day
        \item Filtered data to get 23:45 - 00:00 every day for daily consumption.
        \item Figured out that charts were being sorted when getting added to the chart JS object, fixed.
        \item Began storing Acquisuite meters with appended "address" serial numbers.
        \item Created dynamic getter function for google map polygon coordinates
    
    \end{itemize} 
    \vspace{-0.3cm}\textbf{Problems:}
    \vspace{-0.5cm}
    \begin{itemize}
        \item Did not understand why there were so many meters per Acquisuite server
        \item Date filter system is buggy. Need to fix when a week rolls into another week.
    \end{itemize}  
    \vspace{-0.3cm}\noindent\textbf{Summary:}\\
    \noindent Getting back into the swing of things after spring break. We have new information about the AcquiSuites and how building meters are set up. We will have to begin summing and aggregating data from multiple meters to get total consumption for certain buildings. Not the best news this late in the year.
\end{adjustwidth}
\paragraph{Week 2}
\begin{adjustwidth}{2.5em}{0pt}
    \vspace{-0.5cm}\textbf{Plans:}
    \vspace{-0.5cm}
    \begin{itemize}
        \item New Data Model for Data Entries
        \item Reformat Design Doc
        \item Continue to debug AcquiSuite data and large spikes that are occurring.
    \end{itemize} 
    \vspace{-0.3cm}\textbf{Progress:}
    \vspace{-0.5cm}
    \begin{itemize}
        \item Transfer data aggregation functionality into mongoDB query. % \ref{Appendix A?}
        \item Major debugging of Acquisuite XML data and data in Database.
        \item Expo registration complete, release docs filled out and turned in. 
    \end{itemize} 
    \vspace{-0.3cm}\textbf{Problems:}
    \vspace{-0.5cm}
    \begin{itemize}
        \item Did not understand why there were so many meters per Acquisuite server
        \item There are weird values coming in ever so often from AcquiSuite servers.
        \item AcquiSuite's are pushing data entries with timestamps at XX:XX:01, we have not seen this before.
    \end{itemize}  
    \vspace{-0.3cm}\noindent\textbf{Summary:}\\
    \noindent This week we had a good push in the right direction towards better understanding the data being received by the AcquiSuite's and how we should collect and store them. Our client gave us new information about meter addressing and how the data acquisition servers are set up that is requiring us totally re-write our backend and data retrieval implementations. Additionally, we will need to find ways to sum and subtract a number of meters together to get the correct values for specific buildings. 
\end{adjustwidth} 
\paragraph{Week 3}
\begin{adjustwidth}{2.5em}{0pt}
    \vspace{-0.5cm}\textbf{Plans:}
    \vspace{-0.5cm}
    \begin{itemize}
        \item New Data Model for Data Entries
        \item Reformat Design Doc
        \item Begin WIRED Article
    \end{itemize} 
    \vspace{-0.3cm}\textbf{Progress:}
    \vspace{-0.5cm}
    \begin{itemize}
        \item Began working on the WIRED article for class.
        \item Interviewed friend from Ninkasi Brewery System group
        \item Partitioned McNary data into McNary Complex and McNary Dining through a complex data aggregation process. %\cite{}
    
    \end{itemize} 
    \vspace{-0.3cm}\textbf{Problems:}
    \vspace{-0.5cm}
    \begin{itemize}
        \item Still trying to wrap our heads around AcquiSuite format
    \end{itemize}  
    \vspace{-0.3cm}\noindent\textbf{Summary:}\\
    \noindent This week I spent a good amount of time trying to update our data retrieval process to be more reliable and more efficient. I want to perform summations and date filtering from the MongoDB query itself with Mongo's native aggregation functionality, but it has turned out to be a difficult process. 
\end{adjustwidth} 
\paragraph{Week 4}
\begin{adjustwidth}{2.5em}{0pt}
    \vspace{-0.5cm}\textbf{Plans:}
    \vspace{-0.5cm}
    \begin{itemize}
        \item New Data Model for Data Entries
        \item Reformat Design Doc
        \item Finish WIRED article
        \item Finish Poster
        \item Begin Progress Report
        \item Begin documenting all functions and files
    \end{itemize} 
    \vspace{-0.3cm}\textbf{Progress:}
    \vspace{-0.5cm}
    \begin{itemize}
        \item Finished WIRED
        \item Worked a lot on the new charting algorithms
        \item Summation of multiple meters and filtering by date 
        \item Got the poster done, I like how its evolved.
        \subitem We added screenshots, captions, group photo, and reduced some text into bullet points. 
        \item Now performing "daily" calculations in the backend to reduce computation after request response. 
        \item Took group photo for poster
    Continued working on back-end data aggregation and charting
    \end{itemize} 
    \vspace{-0.3cm}\textbf{Problems:}
    \vspace{-0.5cm}
    \begin{itemize}
        \item Still trying to wrap our heads around AcquiSuite format
    \end{itemize}  
    \vspace{-0.3cm}\noindent\textbf{Summary:}\\
    \noindent I finished my WIRED article and submitted the Poster for final review. We took a nice picture in Kelley engineering, updated some graphics, and edited some of the text. I also updated the building data retrieval service to perform a lot of logic in the query prior to doing manual aggregation after receiving all the data entries. I think this will increase performance.
\end{adjustwidth} 
\paragraph{Week 5}
\begin{adjustwidth}{2.5em}{0pt}
    \vspace{-0.5cm}\textbf{Plans:}
    \vspace{-0.5cm}
    \begin{itemize}
        \item New Data Model for Data Entries
        \item Reformat Design Doc
        \item Finish Progress Report
        \item Complete meter subtraction for monitoring McNary residence hall
    \end{itemize} 
    \vspace{-0.3cm}\textbf{Progress:}
    \vspace{-0.5cm}
    \begin{itemize}
        \item Marked where all requirements are met in our code
        \item Continued working on back-end data aggregation and charting
        \item Submitted code snippets with Carbon to create nice screenshots in presentation
    Made progress report and presentation
    \end{itemize} 
    \vspace{-0.3cm}\textbf{Problems:}
    \vspace{-0.5cm}
    \begin{itemize}
        \item Having difficulties finding an efficient way to perform complex data aggregation
    \end{itemize}  
    \vspace{-0.3cm}\noindent\textbf{Summary:}\\
    \noindent Spent a large portion of this week on the spring midterm and progress report. I continued developing data aggregation queries to match the new format we are receiving and storing data. Instead of performing a lot of for each function over and over to perform subtractions and different filtering, I'm trying to do as much as possible in the Mongo query so that we can reduce runtime. McNary hall is the difficult one to accomplish. One building needs data subtracted from two meters (i.e. it does not have its own energy meter, but rather is the difference between McNary complex consumption and McNary dining consumption). Not exactly the most intuitive calculation to perform in a Mongo query.
\end{adjustwidth} 
\paragraph{Week 6}
\begin{adjustwidth}{2.5em}{0pt}
    \vspace{-0.5cm}\textbf{Plans:}
    \vspace{-0.5cm}
    \begin{itemize}
        \item New Data Model for Data Entries
        \item Reformat Design Doc
        \item Finish for code freeze
    \end{itemize} 
    \vspace{-0.3cm}\textbf{Progress:}
    \vspace{-0.5cm}
    \begin{itemize}
        \item Updated UML for design document
        \item Got entire code source finished for code freeze
        \item Finally finished the McNary hall subtraction issues
    \end{itemize} 
    \vspace{-0.3cm}\textbf{Problems:}
    \vspace{-0.5cm}
    \begin{itemize}
        \item None. Good Week.
    \end{itemize}  
    \vspace{-0.3cm}\noindent\textbf{Summary:}\\
    \noindent This week we finished implementing and testing our code before the code freeze on Friday. We had a couple small bugs and small implementations to finish, including McNary data aggregation, permissions based access, and UI bug fixes. All in all, a very productive week.
\end{adjustwidth} 
\paragraph{Week 7}
\begin{adjustwidth}{2.5em}{0pt}
    \vspace{-0.5cm}\textbf{Plans:}
    \vspace{-0.5cm}
    \begin{itemize}
        \item Prepare for Expo
        \item Get demo video prepared for Expo demonstration
    \end{itemize} 
    \vspace{-0.3cm}\textbf{Progress:}
    \vspace{-0.5cm}
    \begin{itemize}
        \item 	Created Expo demo video
    \end{itemize} 
    \vspace{-0.3cm}\textbf{Problems:}
    \vspace{-0.5cm}
    \begin{itemize}
        \item None. 
    \end{itemize}  
    \vspace{-0.3cm}\noindent\textbf{Summary:}\\
    \noindent This was Expo week, no more need be said.
\end{adjustwidth} 
\paragraph{Week 8}
\begin{adjustwidth}{2.5em}{0pt}
    \vspace{-0.5cm}\textbf{Plans:}
    \vspace{-0.5cm}
    \begin{itemize}
        \item Begin working on final assignment. 
    \end{itemize} 
    \vspace{-0.3cm}\textbf{Progress:}
    \vspace{-0.5cm}
    \begin{itemize}
        \item Nothing.
    \end{itemize} 
    \vspace{-0.3cm}\textbf{Problems:}
    \vspace{-0.5cm}
    \begin{itemize}
        \item None.
    \end{itemize}  
    \vspace{-0.3cm}\noindent\textbf{Summary:}\\
    \noindent Did not work on very much this week. Plan on getting a big push on the final paper and presentation next week. 
\end{adjustwidth} 
\paragraph{Week 9}
\begin{adjustwidth}{2.5em}{0pt}
    \vspace{-0.5cm}\textbf{Plans:}
    \vspace{-0.5cm}
    \begin{itemize}
        \item Begin working on final assignment. 
    \end{itemize} 
    \vspace{-0.3cm}\textbf{Progress:}
    \vspace{-0.5cm}
    \begin{itemize}
        \item Compiled a majority of the final tex assignment
        \item Defined new command for subsubsubsections as we will need them for report
        \item Using ``input'' command and temporary tex files to keep the report modular and cleaner
    \end{itemize} 
    \vspace{-0.3cm}\textbf{Problems:}
    \vspace{-0.5cm}
    \begin{itemize}
        \item Using the bibliography with the temporary files was a struggles. Needed to clear aux files and recompile.
    \end{itemize}  
    \vspace{-0.3cm}\noindent\textbf{Summary:}\\
    \noindent Made a great push towards finishing the final report. Plan on being done by next week and have finals week totally clear.
\end{adjustwidth} 